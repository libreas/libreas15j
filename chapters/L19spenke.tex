\documentclass[output=paper]{langscibook}

\title{Bibliothekarische Berufsethik: Mit welchen Themen befassen sich bibliothekarische Ethikkodizes?}
\author{Julia Spenke}

\abstract{
Julia Spenke, \enquote{Bibliothekarische Berufsethik: Mit welchen Themen befassen sich bibliothekarische Ethikkodizes?}. LIBREAS. Library Ideas, 19 (2011). \url{https://libreas.eu/ausgabe19/texte/01spenke.htm}

URN: \href{http://nbn-resolving.de/urn:nbn:de:kobv:11-100195045}{urn:nbn:de:kobv:11-100195045}

DOI: \url{https://doi.org/10.18452/8987}

Keywords: Bibliothekarische Berufsethik, Ethikkodices

Abstract: Der vorliegende Aufsatz beschäftigt sich mit bibliothekarischer Berufsethik und ihrer Formulierung in Form von Ethikkodizes. Es wird untersucht, welches die Themen bibliothekarischer Ethik sind und ob es ethische Themenstellungen gibt, die für den bibliothekarischen Beruf zentral sind. Grundlage ist eine Auswertung bibliothekarischer Fachliteratur und bibliothekarischer Ethikodizes aus unterschiedlichen Ländern bezüglich deren Kernaussagen zur Ethik für Bibliothekare. Der deutsche bibliothekarische Ethikkodizes, der 2007 verfasst wurde, wird in Bezug auf Entstehung und Inhalt genauer untersucht
}

\begin{document}
\maketitle

%body
\hypertarget{einleitung}{%
\section*{Einleitung}\label{einleitung}}

Einige Berufsgruppen, welche schwerwiegende ethische Entscheidungen zu
treffen haben, verfügen seit Jahren über eine Berufsethik. Die
bekannteste Berufsethik ist wohl das Genfer Gelöbnis des Weltärztebundes
von 1948. Die deutschen Bibliothekare haben seit 2007 einen Ethikkodex.
Mit welchen ethischen Problemstellungen müssen ein Bibliothekar bzw.
eine Bibliothekarin umgehen?

Bibliothekare verstehen sich selbst häufig als Informationsvermittler
und die Bibliothek als Schnittstelle für den freien Zugang zu
Informationen. Ethische Themenstellungen, die sich daraus ergeben
können, sind beispielsweise die Sicherung der Informationsfreiheit, der
Schutz des geistigen Eigentums, die Zensur und der Datenschutz.

Der Aufsatz behandelt die Fragen, mit welchen Themenstellungen sich
bibliothekarische Berufsethik beschäftigt, ob es Kernaussagen zur
bibliothekarische Ethik gibt, die unabhängig von Land und Kultur sind
und welche Kernaussagen die deutsche Berufsethik im Allgemeinen trifft.
Grundlage dieses Textes sind die Ergebnisse und Untersuchungen meiner
Diplomarbeit, die sich mit Ethikkodizes unterschiedlicher Länder und dem
deutschen Ethikkodex befasst.\footnote{Vgl. Spenke, Julia: Ethik für den
  Bibliotheksberuf: Zu Entwicklung und Inhalt eines bibliothekarischen
  Ethikkodexes in Deutschland, 2011, S. 31.
  \url{http://opus.bibl.fh-koeln.de/volltexte/2011/305/}}

\hypertarget{bibliothekarische-berufsethik-und-informationsethik}{%
\section*{Bibliothekarische Berufsethik und
Informationsethik}\label{bibliothekarische-berufsethik-und-informationsethik}}

Bei der Berufsethik oder auch professionellen Ethik handelt es sich um
eine Form der Ethik, die speziell auf die Anforderungen der Berufsgruppe
abgestimmt ist. Sie entwickelt keine grundsätzlich neuen ethischen
Prinzipien, sondern wendet allgemeine Prinzipien auf den Bereich des
Berufes an. Die bibliothekarische Berufsethik beschäftigt sich demnach
mit ethischen Fragestellungen, die das Betätigungsfeld des
Bibliothekares betreffen. \footnote{Vgl. Froehlich, Thomas J.: Survey
  and analysis of the major ethical and legal issues facing library and
  information services, 1997, S. 5.}

Die Prinzipien einer professionellen Ethik können in Form eines
Ethikkodexes manifestiert werden. Dieser kann als das kollektives
Gewissen eines Berufsstandes (\enquote{collective conscience of a
profession})\footnote{Frankel, Mark S.: Professional Codes. In: Journal
  of Business Ethics, 1989, S. 110.} angesehen werden. Es lässt sich
vermuten, dass ein Ethikkodex, wenn möglichst viele Mitglieder des
Berufsstandes in seine Entstehung mit eingebunden sind, das Potenzial
hat, zu mehr Bewusstsein für ethische Problemstellungen innerhalb des
Berufsstandes beizutragen.\footnote{Vgl. McMenemy, David; Poulter, Alan;
  Burton, Paul F.: A handbook of ethical practice, 2007, S. 28.}
Weiterhin dient er als Entscheidungshilfe bei ethischen
Problemstellungen.\footnote{Vgl. Preer, Jean L: Library ethics, 2008, S.
  8.}

Für den bibliothekarischen Berufsstand existieren neben dem deutschen
Ethikkodex, weitere Kodizes aus anderen Ländern. Alex Byrne beobachtet,
dass zwar die Formulierungen von Themen bibliothekarischer Ethik in
verschiedenen Kulturen unterschiedlich sind, die Kernaussagen aber
größtenteils die gleichen Themengebiete betreffen.\footnote{Vgl. Byrne,
  Alex: Introduction. In: The ethics of librarianship, 2002, S. 14f.}
Eng verknüpft sind diese Themen mit den Aufgaben, die dem Bibliothekar
zugesprochen werden, der sich selbst zunehmend als
Informationsvermittler versteht. Daraus resultiert die Frage, wie
bibliothekarische Ethik zur Informationsethik zu positionieren ist.
Informationsethik ist, weit gefasst, jede Beschäftigung mit ethischen
Fragen, die im Zusammenhang mit den Wechselbeziehungen zwischen Mensch
und Information auftreten.\footnote{Vgl. Smith, Martha: Information
  Ethics. In: Advances in Librarianship, 2001, S. 32.} Der Zugang zu
Informationen, das geistige Eigentum des Urhebers, die Vertraulichkeit,
die Sicherheit und die Auswirkungen von Information auf die Gemeinschaft
sind für sie relevante Themen. Diese überschneiden sich mit denen
bibliothekarischer Berufsethik. Allerdings befasst sich die
Informationsethik übergreifender mit diesen Fragen und den
gesellschaftlichen Auswirkungen von Informationsstrukturen, als es die
bibliothekarische Berufsethik bisher tut. Sie ist dennoch nicht als eine
Unterdisziplin der Informationsethik anzusehen, sondern als eine
verwandte Disziplin. Bibliothekarische Berufsethik behandelt auch
Fragestellungen zum Berufsstand, die nicht Gegenstand der
Informationsethik sind.

\hypertarget{was-sind-themen-bibliothekarischer-berufsethik}{%
\section*{Was sind Themen bibliothekarischer
Berufsethik?}\label{was-sind-themen-bibliothekarischer-berufsethik}}

Diese Frage soll von zwei Seiten beleuchtet werden. Zunächst soll
untersucht werden, was ausgewählte Autoren als wesentliche Bereiche der
Berufsethik ansehen. Diese Ergebnisse sollen anschließend mit den
Aussagen verglichen werden, die tatsächlich in den Ethikkodizes
verschiedener Länder getroffen werden.

Ausgewählt werden vier Autoren, die in der Literatur besonders häufig
genannt werden. Die von ihnen als zentral angesehenen Kernaussagen
werden miteinander verglichen und versucht, vergleichbare Aussagen
zusammenzuführen. Es handelt sich um die Kernaussagen, die Wallace
Koehler und Michael J. Pemberton,\footnote{Koehler, Wallace; Pemberton,
  Michael J.: A Search for Core Values. In: Journal of Information
  Ethics, 2000, S. 26--54.} Richard Rubin und Thomas J.
Froehlich,\footnote{Rubin, Richard; Froehlich, Thomas J.: Ethical
  Aspects of Library and Information Science. In: Encyclopedia of
  library and information science, 1996, S. 33--52.} Michael
Gorman\footnote{Gorman, Michael: Our Enduring Values, 2000, hier: S.
  26f.} sowie Bob Usherwood\footnote{Usherwood, Bob: Towards a Code of
  Professional Ethics. In: Aslib Proceedings, 1981, S. 233--242.} in
ihren Texten formuliert haben. Die von den Autoren getroffenen
Kernaussagen (zwischen sechs und neun) unterscheiden sich in ihrer
Formulierung und darin, wie weit sie gefasst sind. Sie lassen sich aber
auf sinngemäß ähnliche Aussagen zusammenführen. So befassen sie sich
alle mit dem freien und gleichberechtigten Zugang zu Informationen sowie
dem Datenschutz. Drei Autoren nennen die gesellschaftliche Verantwortung
des Bibliothekars (\enquote{social and legal responsibilities},
\footnote{Koehler, Wallace; Pemberton, Michael J.: A Search for Core
  Values. In: Journal of Information Ethics, 2000, S. 33.}
\enquote{societal issues}\footnote{Rubin, Richard; Froehlich, Thomas J.:
  Ethical Aspects of Library and Information Science. In: Encyclopedia
  of library and information science, 1996, S. 38f.} oder
\enquote{democracy}\footnote{Gorman, Michael: Our Enduring Values, 2000,
  S. 27.} und drei beschäftigen sich mit dem Konflikt, der sich aus der
Verantwortung gegenüber unterschiedlichen Interessensgruppen ergibt.
Zwei Autoren thematisieren Zensur und Auswahl der Medien betreffende
Themen als ethische Kerngegenstände.

% \usepackage{colortbl}


\begin{table}[htp!]
\centering
\begin{tabularx}{\textwidth}{@{} X X X X @{}}
\toprule
\textbf{Koehler und Pemberton} & \textbf{Rubin und Froehlich} & \textbf{Gorman} & \textbf{Usherwood} \\ 
\midrule
Client/ Patron Rights and Privileges & Privacy Issues / Reference Service & Service / Privacy & Discretion and respect of the client's privacy \\ 
\midrule
Selection Issues & Selecting Materials and Censorship & ~ & ~ \\ 
\midrule
Access Issues & Issues of Access / Technology-Related Issues & Intellectual Freedom / Equity of Access to recorded knowledge and information & Professional independence and intellectual freedom \\ 
\midrule
Professional Practice and relationship & ~ & ~ & Competence of the librarian \\ 
\midrule
Responsibilities to employers & Conflicting Loyalities & ~ & Integrity of members \\ 
\midrule
Social and legal responsibilities & Societal Issues & Democracy & ~ \\ 
\midrule
~ & Administrative Issues & ~ & ~ \\ 
\midrule
~ & Copyright Issues & ~ & ~ \\ 
\midrule
~ & ~ & Stewardship & ~ \\ 
\midrule
~ & ~ & Rationalism & ~ \\ 
\midrule
~ & ~ & Literacy and learning & ~ \\ 
\midrule
~ & ~ & ~ & Impartiality of the library profession \\ 
\midrule
~ & ~ & ~ & Financial ethics \\
\bottomrule
\end{tabularx}
\caption{Themengebiete der bibliothekarischen Berufsethik in Deutschland}
\end{table}

Zusammenfassend können der freie Informationszugang, die
bibliothekarische Dienstleistung, der Datenschutz, die gesellschaftliche
Verantwortung des Bibliothekars, Interessenkonflikte, sowie die Zensur
und Aspekte, die konkreter den Berufsstand betreffen, als zentrale
Themengebiete bibliothekarischer Ethik bezeichnet werden.

Um im Vergleich zu diesen Ergebnissen feststellen zu können, ob es in
Ethikkodizes unterschiedlicher Länder ebenfalls überschneidende
Kernaussagen gibt und welche dies sind, wurden 38 verschiedene
Ethikkodizes aus verschiedenen Ländern untersucht. Es handelt sich
hierbei ausschließlich um Ethikkodizes, die sich an das gesamte
Bibliothekswesen eines Landes richten und in englischer oder deutscher
Sprache online zugänglich sind.\footnote{Eine Übersicht der untersuchten
  Länder und die URLs zu den untersuchten Ethikkodizes befinden sich am
  Ende des Aufsatzes.}

Hauptergebnis dieser Untersuchung ist, dass sich unabhängig vom
jeweiligen Land und Kultur vergleichbare Kernaussagen feststellen
lassen. Es wurden elf verschiedene Themengebiete identifiziert, die in
mindestens 30\,\% der untersuchten Ethikkodizes behandelt werden.

% \usepackage{booktabs}


\begin{table}[htp!]
\centering
\begin{tabularx}{\textwidth}{@{} X r r @{}}
\toprule
\textbf{Kernaussage}  & \textbf{Häufigkeit in Prozent}  & \textbf{Häufigkeit total} \\
\midrule
Professionalität / Berufsstand & 100\,\% & 38 \\
\midrule
Freier Zugang zu Informationen & 86,84\,\% & 33 \\
\midrule
Datenschutz & 86,84\% & 33 \\
\midrule
Dienstleistung gegenüber dem Nutzer & 84,21\,\% & 32 \\
\midrule
Neutralität und Objektivität & 71,05\,\% & 27 \\
\midrule
Gesellschaftlicher Auftrag & 63,16\,\% & 24 \\
\midrule
Interessenkonflikt & 60,52\,\% & 23 \\
\midrule
Zensur & 56,71\,\% & 21 \\
\midrule
Verhalten gegenüber Kollegen & 52,63\,\% & 20 \\
\midrule
Urheberrecht & 39,47\,\% & 15 \\
\bottomrule
\end{tabularx}
\caption{Themengebiete der aus 38 Ländern stammenden Ethikkodizes}
\end{table}

Obwohl die Ethikkodizes zu den gleichen Themenstellungen Aussagen
treffen, können sich die Empfehlungen, welche die verschiedenen
Ethikkodizes zu einem Themengebiet treffen, voneinander unterscheiden.
Ebenfalls verschieden ist die Gewichtung der Kernaussagen: Der
Datenschutz wird beispielsweise in nahezu allen Ethikkodizes als ein
Kernelement bibliothekarischer Ethik genannt. Die getroffenen Aussagen
hierzu unterscheiden sich allerdings. Meistens wird hier der Schutz von
persönlichen Nutzerdaten verlangt. In seltenen Fällen wird die
Einschränkung getroffen, dass sich der Bibliothekar bezogen auf den
Datenschutz innerhalb der rechtlichen Rahmenbedingungen bewegen soll.
Bei dem Themengebiet Zensur sticht der Ethikkodex der Association des
Bibliothécaires Français (ABS) besonders hervor. Er lehnt nicht wie alle
anderen untersuchten Ethikkodizes die Zensur ab, sondern fordert, die
gesetzlichen Restriktionen bezüglich Diskriminierung und Gewalt
einzuhalten.\footnote{Association des Bibliothécaires Français (ABS):
  The librarians' code of ethics, 2003.
  \url{http://archive.ifla.org/faife/ethics/frcode.htm}.}

Datenschutz und der freie Zugang zu Informationen scheinen zentrale
Themenbereiche der bibliothekarischen Ethik zu sein. Diese Themen werden
sowohl von den untersuchten Aussagen der oben genannten Autoren, als
auch in den untersuchten Ethikkodizes häufig angesprochen. Auffällig
ist, dass die Ausführungen der Einzelautoren dem bibliothekarischen
Berufsstand wenig Bedeutung beimessen. Im Vergleich dazu ist dies die
Kernaussage aller Ethikkodizes. Eine Erklärung könnte sein, dass die
Ethikkodizes in den meisten Fällen von dem oder den bibliothekarischen
Verbänden eines Landes verfasst wurden, denen der Berufsstand besonders
wichtig ist.

Es scheint also eine weitgehende Übereinstimmung darüber zu herrschen,
welches als die zentralen Themen bibliothekarischer Ethik -- auch
länderübergreifend- angesehen werden.

\hypertarget{deutsche-bibliothekarische-berufsethik.}{%
\section*{Deutsche bibliothekarische
Berufsethik.}\label{deutsche-bibliothekarische-berufsethik.}}

\hypertarget{wie-und-warum-wurde-der-ethikkodex-verfasst}{%
\subsection*{Wie und warum wurde der Ethikkodex
verfasst?}\label{wie-und-warum-wurde-der-ethikkodex-verfasst}}

Der deutsche bibliothekarische Ethikkodex wurde im Jahr 2007 unter dem
Namen \enquote{Ethische Grundsätze der Bibliotheks- und
Informationsberufe} \footnote{Bibliothek \& Information Deutschland:
  Ethik und Information, 2007, S. 1f.
  \url{http://www.bideutschland.de/download/file/allgemein/EthikundInformation.pdf}.}
auf dem Bibliothekskongress in Leipzig, dessen Motto \enquote{Ethik und
Information} lautete, vorgestellt. Urheber des Ethikkodexes war die
Arbeitsgruppe Informationsethik des BID. Grundlegend für die inhaltliche
Gestaltung des deutschen Ethikkodexes waren die Webpräsenz der
FAIFE\footnote{Committee on Free Access to Information and Freedom of
  Expression der IFLA, welches intellektuelle Freiheit als
  Hauptthemengebiet hat. Die FAIFE veröffentlicht regelmäßig ihren World
  Report, in dem sie über die Lage der intellektuellen Freiheit auf der
  Welt berichtet. \url{http://www.ifla.org/FAIFE}.} und die Untersuchung
der auf den FAIFE-Webseiten veröffentlichten Ethikkodizes verschiedener
Länder.\footnote{Hohoff, Ulrich: Ethische Grundsätze der Bibliotheks- und Informationsberufe, 2008, S. 3.
  \url{http://www.opus-bayern.de/bib-info/volltexte/2008/498}.}

Das FAIFE-Programm scheint ebenfalls ein Auslöser dafür zu sein, dass
der Ethikkodex verfasst wurde. FAIFE sammelt auf seiner Webpräsenz
Ethikkodizes und fragt in seinen Weltberichten regelmäßig ab, in welchen
Ländern es bibliothekarische Ethikkodizes gibt. Die Verfasser des
Ethikkodizes waren der Meinung, dass auch ein deutscher Ethikkodex auf
den Website der FAIFE zu finden sein sollte. Weiterhin soll die
Publikation der ethischen Grundsätze dem Zweck dienen, das Bewusstsein
darüber zu verbessern, dass es sich bei den bibliothekarischen Berufen
um einen Berufsstand handelt, welcher gemeinsame ethische Werte
vertritt. Sie sollen als ein Mittel dienen, um den Berufsstand in der
Öffentlichkeit positiv darzustellen. Darüber hinaus bieten sie nach
Aussage der Verfasser die Chance, die Diskussion innerhalb des
Bibliothekswesens über ethische Fragestellungen anzuregen.\footnote{Vgl.
  Hohoff, Ulrich: Ethische Grundsätze der Bibliotheks- und
  Informationsberufe, 2008, S. 4--6.
  \url{http://www.opus-bayern.de/bib-info/volltexte/2008/498}.}

Am Entstehungsprozess ist zu bemängeln, dass der Ethikkodex nicht auf
Grundlage einer Diskussion in der bibliothekarischen Öffentlichkeit
entstanden ist, sondern erst als bereits fertig ausgearbeitetes Papier
vorgestellt wurde. Die mangelnde Akzeptanz des Papiers zeigt sich daher
nicht in der inhaltlichen Ablehnung, sondern in dem geringen Interesse.
Eine Beteiligung am Entstehungsprozess hätte zur Auseinandersetzung mit
ethischen Themen innerhalb der Berufsgruppe beitragen können. Ein
zweiter Kritikpunkt ist, dass zu wenige Maßnahmen ergriffen wurden, um
den Ethikkodex innerhalb des Berufsstandes bekannt zu machen. Dazu hätte
sich zum Beispiel eine gesonderte Veranstaltung zur Vorstellung des
Ethikkodexes auf dem Bibliothekskongress 2007 in Leipzig angeboten.

Von Seiten der bibliothekarischen Öffentlichkeit gab es wenige
Reaktionen auf den Ethikkodex. Untersuchungen der wichtigsten
Fachzeitschriften des Bibliothekswesens\footnote{Bibliotheksdienst,
  ABI-Technik, ZfBB, Information, Wissenschaft \& Praxis, BIT-online und
  Bibliothek, Forschung und Praxis.} aus dem Jahr 2007, der
bibliothekarischen E-Maillisten Forumoeb und Inetbib aus dem Jahr 2007
und verschiedener bibliothekarischer Fachblogs\footnote{bibliothekarisch.de:
  \url{http://bibliothekarisch.de/blog/}, IBI-Weblog:
  \url{http://weblog.ib.hu-berlin.de/}, Infobib:
  \url{http://infobib.de/}, netbib: \url{http://log.netbib.de/}.}
ergaben, dass es als Reaktion auf den deutschen Ethikkodex kaum
Veröffentlichungen zu dem Thema gab.\footnote{Vgl. Spenke, Julia: Ethik
  für den Bibliotheksberuf: Zu Entwicklung und Inhalt eines
  bibliothekarischen Ethikkodexes in Deutschland, 2011, S.~31.
  \url{http://opus.bibl.fh-koeln.de/volltexte/2011/305/}.} Mittlerweile
lässt sich ein etwas stärkeres Interesse in der bibliothekarischen Welt
für bibliothekarische Ethik verzeichnen. Dies zeigt sich beispielsweise
durch den Blog \enquote{Ethik von unten} zum Thema Ethik für
Bibliothekare, den es seit 2010 gibt.\footnote{Ethik von unten:
  \url{http://ethikvonunten.wordpress.com/}.} Auch der Themenschwerpunkt
\enquote{Bibliotheksethik}\footnote{Schwerpunkt Bibliotheksethik: BuB:
  Forum Bibliothek und Information, S. 270-286.} in der Zeitschrift BuB
und die Vortragsreihe auf dem Deutschen Bibliothekartag 2011 mit dem
Titel \enquote{Berufsethik: Randerscheinung oder Grundlage
bibliothekarischer Praxis}, lassen auf ein wachsendes Interesse an Ethik
in dieser Profession schließen. Es bleibt jedoch fraglich, ob sich das
auf die deutschen bibliothekarischen Grundsätze zurückführen lässt.

\hypertarget{welche-kernaussagen-werden-getroffen-wie-gestaltet-sich-der-ethikkodex-inhaltlich-im-vergleich-zu-anderen-ethikkodizes}{%
\subsection*{Welche Kernaussagen werden getroffen? Wie gestaltet sich der
Ethikkodex inhaltlich im Vergleich zu anderen
Ethikkodizes?}\label{welche-kernaussagen-werden-getroffen-wie-gestaltet-sich-der-ethikkodex-inhaltlich-im-vergleich-zu-anderen-ethikkodizes}}

Der deutsche Ethikkodex gliedert sich in drei Abschnitte. Nach der
Einleitung, in der kurz Entstehungshintergrund und Zweck des Kodexes
erläutert werden, werden im zweiten Abschnitt ethische Richtlinien für
den Umgang mit dem Bibliothekskunden formuliert. Die angesprochenen
Kernthemen in diesem Zusammenhang sind Dienstleistung (Gleichbehandlung
und qualitativ hochwertig), Zugang zu Informationen, Datenschutz,
Berufsstand und Neutralität.\footnote{Vgl. Bibliothek \& Information
  Deutschland: Ethik und Information, 2007, S. 1f.
  \url{http://www.bideutschland.de/download/file/allgemein/EthikundInformation.pdf}.}
Im darauf folgenden Abschnitt werden unterschiedliche Tätigkeitsgebiete
des Bibliothekars näher beleuchtet. Daraus lassen sich folgende
Kernthemen erkennen: freier Zugang zu Informationen, Zensur,
gesellschaftliche Aufgabe der Bibliothek, Bestandsschutz, kollegiales
Verhalten, Neutralität, Dienstleistung, Urheberschutz und
Bestandsschutz.\footnote{Vgl. ebd., S. 2f.}

Vergleicht man die Kernaussagen des deutschen Ethikkodexes mit den
gefundenen Kernaussagen in der Literatur und den Ethikkodizes anderer
Länder, fällt auf, dass keine wesentlichen Aspekte fehlen. Lediglich
Kernaussagen bezogen auf einen möglichen Interessenskonflikt, zum
Beispiel gegenüber den Interessen des Berufsstandes, der
Trägerinstitution und dem Nutzer werden nicht formuliert.

Der deutsche Ethikkodex weist verschiedene Besonderheiten auf. Was den
deutschen Ethikkodex von den Kodizes anderer Länder unterscheidet sind
seine Länge (zweieinhalb Seiten: so lang ist kaum ein anderer
Ethikkodex) und die Wiederholungen von Kernaussagen bzw. Bezüge. Der
deutschen Gesetzgebung wird durch den Verweis auf das Jugendschutzgesetz
und die gesetzlichen Regelungen zum Datenschutz auffällig viel Bedeutung
zugemesssen. Andere Ethikkodizes verweisen sehr selten auf die geltende
Gesetzgebung. Problematisch kann ein solcher Verweis dann sein, wenn
sich Gesetze so ändern, dass sie den ethischen Grundsätzen der
Berufsgruppe nicht mehr entsprechen. Ein Ethikkodex, der sich auf
Gesetze beruft, kann in diesem Fall nicht seine Aufgabe als Gewissen des
Berufsstandes erfüllen.

Ebenfalls unüblich sind die recht konkreten Forderungen, zum Beispiel
nach Barrierefreiheit oder Veranstaltungen zur Leseförderung. In anderen
Kodizes werden solche Themen nicht explizit angesprochen. Sie lassen
sich aber allgemeineren Kernaussagen zuordnen und werden daher indirekt
angesprochen. Es geht dann zum Beispiel nicht konkret um Leseförderung,
sondern um die gesellschaftliche Aufgabe der Bibliothek sich in der
Bildung zu engagieren.

Insgesamt erfüllt der deutsche Ethikkodex jedoch, obwohl er knapper und
weniger detailliert ausfallen könnte, die inhaltlichen Anforderungen an
einen bibliothekarischen Ethikkodex, da er zu den zentralen
Themengebieten Aussagen trifft.

\hypertarget{was-luxe4sst-sich-aus-diesen-ergebnissen-schlieuxdfen}{%
\subsection*{Was lässt sich aus diesen Ergebnissen
schließen?}\label{was-luxe4sst-sich-aus-diesen-ergebnissen-schlieuxdfen}}

Es lässt sich festhalten, dass ethische Problemstellungen existieren,
die speziell den Beruf des Bibliothekares betreffen. Dieser Schluss
lässt sich auf Grundlage der Untersuchung der Fachliteratur zu diesem
Thema und der Untersuchung der Ethikkodizes verschiedener Länder ziehen.

Der deutsche Ethikkodex greift die wesentlichen Themengebiete
bibliothekarischer Ethik auf. Kritik an dem deutschen Papier ergibt sich
eher aus seinem Entstehungsprozess. Wünschenswert wäre es, die
Diskussion innerhalb des deutschen Bibliothekswesens über aktuelle
ethische Konflikte anzustoßen. Ein Mittel hierzu wären regelmäßige
Veranstaltungen zu konkreten Kernthemen bibliothekarischer Ethik, wie
zum Beispiel Zensur oder Urheberrecht (vergleichbar mit der
\enquote{banned books week} der ALA
(\url{http://www.bannedbooksweek.org/}). Die im deutschen Ethikkodex
getroffenen Aussagen können als Ausgangspunkt für die Diskussion dienen.

\begin{center}\rule{0.5\linewidth}{0.5pt}\end{center}

\hypertarget{literatur}{%
\section*{Literatur}\label{literatur}}

Bibliothek \& Information Deutschland: Ethik und Information. Ethische
Grundsätze der Bibliotheks- und Informationsberufe, 2007. Online
verfügbar unter
\url{http://www.bideutschland.de/download/file/allgemein/EthikundInformation.pdf}
(zuletzt aktualisiert am 19.03.2007).

Byrne, Alex: Introduction. Information Ethics for a New Millenium. In:
The Ethics of Librarianship. An international survey. Vaagan, Robert W.
(Hg.). München: Saur, 2002, S. 8--18.

Frankel, Mark S.: Professional Codes. Why, how, and with what impact?
In: Journal of Business Ethics, 8, 1989, 2-3, S. 109--115. Online
verfügbar unter
\url{http://www.springerlink.com/content/xt675454l5j72v3h/fulltext.pdf}
(zuletzt geprüft am 01.04.2010).

Froehlich, Thomas J.: Survey and analysis of the major ethical and legal
issues facing library and information services. München: Saur, 1997
(IFLA publications, 78).

Gorman, Michael: Our Enduring Values. Librarianship in the 21st Century.
Chicago, 2000 (ALA Editions).

Hohoff, Ulrich: Ethische Grundsätze der Bibliotheks- und
Informationsberufe. Eine Einführung in das Papier der BID (2007), 97.
Deutscher Bibliothekartag 2008. Online verfügbar unter
\url{http://www.opus-bayern.de/bibinfo/volltexte/2008/498/} zuletzt
aktualisiert am 22.05.09 (zuletzt geprüft am 24.02.2010).

Koehler, Wallace; Pemberton, Michael J.: A Search for Core Values.
Toward a Model Code of Ethics for Information Professionals. In: Journal
of Information Ethics, 9, 2000, 1, S. 26--54.

McMenemy, David; Poulter, Alan; Burton, Paul F.: A handbook of ethical
practice. A practical guide to dealing with ethical issues in
information and library work. Oxford: Chandos Publ., 2007 (Chandos
information professional series).

Preer, Jean L: Library ethics. Westport, Conn.: Libraries Unlimited,
2008.

Rubin, Richard; Froehlich, Thomas J.: Ethical Aspects of Library and
Information Science. In: Encyclopedia of library and information
science. Kent, Allen (Hg.). New York: Dekker, 1996, S. 33--52.

Smith, Martha: Information Ethics. In: Advances in Librarianship.
Lynden, Frederick C. (Hg.). San Diego: Academic Press, 2001, S. 29--66.

Spenke, Julia: Ethik für den Bibliotheksberuf: Zu Entwicklung und Inhalt
eines bibliothekarischen Ethikkodexes in Deutschland, 2011,
\url{http://opus.bibl.fh-koeln.de/volltexte/2011/305/} zuletzt
aktualisiert am 07.02.11 (zuletzt geprüft am 31.07.2011).

Usherwood, Bob: Towards a Code of Professional Ethics. In: Aslib
Proceedings, 33, 1981, 6, S. 233--242.

\hypertarget{anhang}{%
\section*{Anhang}\label{anhang}}

Für \emph{Liste der inhaltlich untersuchten Ethikkodizes} siehe Julia
Spenke, \enquote{Bibliothekarische Berufsethik: Mit welchen Themen
befassen sich bibliothekarische Ethikkodizes?}. LIBREAS. Library Ideas,
19 (2011). \url{https://libreas.eu/ausgabe19/texte/01spenke.htm}


\end{document}