\documentclass[output=paper]{langscibook}

\title{Scheitern in der Schreibwerkstatt: Aus der Redaktion der LIBREAS. Library Ideas}
\author{Redaktion LIBREAS}

\abstract{
Zitiervorschlag: Redaktion LIBREAS, \enquote{Scheitern in der Schreibwerkstatt: Aus der Redaktion der LIBREAS. Library Ideas}. LIBREAS. Library Ideas, 20 (2012). \url{https://libreas.eu/ausgabe20/texte/04redaktion01.htm}

URN: \href{http://nbn-resolving.de/urn:nbn:de:kobv:11-100199730}{urn:nbn:de:kobv:11-100199730}

DOI: \url{http://dx.doi.org/10.18452/9000}

Keywords: Wissenschaftliches Publizieren, Scheitern, LIBREAS. Library Ideas
}

\begin{document}
\maketitle

%abstracts

%body
\noindent Beim Thema Scheitern ist die Arbeit an LIBREAS. Library Ideas überhaupt nicht auszunehmen. Vielmehr ist die Herausgabe jeder Ausgabe mit mehr Scheitern und Kompromissen verbunden, als mit Erfolgen. Dies lernt man sehr schnell, wenn man sich auf ein Projekt wie LIBREAS einlässt: Zu jedem publizierten Artikel, zu jeder umgesetzten Idee, zu jedem eingehaltenem Anspruch lässt sich auch das Gegenteil anführen. Ist das eine Eigenheit unserer Redaktion? Überhaupt nicht. Egal, in welche Redaktion man Einblick erhält, es ist ähnlich. Dabei unterscheiden sich nicht einmal wissenschaftliche, journalistische oder literarische Publikationen groß voneinander. Der Unterschied liegt höchstens darin, dass das Scheitern dort praktisch nie ein öffentliches Thema ist.

Warum machen wir das dann überhaupt? Das ist nicht so klar, wie es vielleicht nach außen erscheint. Die Zeitschrift lebt vom Engagement Einzelner und diese Einzelnen haben immer wieder unterschiedliche Meinungen, die zumindest in Redaktionskonferenzen und in den Tagen vor der Veröffentlichung einer neuen Ausgabe jedesmal neu zur Sprache kommen.

Der Grundthese der nächsten Ausgabe der LIBREAS folgend, dass Scheitern immer auch eine Lernmöglichkeit darstellt – und angesichts des aktuellen gesellschaftlichen Hochs des Transparenz-Paradigmas – soll hier kurz dargestellt werden, wie und woran LIBREAS immer wieder scheitert, vollständig oder ganz. Warum nicht ehrlich dazu stehen? Dieses Scheitern hat, trotz aller zeitweiser Krisen und dem temporären oder vollständigen Rückzug einzelner Redaktionsmitglieder, die Zeitschrift selber nicht zum Zusammenbruch gebracht. Vielmehr hat sie sich entwickelt und ist für uns in der Redaktion ein wichtiger Teil unserer Identität geworden.

\section*{Was will LIBREAS?}\label{Was will LIBREAS?}

Es ist schon einige Jahre her seit die Zeitschrift gegründet wurde. Damals bestand die gesamte Redaktion aus Studierenden am Institut für Bibliothekswissenschaft der Humboldt-Universität, die sich eine Zeitschrift wünschten, in der sie auf der einen Seite mehr wissenschaftliche Texte publiziert sehen wollten, als dies in den meisten bibliothekarischen Publikationen der Fall war. Auf der anderen Seite sollte die Zeitschrift offen sein für formale Experimente und exotische Themen, passend zur Tatsache, dass es sich bei Bibliothekswissenschaft um ein sogenanntes „Orchideen-Fach” handelt.

Seitdem hat sich einiges verändert. Niemand in der – inzwischen gewachsenen – Redaktion studiert mehr, vielmehr arbeiten wir im Bibliotheks- und Informationswesen sowie der Wissenschaft, nicht nur in Berlin, sondern zunehmend verteilt. Das Institut heißt heute anders und hat andere Schwerpunkte, auch die restliche bibliothekarische Presse hat sich gewandelt. Was sich nicht verändert hat, ist der Fakt, dass der Anspruch von LIBREAS immer noch nicht wirklich feststeht. Vielmehr wird er regelmäßig neu verhandelt.

Wir wollen bibliotheks- und informationswissenschaftliche Themen bearbeiten, die nicht unbedingt praxisfern sein müssen, ohne aber dem teilweise vertretenden Praxisfetisch bibliothekarischer Publikationen zu frönen. Außerdem stehen wir für sehr exotische Themen offen. (Als Online-Zeitschrift müssen wir ja auf keinen Druckplatz achten. Zweiundzwanzig Seiten über die Digitalisierung hochmittelalterlicher Blockbücher aus dem Bestand südschwedischer Bibliotheken? Kein Problem. Theoretische Reflexionen über die Getränkeauswahl im Rahmen von Veranstaltungen für Erstleserinnen und -leser mit fünfzig Abbildungen? Unbedingt, immer her damit.) Wir betreiben eine Form der Qualitätskontrolle und des Peer-Review (wobei wir vor allem selber die Peers sind, aber so ist das in einem kleinen Fach mit einer fachlich zusammengesetzten Redaktion nun einmal), gleichzeitig versuchen wir, auch immer wieder andere Formen als die wissenschaftlichen Texte zu fördern. Bildstrecken, Essays und so weiter versuchen wir immer wieder zu motivieren.

Doch: Das alles in ein klares Programm zu fassen, daran scheitern wir. Nicht, dass es nicht genügend Positionen innerhalb der Redaktion gäbe. Beständig gab es Bestrebungen, die Zeitschrift noch stärker am wissenschaftlichen Normalbetrieb (Double Blind Review, Ablehnungsquoten et cetera) zu orientieren, aber auch immer das Bestreben, dass gerade nicht zu tun. Schon die Frage, was eigentlich die Aufgabe des Reviews ist, konnten wir bislang nicht so richtig klären. Die eine Extremposition ist die der rabiaten Qualitätskontrolle, die andere die, dass es Aufgabe der Redaktion und Reviewenden sei, einen Text zusammen mit den Schreibenden solange zu reviewen, bis der Text veröffentlicht werden kann (was gleichzeitig heißt, tendenziell nichts abzulehnen, sondern immer nur besser zu machen). Dazwischen findet sich immer wieder ein Kompromiss.

Auch die Schwerpunkte sind immer wieder ein Kompromiss. Es hat sich immerhin etabliert, dass jede Ausgabe einen Schwerpunkt hat und ein Großteil der Texte der jeweiligen Ausgabe sich auf diesen Schwerpunkt bezieht. Obgleich dies vor allem von außen als Eigenheit wahrgenommen wird, tun wir uns intern jedes Mal schwer damit, die Enge und Weite eines Themas zu bestimmen, auch deshalb, weil wir uns und den Autorinnen und Autoren nicht zu früh künstliche Grenzen setzen wollen, ohne gleichzeitig zu beliebig zu werden. Auch dies ist immer wieder ein Kompromiss, bei dem unterschiedliche Ansichten zusammengeführt werden müssen.

Grundsätzlich sind wir uns als Redaktion noch nie über den Status und die Aufgabe der LIBREAS einig geworden. Wir sind keine Verbandszeitschrift, das macht beispielsweise die BuB besser. Soviel ist klar. Aber: Sind wir bibliothekswissenschaftliches Feuilleton? Sind wir unser gemeinsames Freizeitprojekt? Sind wir dazu da, die bibliotheks- und informationswissenschaftliche Kommunikation zu ermöglichen und voranzutreiben? Strukturieren wir das Feld mit oder folgen wir den Themen? Wollen wir Trends aufgreifen, um sie zu diskutieren oder setzen, bestimmen, brechen wir sie? Wollen wir das eine oder das andere? Wollen wir innovativ sein oder irritieren? Wie ernst nehmen wir uns? Sollen wir Zitationsstile vorgeben oder nicht? Sollen wir bestimmte Formate erzwingen oder nicht? Regelmäßig sind diese Fragen direkt und indirekt Thema unserer Redaktionsarbeit und immer scheitern wir daran, sie endgültig zu klären.

\section*{Welchen Einfluss hat LIBREAS?}\label{Welchen Einfluss hat LIBREAS?} 

Wenn wir uns schon nicht darauf einigen können, was wir eigentlich als unser Ziel oder unsere Aufgabe ansehen, können wir auch gar nicht sagen, ob wir damit Erfolg haben, ein Ziel zu verfolgen. Eine Hoffnung allerdings, die man trotzdem immer hat, wenn man Zeitschriften macht oder auch Texte schreibt, ist es ja, Einfluss zu gewinnen; zumindest wahrgenommen zu werden und Denkanstöße zu geben. Aber passiert das?

Sicherlich kann man – das ist keine überraschende Aussage in der Bibliotheks- und Informationswissenschaft – die Zitationen und Verweise auf die Zeitschrift nachvollziehen. Doch darüber hinaus wissen wir einfach nicht, ob wir mit unseren Themensetzungen Debatten anstoßen, verändern, abbrechen. Wir wissen nicht, ob die Texte gelesen werden, selbst wenn wir die Abrufe zählen können. Wir wissen auch nicht, ob man LIBREAS überhaupt als Zeitschrift wahrnimmt oder nicht immer noch eher als Weblog-Projekt von Studierenden. Wir meinen es ernst, aber ob das auch so wahrgenommen wird, ist nicht klar.

Man kann sich immer wieder Gedanken darüber machen und einzelne Aussagen zur LIBREAS, die man irgendwo wahrnimmt, zu interpretieren versuchen. Doch letztlich scheitert man immer wieder daran, den tatsächlichen Status festzustellen. Der gangbare Weg ist offenbar, diese Frage regelmäßig zu ignorieren und davon auszugehen, dass Einfluss eh vor allem langfristig ist. (Außerdem ist es bekanntlich moralisch immer besser, zu versuchen, positiven Einfluss zu nehmen und zu scheitern, als es nicht zu versuchen.)

\section*{Scheitern mit jeder Ausgabe}\label{Scheitern mit jeder Ausgabe} 

Ein weiterer erstaunlicher Aspekt bei der Arbeit an der Zeitschrift ist, dass fast alle angedachten Artikel scheitern. Und zwar bei jeder Ausgabe. Auch dies wird kein Spezifikum der LIBREAS sein. Aber: Nachdem ein Schwerpunktthema gefunden ist, werden selbstverständlich Listen angefertigt von möglichen Themen, Unterthemen sowie von Personen, die einen Text dazu liefern könnten. Das ist ein kontinuierlicher Flow. Hat jemand in der Redaktion den Eindruck, jemand anders hätte etwas Ausbaufähiges gesagt, wird eine Anfrage geschrieben (insoweit sollte man darauf achten, was man so bloggt, mikrobloggt, auf Konferenzen sagt). Aber auch bei jedem organisierten oder zufälligen Treffen kommen immer wieder Listen zustande und wird sich vorgenommen, Personen wegen Texten anzufragen. Sicherlich: Es werden immer auch ungefragt Texte als Antwort auf die Calls for Paper angeboten, aber der Drang jeder Redaktion ist es wohl, auch ein wenig die Richtung der eigenen Zeitschrift mit zu beeinflussen – und wenn es um des Beeinflussens Willen ist.

Doch schon an diesem Punkt scheitern die meisten Texte und Beiträge. Manchmal werden andere Formen der Mitarbeit angeboten, zumeist hört man aber nichts oder klare Absagen. Mitunter am Tag vor dem Redaktionsschluss. Sicherlich: Wir alle haben andere Dinge zu tun, Lohnarbeit, Familie, Sozialleben, Projekte, Ehrenamt; gleichzeitig treffen wir nicht selten auf eine überraschende Zurückhaltung von Personen aus der bibliothekarischen Praxis. Viele trauen es sich offenbar – trotz oft akademischer Ausbildung – nicht zu, ein Thema zu bearbeiten oder einen Text zu schreiben.

Allerdings: Selbst wenn diese Hürde überwunden ist, heißt dies noch lange nicht, dass ein Text nicht scheitert. Es gibt ebenso regelmäßig Texte, die doch nicht geschrieben werden oder aber auch nach einiger Zeit zurückgezogen werden. Auch dagegen kann man als Redaktion nichts tun. Es ist teilweise ärgerlich, wenn mit einem Autor oder einer Autorin eine Zeit lang an einem Text gearbeitet, er mehrfach gelesen, korrigiert, erweitert oder gestrafft wurde – und dann zurückgenommen wird. Immerhin kann man dann hoffen, beim Bearbeitungsprozess auch etwas über das Schreiben von wissenschaftlichen Texten vermittelt zu haben. Dennoch: Auch dies ist ein Scheitern.

Nur ein kleiner Teil der angedachten Texte entsteht und wird auch veröffentlicht. Jede Ausgabe ist mit mehr Scheitern als Erfolgen verbunden. Unzählige vorgeschlagene Themen verpuffen, unzählige sind noch in alten Calls for Paper versteckt.

Und dennoch: Jedes Mal erscheint eine Ausgabe der Zeitschrift. Zwischen all dem Scheitern finden sich genug Erfolge, um die gesetzten Schwerpunkte inhaltlich zu füllen. Wäre es nicht so, niemand würde sich mehr in Redaktionen engagieren. (Außer in solchen, die Redakteurinnen und Redakteure anstellen können. Aber das wird bei der LIBREAS absehbar wohl nie passieren. Reden wir also nicht davon.)

Woraus wir aber zwangsläufig eine gewisse Grundlegitimation sowie im Laufe der Jahre eine bestimmte Urteilskraft entwickelt haben, ist der zeitstabile Grundanspruch, die Zeitschrift herauszugeben, die wir selbst gern lesen möchten. Die sich dahinter verbergenden unterschiedlichen Perspektiven sorgen dabei für eine Art internes Checks and Balances. Der Zusammenschluss von einem halben Dutzend völlig konträrer subjektiver Ansichten mündet dann doch in eine Form von Objektivität.

\section*{Etwas beibringen?}\label{Etwas beibringen?} 

Redaktionsarbeit hat für einige von uns eine pädagogische Komponente. Eventuell lernen wir in unseren Ausbildungen nicht oder zumindest nicht genug, wissenschaftliche Texte zu schreiben und Textsorten zu bedienen. So tritt neben das Ermutigen oft auch die Hoffnung, die Schreibenden beim Schreiben zu unterstützen und ihnen etwas mehr zu zeigen, wie ein solcher Text verfasst werden kann. Ist solch ein Anspruch überhaupt gerechtfertigt? Was legitimiert uns eigentlich dazu, außer unserer Position als Redaktion (eine Position, die sich jede und jeder anmaßen kann, der oder die eine Zeitschrift begründet)? Das wir selber schon einiges geschrieben haben? Vielleicht. Aber keiner unserer Texte hat bisher einen Preis erhalten, viele wurden nur zwei, drei Dutzendmal aufgerufen. Weder eine offiziell zugeschriebene Güte noch Popularität haben uns bisher mit übergeordneter Autorität ausgestattet. Vielleicht, das wir uns öfter mit Texten und deren Entstehungsprozessen auseinandersetzen? Zum Teil. Aber mehr auch nicht.

Wenn Rezensionen eintreffen, die keine Rezensionen, sondern Nacherzählungen eines Textes mit einem „ich finde das Buch spannend” oder „das Buch führt spannend in das Thema ein”-Satz als Abschluss sind, kommt bei einem Teil der Redaktion trotzdem die pädagogische Ader durch. Manchmal funktioniert es und man kann den Schreibenden vermitteln, warum diese Texte so nicht sinnvoll sind, oft auch nicht. Auch hier scheitern Texte eher, als das sie nochmal und besser geschrieben werden. (Außerdem sind wir in der Vergangenheit auch oft als Redaktion daran gescheitert, auf solche Texte nicht erst zwei Wochen nach der letzten Deadline einzugehen. An sich gibt es oft die Tendenz, Arbeit zeitlich nach hinten zu verschieben. Wie überall.)

Insoweit ist auch nicht klar, ob diese Anmaßung einer gewissen pädagogischen Rolle überhaupt zu Ergebnissen führt. Auch gibt es einen anderen, großen Teil der Redaktion, welcher eine solche Rolle explizit ablehnt. Es sei nicht die Aufgabe von LIBREAS, einen virtuellen „Wie schreibe ich eine wissenschaftlichen Text”-Kurs anzubieten.

\section*{Die anderen Projekte}\label{Die anderen Projekte}

LIBREAS ist nicht nur die Zeitschrift, sondern auch das Weblog, die Podcasts, jetzt sogar der Förderverein. (Dem man übrigens demnächst beitreten kann.) Doch was ist der Status dieser Projekte? Sind sie gleichberechtigt? Sind sie der Zeitschrift beigeordnet, untergeordnet? Wie ist ihr Bezug, beispielsweise, zur aktuellen Ausgabe?

Oder auch: Welche Texte werden im Weblog veröffentlicht, welche in der Zeitschrift? Ist das Weblog für Preprints da oder nicht? Ist es für Rezensionen da und wenn ja, für welche? Wenn wir im Weblog Rezensionen veröffentlichen, warum dann noch mal in der Zeitschrift? Auch das sind regelmäßig besprochene Themen auf unserer Treffen. Dabei scheitern wir immer wieder an klaren Definitionen. Letztlich sind es Fall-zu-Fall-Unter- und Entscheidungen, die praktisch jedes Mal neu mit der Aufforderung abgeschlossen werden, jetzt endlich klare Regelungen zu finden. Im Gegensatz zu einigen Kolleginnen und Kollegen halten wir alle – immerhin das – daran fest, dass das Medium Zeitschrift weiterhin seine Bedeutung für die wissenschaftliche Kommunikation hat und diese auch nicht so schnell an andere Publikationsformen abgegeben wird. Aber alles andere ist innerhalb der Redaktion im Fluss.

\section*{How did we get here?}\label{How did we get here?}

In der Zusammensicht ist das schon erstaunlich: Eigentlich funktioniert nichts oder nicht viel, weder intern noch in der Zusammenarbeit mit den Autorinnen und Autoren – und dennoch: Sieben Jahrgänge, 19 Ausgaben, 16 Podcasts, eine Unkonferenz später gibt es LIBREAS immer noch und wird es auch weiter geben. Warum? Sicherlich müsste man das alle Beteiligten gesondert fragen und würde jeweils andere Antworten erhalten. Aber eine These wäre, dass wir mit dem ganzen Scheitern auch immer wieder etwas gelernt haben: Über uns, über das Bibliotheks- und Informationswesen und den Wissenschaftsbetrieb. Wir haben gelernt, was andere Zeitschriften anders machen und wir vielleicht gerade nicht machen wollen. Wir haben gelernt, dass ein Scheitern nicht gleich das Ende eines Projektes bedeutet. Und vor allem haben wir gelernt, dass nach dem ganzen Scheitern doch etwas herauskommt, für das wir uns nicht schämen müssen, auf das wir manchmal sogar mit einer gewissen Erfüllung blicken. Wer viel scheitert, auch das eine Erkenntnis, lernt das wenige Gelungene umso mehr zu schätzen.

Ob das den ganzen Stress wert ist, ist eine Frage, die von der Definition von „wert” abhängt. Zeitökonomisch ist die Arbeit reine Selbstausbeutung. Reputationsökonomisch werden wir oft, auch das hört man, wahlweise in die Schubladen „die Selbstverliebten” oder „die Irren aus Berlin” gesteckt. Manchmal hört man aber auch: Gut gemacht! Beides ist uns wichtig und mehr Feedback jeder Art ist immer willkommen, am Ende aber nicht alleine entscheidend. Maßgeblich ist wohl, dass wir uns – wie auch immer motiviert – mit diesem eigenartigen Fach der Bibliotheks- und Informationswissenschaft von Grund auf identifizieren, dass es uns – so obskur das klingen mag – am Herzen liegt und dass wir uns freuen, wenn wir unser Scherflein dazu beitragen können, dass es sich entwickelt.

\end{document}