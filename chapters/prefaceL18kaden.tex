\documentclass[output=paper]{langsci/langscibook} 
\title{Das Warum} 
\author{Linda Freyberg  \affiliation{Museum für Naturkunde}}  

\abstract{Ben Kaden, \enquote{Notizen zur Bibliothekswissenschaft. Teil 1 und 2}. LIBREAS. Library Ideas, 18 (2011). \url{https://libreas.eu/ausgabe18/texte/02kaden02.htm}}

\begin{document}
\maketitle

\noindent Die Zukunft der Bibliotheks- und Informationswissenschaft und die generelle Frage nach ihrer Daseinsberechtigung als Wissenschaft verbunden mit Fragen nach ihren Methoden und ihrem Gegenstand wird immer wieder auf Neue diskutiert. Es existieren durchaus eine Vielzahl von theoretischen Anknüpfungspunkten und Fundierungsmöglichkeiten, u.\,a. durch die Semiotik, die nur am Rande aufgegriffen werden oder ignoriert werden. Die Bibliothekswissenschaft scheint mehr nach Technisierung und Messbarkeit zu streben, was sich in den Studiengängen durch Fusionen mit der Informatik manifestiert, welche sicherlich für die berufliche Praxis durchaus sinnvoll sind und auf der Fokussierung auf Methoden wie Bibliometrie ausdrückt. An Ben Kadens Text von 2011 gefällt mir seine Aktualität sowie die geisteswissenschaftlichen und literarischen Bezüge, die sich sicherlich auch als Lektüre für Studierende der Bibliotheks- und Informationswissenschaft in 2020 eignen; hier vor allem Elena Esposito, aber auch Klassiker wie Capurro und Derrida. Traditionelle Funktionen einer Bibliothek, wie das Sammeln und Bewahren werden in diesem Beitrag neu interpretiert und mit aktuellen bibliothekstopologischen Fragestellungen verbunden.

\end{document}