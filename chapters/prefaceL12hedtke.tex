\documentclass[output=paper]{langsci/langscibook} 
\title{Das Warum} 
\author{Najko Jahn  \affiliation{SUB Göttingen}}  

\abstract{Lacey Prpic Hedtke, \enquote{Cereal Boxes and Milk Crates Zine Libraries and Infoshops are… Now}. LIBREAS. Library Ideas, 12 (2008). \url{https://libreas.eu/ausgabe12/004prp.htm}}

\begin{document}
\maketitle

\noindent Für mich waren Infoläden und Bibliotheken lange zwei getrennte Welten, bis ich diesen Artikel las. Auf der einen Seite die selbstorganisierten, manchmal chaotischen Läden in den besetzten Häusern in der Nachbarschaft. Auf der anderen Seite die professionalisierte öffentliche Bibliothek. Lacey Prpic Hedtke berichtet in ihren Beitrag, worin sich Infoläden und Bibliotheken unterscheiden und zeigt Gemeinsamkeiten. Entstanden ist der Beitrag 2007, den sie zuerst als Zine in den USA und dann in Europa verteilt hat. Sie gibt Einblick in die sozialen Bewegungen der USA, bibliothekarische Kritik des Radical References Collective, das nachbarschaftliche Engagement und die Suche nach Räumen und Gegenöffentlichkeit. Der Beitrag zeigt anhand von Zines aber auch, was in einer Öffentlichen Bibliothek nicht geht und warum Infoläden wichtig sind. Die Infoläden in meiner Nachbarschaft existieren nicht mehr und auch die Öffentliche Bibliothek ist schon lange geschlossen. Ich weiß auch nicht, ob das Zine erhalten ist und bin froh, dass LIBREAS den Text digital bewahren konnte.

\end{document}