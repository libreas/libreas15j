\documentclass[output=paper,toc=false]{langsci/langscibook} 
\title{Das Warum} 
\author{Najko Jahn}  

\abstract{Lacey Prpic Hedtke, \foreignquote{english}{Cereal Boxes and Milk Crates Zine Libraries and Infoshops are… Now}. LIBREAS. Library Ideas, 12 (2008). \url{https://libreas.eu/ausgabe12/004prp.htm}}

\begin{document}
\maketitle

\noindent  In meiner Berliner Jugend waren Infoläden und Bibliotheken zwei getrennte Welten. Auf der einen Seite die selbstorganisierten, manchmal chaotischen Läden in den besetzten Häusern in der Nachbarschaft. Auf der anderen Seite die geordnete Öffentliche Bibliothek, für die ein Ausweis benötigt wurde. Lacey Prpic Hedtke berichtet in ihren Beitrag, worin sich Infoläden und Bibliotheken unterscheiden, weist aber auch auf Gemeinsamkeiten hin. Entstanden ist der Beitrag 2007, der zuerst als Zine in den USA und dann in Europa verbreitet wurde. Lacey Prpic Hedtke gibt Einblick in die sozialen Bewegungen der USA in den 2000er Jahren mit Fokus auf die bibliothekarische Kritik des Radical References Collective. Sie argumentiert, dass Infoläden  Nachbarschaft und sichere Räume ermöglichen, Konzepte, die auch immer mal wieder im Kontext Öffentlicher Bibliotheken erörtert werden. Der Beitrag zeigt anhand von Zines allerdings, was im Gegensatz zum Infoladen in einer Öffentlichen Bibliothek nicht geht. Die Infoläden meiner Nachbarschaft existieren nicht mehr und auch die Öffentliche Bibliothek ist zu. Ich weiß auch nicht, ob Exemplare des Zines erhalten sind. Daher bin ich froh, dass LIBREAS den Text nun wieder druckt.

\end{document}