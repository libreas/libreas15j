\documentclass[output=paper]{langscibook}

\title{Mantra Rechtssicherheit}
\author{Thomas Hartmann}

\abstract{
Zitiervorschlag: Thomas Hartmann, \enquote{Mantra Rechtssicherheit}. LIBREAS. Library Ideas, 22 (2013). \url{https://libreas.eu/ausgabe22/01hartmann.htm}

URN: \href{http://nbn-resolving.de/urn:nbn:de:kobv:11-100208904}{urn:nbn:de:kobv:11-100208904}

DOI: \url{https://doi.org/10.18452/9028}

Keywords: Urheberrecht, Rechtssicherheit
}

\begin{document}
\maketitle
\renewcommand{\thesection}{\Roman{section}} 

%body
\noindent Von einem demokratischen Rechtsstaat erwarten Bürgerinnen und Bürger
ebenso wie Bibliotheken eine verlässliche Verhaltensordnung. Nicht die
Rechtslage an sich, sondern deren oftmals offene oder antiquiert
wirkende Ausformung stößt auf Kritik gerade der Bibliotheken und
Informationseinrichtungen. Die Bibliothekswelt beklagt eine
\enquote{sich in den letzten Jahren verschärfende Situation}\footnote{\emph{Talke},
  Urheberrecht, Datenschutz, Haftung: Wer befreit die Bibliotheken aus
  der Unsicherheit? In: RDB 41/2011, Hefte 1--3, S. 28.}, die
Bibliotheken und ihre MitarbeiterInnen wähnen sich auf
\enquote{juristischem Glatteis}\footnote{Ebenda}. Gerade
öffentliche Stellen verlangen rechtssichere Grundlagen für ihr Handeln,
sie sehen sich \enquote{noch über die allgemeine Pflicht zur Einhaltung
der Gesetze hinaus}\footnote{Ebenda} an Recht und Gesetz
gebunden. Ihr Ruf nach mehr Rechtssicherheit ist beständig und laut. Ob
Rechtssicherheit gelegentlich überschätzt wird, diskutiert dieser
Beitrag am Beispiel des Urheberrechts in Wissenschaft und Bildung.
Berechtigte Kritik an einzelnen Fehlentwicklungen kann nicht den Blick
dafür verstellen, dass zentrale Anliegen an Recht und Gesetz in der
geltenden Urheberrechtssystematik zumindest angelegt sind. Im Folgenden
werden wesentliche rechtsstaatliche Verfahren sowie Prinzipien der
Rechtsetzung und der Rechtsanwendung vorgestellt, welche die Forderung
nach Rechtssicherheit relativieren und erklären helfen, weshalb
Rechtssicherheit in unserem Rechtsstaat nur unvollkommen ausgebildet
sein kann.

\hypertarget{ursachen-partiell-fehlender-rechtssicherheit}{%
\section{Ursachen partiell fehlender
Rechtssicherheit}\label{ursachen-partiell-fehlender-rechtssicherheit}}

\hypertarget{recht-als-adaptiver-prozess}{%
\subsection{Recht als adaptiver
Prozess}\label{recht-als-adaptiver-prozess}}

Die Anpassung und Weiterentwicklung des Rechts ist ein träger Prozess.
Es ist Aufgabe des Rechts, auf gesellschaftliche, ethische,
internationale oder technologische Veränderungen zu reagieren, nicht
aber sie determinativ zu kanalisieren. Es erscheint plausibel, dass die
globale Internettechnologie ihren Erfolg auch, jedenfalls in den
Anfangsjahren, einer weithin fehlenden Regulierung zu verdanken hat.

\hypertarget{kontinuituxe4t-des-rechts}{%
\subsection{Kontinuität des Rechts}\label{kontinuituxe4t-des-rechts}}

Hierzulande wird Recht äußerst selten abrupt, dafür stetig in
(Mini-)Schritten fortgeschrieben. Insofern ist Recht berechenbar, dafür
sind -- vor allem bei technologischen Neuerungen -- zwischenzeitlich
rechtliche Lücken und Unsicherheiten in Kauf zu nehmen.

\hypertarget{allgemein-gefasste-rechtsbestimmungen}{%
\subsection{Allgemein gefasste
Rechtsbestimmungen}\label{allgemein-gefasste-rechtsbestimmungen}}

Es ist eine bewährte Regelungstechnik im Zivilrecht, gesetzliche
Tatbestände eher offen zu formulieren. Das im Jahr 1896 ausgefertigte
Bürgerliche Gesetzbuch bildet noch heute das Fundament des Zivilrechts,
weil seine Bestimmungen weit gefasst und anwendbar sind.
Wertungsbedürftige Tatbestandsmerkmale, Verweisungen oder
Ermessensklauseln kennzeichnen eine gute legistische Praxis und dienen
der Kontinuität des Rechts (siehe oben). Beispiele dafür finden sich auch im
deutschen Urheberrecht, etwa die Zitierfreiheit des §\,51
Urheberrechtsgesetz, die über Jahrzehnte hinweg allenfalls behutsam und
redaktionell angepasst wurde.

Wichtiger aber ist der anhand eher offen formulierter Normen angestrebte
Anspruch unserer Rechtsordnung, sich \emph{der} Gerechtigkeit im
konkreten Einzelfall zumindest anzunähern.

\hypertarget{spezieller-gefasste-rechtsbestimmungen}{%
\subsection{Spezieller gefasste
Rechtsbestimmungen}\label{spezieller-gefasste-rechtsbestimmungen}}

Moderne Gesetzgebung tendiert dahin, spezielle Regelungsmaterien
detailliert abzubilden. Als Beispiele können der E-Learning-Paragraph
52a Urheberrechtsgesetz oder die Bibliotheksschranke des
Urheberrechtsgesetzes, §\,52b, dienen. Letztgenannte vermag zu
verdeutlichen, weshalb speziellere Regelungen nur eingeschränkt
Rechtssicherheit herstellen können. Die seinerzeit durchaus noch
nachvollziehbare Vorstellung typischer Bibliotheksleistungen leitete den
Gesetzgeber, dass öffentliche Bibliotheken ihre digitalisierten
Bestandsexemplare nur an elektronischen \emph{Lese}plätzen anbieten
dürfen. Nach nur wenigen Jahren sind Leseplätze für modernes
wissenschaftliches Arbeiten überholt, doch angesichts der speziellen
Formulierung kann die Norm nicht auf vergleichbare Sachverhalte
erstreckt werden. Auch der richterlichen Rechtsfortbildung sind Grenzen
gesetzt, die urhebergesetzliche Grenzlinie \emph{Leseplätze} zu
überwinden erfordert fast eine Interpretation gegen den gesetzlichen
Wortlaut (contra legem).\footnote{Näher dazu \emph{Hartmann}, Streit ums
  Buch zulasten Dritter, in: F.A.Z. vom 26.09.2012,
  \url{http://www.ip.mpg.de/files/pdf2/FAZ_Hartmann_StreitumsBuchzuLastenDritter26.09.2012.pdf}.}
So droht eine der jüngsten Bestimmungen im Urheberrechtsgesetz überhaupt
viele Bibliotheksangebote zu blockieren, die nurmehr nach Einführung
elektronischer Leseplätze aufgekommen sind und noch weiter aufkommen
werden.

Auch diese Erfahrungen aus dem Wissenschaftsurheberrecht lassen daran
zweifeln, dass ein reger und stetig aktiver Gesetzgeber mehr
Rechtssicherheit bewirken kann, ohne sich in gesetzlichen
Momentaufnahmen zu erschöpfen, während das Gros urheberrechtlicher
Sachverhalte auf allgemeine Rechtsprinzipien -- etwa das Analogieverbot
für Schrankenbestimmungen -- zurückgeworfen ist und damit häufig der
allgemeinen Genehmigungs- bzw. Lizenzpflichtigkeit unterliegt.

\hypertarget{primat-der-legislative}{%
\subsection{Primat der Legislative}\label{primat-der-legislative}}

In Deutschland ist die Rechtsetzung zuvorderst der gewählten
Volksvertretung zugewiesen. Das Parlament hat die höchste demokratische
Legitimation in unserer gewaltenverschränkten Staatsordnung. Dennoch ist
auch das deutsche Parlament nur ein Akteur in einem korporativen
politischen Meinungsbildungsprozess. In rechtsformeller Hinsicht gelten
gerade im Wissenschafts- und Bibliotheksbereich die starken
Mitwirkungsrechte der Bundesländer, zudem sind insbesondere in Hinblick
auf die Eigentumsgarantie verfassungsrechtliche Vorgaben\footnote{Allg.
  zur immer wieder als Ersatzgesetzgeber kritisierten Rolle des
  Bundesverfassungsgerichts in Karlsruhe vergleiche zum Beispiel \emph{Seils}, Die
  Überpolitiker aus Karlsruhe, in: Tagesspiegel vom 04.03.2013,
  \url{http://www.tagesspiegel.de/meinung/bundesverfassungsgericht-die-ueberpolitiker-aus-karlsruhe/7873218.html}.}
sowie internationale Abkommen\footnote{Siehe unten I.9.} einzuhalten.

Dennoch liegt es keineswegs auf der Hand, dass anders strukturierte
Gesetzgebungsverfahren zu präferieren wären. Für das Urheberrecht kann
vergleichshalber die Europäische Union betrachtet werden.\footnote{Zur
  Vertiefung empfehlenswert \emph{Callies/Beichelt}, Auf dem Weg zum
  Europäisierten Bundestag: Vom Zuschauer zum Akteur? (Studie im Auftrag
  der Bertelsmann Stiftung), vorgestellt am 27.02.2013,
  \url{http://www.bertelsmann-stiftung.de/cps/rde/xbcr/SID-82ADC5CD-D555AA1C/bst/xcms_bst_dms_37438_37439_2.pdf}.}
Dort hat die Exekutive, die Europäische Kommission, ein Monopol für
Gesetzesinitiativen.\footnote{Zu \enquote{Commission's so-called
  monopoly of initiative} vergleiche \emph{Kuhlen}, Copyright Issues in the
  European Union -- Towards a science- and education-friendly copyright,
  Preprint verfügbar unter
  \url{http://www.kuhlen.name/MATERIALIEN/Publikationen2013/RK-copyright-issues-in-the-EU-submitted-preprint05032013-PDF.pdf}
  (letzter Abruf 05.03.2013).} Aus demokratietheoretischer und aus der
Legitimationsperspektive kann dieses Modell hinterfragt werden, im
Übrigen macht sich auch bei der Europäischen Kommission ein, obschon
eher informeller, Korporatismus bemerkbar, der eine aktive, moderne
Rechtsetzung verzögert.\footnote{Siehe zum Beispiel EU-Kommission, Commission
  agrees way forward for modernising copyright in the digital economy,
  Memo 12/950 vom 05.12.2012
  (\url{http://europa.eu/rapid/press-release_MEMO-12-950_en.htm?locale=en});
  vergleiche dazu zum Beispiel Kritik \enquote{EU und Urheberrecht - vorwärts im
  Schneckentempo} am 06.12.2012 von Initiative Urheberrecht --
  Kreativität ist was wert,
  \url{http://www.urheber.info/aktuelles/2012-12-06_eu-und-urheberrecht-vorwaerts-im-schneckentempo}.}

\hypertarget{interessenvielfalt-in-parteien-und-parlamenten}{%
\subsection{Interessenvielfalt in Parteien und
Parlamenten}\label{interessenvielfalt-in-parteien-und-parlamenten}}

Ähnlich wie in der Europäischen Union ist auch in Deutschland das
Urheberrecht eine Domäne der Wirtschafts- und Rechtspolitik. Bevor noch
Wissenschafts- und Bildungspolitiker zu Wort kommen, erhebt die
Kulturpolitik ihre Stimme. Weitere Politikbereiche sind angesprochen,
wie etwa der Verbraucherschutz und das Arbeitsrecht.

Die damit verbundenen Auseinandersetzungen reichen längst bis tief in
die einzelnen Parteien hinein. Bei den beiden Volksparteien sind die
Flügel und fachpolitischen Zirkel etabliert, doch auch innerhalb der
anderen Parteien können Fliehkräfte nicht übersehen werden. Exemplarisch
benannt sei insoweit die Partei Bündnis 90/Die Grünen mit ihren
Bundestagsabgeordneten Agnes Krumwiede (Kultur)\footnote{Vergleiche Interview
  in F.A.Z. vom 17.04.2012 (S.~25) mit beispielsweise folgenden Aussagen
  Agnes Krumiwiedes: \enquote{Die Auflösung des Rechts auf Schutz
  geistigen Eigentums wäre ein Verstoß gegen die Menschenrechte.},
  \enquote{Dass Urheber ohne ihre Verwerter grundsätzlich besser
  gestellt wären, gehört genauso zur Mythenbildung wie die Annahme,
  Urheberrechte würden ohnehin nur den Verwertern nutzen.},
  \enquote{Forderungen wie nach einer Verkürzung der urheberrechtlichen
  Schutzfristen bedienen in erster Linie die Interessen großer
  Internetkonzerne.},
  \url{http://www.faz.net/aktuell/feuilleton/debatten/gruenen-politikerin-krumwiede-zum-urheberrecht-die-piraten-verstehen-nicht-11719715.html}.},
Jerzy Montag (Rechtspolitik)\footnote{Unter der Leitung Jerzy Montags
  legte die Projektgruppe Urheberrecht der Bundestagsfraktion Bündnis
  90/Die Grünen am 04.03.2013 ihren Abschlussbericht vor,
  \url{http://jerzy-montag.de/home/volltext-startseite/article/abschlussbericht_der_projektgruppe_urheberrecht/}.},
Dr.~Konstantin von Notz (Innen- und Netzpolitik)\footnote{Vergleiche Interview
  mit Konstantin von Notz bei iRights.info vom 06.09.2012 mit dem Titel
  \enquote{Es geschieht nichts},
  \url{http://irights.info/von-notz-zum-urheberrecht-es-geschieht-nichts/7277}.}
und Krista Sager (Wissenschaft und Forschung)\footnote{Vergleiche zur Arbeit
  Krista Sagers im Ausschuss für Bildung, Forschung und
  Technikfolgenabschätzung des Deutschen Bundestags zum Beispiel
  \emph{Hartmann}, Einstimmige Agenda für ein innovationsfreundliches
  Urheberrecht, in: zwd Bildung.Gesellschaft.Kultur \& Politik Nr.
  1/2013, S.~18/19,
  \url{http://www.ip.mpg.de/files/pdf2/Hartmann_zwd_1-2013.pdf}).}.

\hypertarget{urheberrecht-ist-bundeskompetenz}{%
\subsection{Urheberrecht ist
Bundeskompetenz}\label{urheberrecht-ist-bundeskompetenz}}

Über Bibliotheken, über Universitäten und Hochschulen, über Schulen und
weithin auch über die Forschungseinrichtungen entscheiden die
Bundesländer. Auch liegt es an den Länderfinanzministern, Schulen,
Hochschulen und Bibliotheken finanziell auszustatten. Das Urheberrecht
hingegen fällt in die alleinige Kompetenz des Bundes. Die im Bereich
Wissenschaft und Bildung zuständigen Bundesländer haben somit eine
Urheberrechtspolitik des Bundes auszuführen und zu budgetieren, ohne an
dieser beteiligt zu sein. So fallen die Verantwortlichkeiten für Bildung
und Wissenschaft sowie für das Urheberrecht auseinander, ferner wird dem
Gedanken nach das Konnexitätsgebot unserer föderalen Staatstruktur
durchbrochen, indem die Bundesländer die Urheberrechtspolitik des Bundes
an den Hochschulen, Schulen und Bibliotheken finanzieren müssen. Diese
Divergenzen hemmen klare Verantwortlichkeiten der politischen Akteure
und letztlich eine moderne Urheberrechtsgesetzgebung für den Bildungs-
und Wissenschaftssektor.

\hypertarget{europuxe4isches-urheberrecht-ist-fast-ausschlieuxdflich-wirtschaftsrecht}{%
\subsection{Europäisches Urheberrecht ist (fast) ausschließlich
Wirtschaftsrecht}\label{europuxe4isches-urheberrecht-ist-fast-ausschlieuxdflich-wirtschaftsrecht}}

In Europa sind es Wirtschafts- und Rechtspolitiker, nicht Wissenschafts-
und Bildungspolitiker, die über das Urheberrecht entscheiden. Die
Urheberrechtsdebatte wird nicht selten in einem Atemzug geführt,
beispielsweise mit dem Kampf gegen Produktpiraterie. Bei der
Europäischen Union ist das Bewusstsein besonders ausgeprägt,
Urheberrecht als Wirtschaftsrecht des gemeinsamen Binnenmarkts zu
stärken (siehe nur das inzwischen verworfene Anti-Piraterie-Abkommen
ACTA)\footnote{Zur Fortsetzung der ACTA-Diskussion vergleiche zum Beispiel
  \emph{Mühlbauer}, Nur den Namen ACTA besiegt, in: Telepolis vom
  15.02.2013, \url{http://www.heise.de/tp/artikel/38/38570/1.html}.}. So
lautet Erwägungsgrund eins in der Richtlinie 2001/29/EG zur
Harmonisierung bestimmter Aspekte des Urheberrechts und der verwandten
Schutzrechte in der Informationsgesellschaft:

\begin{quote}
\enquote{Der Vertrag sieht die Schaffung eines Binnenmarkts und die
Einführung einer Regelung vor, die den Wettbewerb innerhalb des
Binnenmarkts vor Verzerrungen schützt. Die Harmonisierung der
Rechtsvorschriften der Mitgliedstaaten über das Urheberrecht und die
verwandten Schutzrechte trägt zur Erreichung dieser Ziele bei.}
\end{quote}

\noindent Mit jenem binnenmarktorientierten Selbstverständnis der Europäischen
Union bleiben Wissenschaft und Bildung zeitgemäße
Urheberrechtsvorschriften nicht selten verwehrt. Die Verantwortung, auch
wichtige Weichenstellungen für Bildung, Wissenschaft und
Informationsversorgung zu treffen, hat sich in Brüssel und Luxemburg
noch nicht ausreichend herumgesprochen. Dies wäre aber Voraussetzung, um
über angemessene Anforderungen an ein Wissenschaftsurheberrecht zu
diskutieren, welches nicht wirtschaftspolitisch vom Binnenmarktgedanken
dominiert wird.

Nicht unerwähnt bleiben darf, dass der europäische Gesetzgeber
ausgerechnet im Bereich des für Wissenschaft, Bildung und Bibliotheken
zentralen Schrankenkatalogs ausdrücklich keine Harmonisierung anstrebt.
Intranet-Plattformen für Lehre und Forschung zum Beispiel unterliegen
daher weiterhin selbst innerhalb europäischer Grenzen einem deutschen,
französischen, italienischen et al.~Flickenteppich.

\hypertarget{internationalisierung-des-rechts}{%
\subsection{Internationalisierung des
Rechts}\label{internationalisierung-des-rechts}}

Über fast allen Entscheidungen, die der Rechtsanwender, der Gesetzgeber
oder der Richter in Deutschland treffen, schwebt das Damoklesschwert der
Europa- und Völkerrechtskonformität. Allein der sogenannte
Drei-Stufen-Test\footnote{Vergleiche dazu näher zum Beispiel Declaration "A
  balanced interpretation oft the ‚Three-Step-Test' in copyright law,
  \url{http://www.ip.mpg.de/de/pub/aktuelles/declaration-threesteptest.cfm\#i34972}.}
ist in einer Art Welturheberrecht zementiert, ebenso wie eine
urheberrechtliche Mindestschutzdauer von 50 Jahren ab Tod der
UrheberInnen. Beim aktuellen Musterprozess um die elektronischen
Leseplätze in öffentlichen Bibliotheken könnte der Drei-Stufen-Test eine
Grenzlinie für die zulässigen Nutzungen elektronischer
Bibliotheksangebote einziehen -- ob diese Fesseln nun der europäische
und der deutsche Gesetzgeber so gewollt haben, steht auf einem anderen
Blatt. Selbst der angerufene Europäische Gerichtshof wird sich dem
Drei-Stufen-Test nicht entziehen können.\footnote{Dazu \emph{Hartmann},
  Anmerkung zu BGH, Beschluss vom 20.09.2012 (Az. I ZR 69/11), den
  Musterprozess elektronische Leseplätze dem EuGH zur Vorabentscheidung
  vorzulegen, GRUR Heft 5/2013.}

\noindent Fakt ist: Änderungen und Anpassungen der völkerrechtlichen
Urheberrechtsabkommen ähneln einem Generationenunterfangen, zeichnen
sich de facto durch eine Art Ewigkeitsgarantie aus. Zugleich schränken
sie die Gestaltungsmöglichkeiten der nationalen und europäischen
Rechtsetzung deutlich ein.

\hypertarget{internationalisierung-des-publizierens}{%
\subsection{Internationalisierung des
Publizierens}\label{internationalisierung-des-publizierens}}

Die Trends zu elektronischem Publizieren und zur Konsolidierung der
Wissenschaftsverlage setzen sich fort. So gehört es in vielen Bereichen
zum Alltag, dass mehrere AutorInnen aus unterschiedlichen Staaten
kollaborativ digitale Publikationen erstellen und dazu mit einem
internationalen Verlag kooperieren. Die Rechtslage ist unübersichtlich,
in grosso modo wird man unterscheiden müssen zwischen der gesetzlichen
und der vertraglichen Rechtslage. Initial stellt sich die Frage, welches
Recht anwendbar ist. Zentral ist, ob zwingend gesetzliches Recht
eingreift. Nur ein einfach gelagertes Beispiel: Eine deutsche Autorin
publiziert bei einem internationalen Wissenschaftsverlag mit Sitz in den
USA. Im Autorenvertrag erfolgt eine Rechtswahl zugunsten amerikanischen
Rechts. Im Übrigen erfolgt vertraglich eine vollständige
Rechte-Übertragung an den Verlag. Könnte sich die deutsche Autorin
dennoch wirksam auf ein nach deutschem Urheberrechtsgesetz unabdingbaren
Zweitveröffentlichungsrecht\footnote{Referentenentwurf zur Änderung des
  deutschen Urheberrechtsgesetzes vor, abrufbar beim Eintrag des Weblog
  Digitale Linke unter
  \url{http://blog.die-linke.de/digitalelinke/referentenentwurf-zur-urheberrechtsreform-verwaiste-werke-zweitveroffentlichungsrecht/}.
  Unter anderem soll demnach ein unabdingbares
  Zweitveröffentlichungsrecht für überwiegend öffentlich finanzierte
  Lehre und Forschung geschaffen werden. Vergleiche dazu die Stellungnahme des
  Max-Planck-Instituts für Immaterialgüter- und Wettbewerbsrecht (MPI
  IP) vom 15.03.2013,
  \url{http://www.ip.mpg.de/files/pdf2/Stellungnahme-BMJ-UrhG_2013-3-15-def1.pdf}.}
berufen und es im Streitfall auch durchsetzen?\footnote{Lösungshinweise
  im IUWIS-Rechtsgutachten \enquote{Spezifische Fragen zum Auslandsbezug
  des geplanten Zweitveröffentlichungsrechts nach §\,38 Abs.~1 S.~3 und 4
  UrhG-neu} von Dr.~Silke v. Lewinski und RA Dorothee Thum vom
  08.06.2011,
  \url{http://www.iuwis.de/publikation/spezifische-fragen-zum-auslandsbezug-des-geplanten-zweitver\%C3\%B6ffentlichungsrechts-nach-\%C2\%A7-3}.}

\hypertarget{handlungsfelder-zur-stuxe4rkung-der-rechtssicherheit}{%
\section{Handlungsfelder zur Stärkung der
Rechtssicherheit}\label{handlungsfelder-zur-stuxe4rkung-der-rechtssicherheit}}

\hypertarget{civil-law-vs.-common-case-law}{%
\subsection{Civil Law vs.~Common (Case)
Law}\label{civil-law-vs.-common-case-law}}

Der weitreichendste Reformvorschlag für das deutsche Urheberrecht zielt
auf eine allgemeine Wissenschaftsschranke. Im Juli 2010 stellte das
Aktionsbündnis \enquote{Urheberrecht für Bildung und Wissenschaft} eine
konkrete Gesetzesformulierung vor.\footnote{Aktionsbündnis
  \enquote{Urheberrecht für Bildung und Wissenschaft}, Pressemitteilung
  06/10 vom 06.07.2010: Ein großer Schritt für Bildung und Wissenschaft
  -- in Richtung einer allgemeinen Wissenschaftsschranke im
  Urheberrecht,
  \url{http://www.urheberrechtsbuendnis.de/pressemitteilung0610.html}.}
Mit Abstufungen haben sich dieser Forderung inzwischen die Allianz der
Wissenschaftsorganisationen, die Kultusministerkonferenz (KMK) und
politische Parteien angeschlossen. Der Deutsche Bibliotheksverband (dbv)
begründet seine Unterstützung damit, dass eine allgemeine
Wissenschaftsschranke \enquote{zu mehr Rechtssicherheit führen und
Streit über erlaubte Nutzungsarten minimieren}\footnote{Deutscher
  Bibliotheksverband e.V. (dbv), Positionspapier vom 20.04.2012:
  Nutzerinteressen stärken, Urheberrechte wahren,
  \url{http://www.bibliotheksverband.de/fileadmin/user_upload/DBV/positionen/2012_04_20_dbv-Positionspapier_Urheberrecht.pdf}.}
würde.

Ob eine Fair-Use-ähnliche Generalklausel mehr Rechtssicherheit als die
geltenden gesetzlichen Detailvorschriften bewirken könnte, ist bislang
kaum absehbar. Solange es einen austarierten Schutz von UrheberInnen und
RechtsinhaberInnen geben soll, blieben auch bei einer allgemeinen
Wissenschaftsklausel sorgfältige fachjuristische Abwägungen
erforderlich. Neben den gelegentlich unerwähnten teils engen
Konditionierungen der Fallgruppen im US-Recht soll an dieser Stelle nur
an eine Schwachstelle des Case Law in puncto Rechtssicherheit erinnert
werden: Kein Fall ist in der Regel (juristisch) wie der andere.
Besonders innovative Nutzungsformen könnten sich nur näherungsweise auf
richterlich gefestigte Fallgruppen stützen. Der beklagte Mangel an
Rechtssicherheit würde nach dem Systemwechsel nicht mehr zuvorderst
allein dem Gesetzgeber, sondern auch der Rechtsprechung angelastet.

\hypertarget{subsidiuxe4re-uxf6ffnungsklausel-bei-technologischen-fortschritt-bzw.-bei-veruxe4ndertem-mediennutzungsverhalten}{%
\subsection{Subsidiäre Öffnungsklausel bei technologischen Fortschritt
bzw. bei verändertem
Mediennutzungsverhalten}\label{subsidiuxe4re-uxf6ffnungsklausel-bei-technologischen-fortschritt-bzw.-bei-veruxe4ndertem-mediennutzungsverhalten}}

Derzeit sind die Schrankenregelungen zugunsten von Nutzerinnen und
Nutzern und den vermittelnden Einrichtungen in Deutschland und Europa
zementiert. Die \enquote{Magna Charta} des deutschen Urheberrechts, die
Privatkopie (§\,53 UrhG), offenbart dies, gerade im Zusammenspiel mit dem
elektronischen Kopienversand öffentlicher Bibliotheken (§\,53a UrhG). Die
Rechtslage ähnelt einer Drei-Klassen-Gesellschaft. Zuerst gilt seit
Inkrafttreten des Urheberrechtsgesetzes anno 1965, dass für den privaten
Gebrauch analog einzelne Vervielfältigungen erstellt werden dürfen. Vor
einigen Jahren erlaubte der deutsche Gesetzgeber auch elektronische
Kopien, allerdings unter -- gerade im Bereich der öffentlichen
Bibliotheken -- drastischen Restriktionen. Neuere Formen der
Mediennutzung, wie zum Beispiel Streaming, erfasst die Schranke der
Privatkopie nicht, obwohl sie ein vergleichbares Nutzungsbedürfnis
befriedigen wie früher analoge Kopien. Prinzipiell sagt das Urheberrecht
bisher, dass der gesetzliche Schrankenkatalog mit etwa der Privatkopie
nicht erweitert und auch keine analoge oder teleologische
Rechtsfortbildung erfolgen darf. So drängt sich gelegentlich der
Eindruck auf, dass die Stärkung der Medienkompetenz der Bevölkerung
sowie der Zugang zu elektronischen Ressourcen in unserer modernen
Informations- und Wissenschaftsgesellschaft urheberrechtspolitische
Versprechen sind, die -- zumindest bisher -- allenfalls rudimentäre
Umsetzung in geltendes Recht erfahren haben. Es wird vielmehr deutlich,
dass Medienkompetenz und ein möglichst souveräner Umgang mit neuen
Medien auch ein breites Wissen über rechtliche Details einfordern.

Wenn der Gesetzgeber keinen radikalen Systemwandel wie oben angesprochen
vornehmen möchte\footnote{Siehe oben I.2. und II.1.}, könnte sich anbieten,
den starren Schrankenkatalog mit einer Öffnungsklausel zu dynamisieren.
Eine solche Rechtssystematik ist in anderen Bereichen des
Immaterialgüterrechts, wie zum Beispiel im Wettbewerbsrecht, seit langem
anerkannt. So lautet §\,3 Absatz 1 des Gesetzes gegen den unlauteren
Wettbewerb (UWG):

\begin{quote}
\enquote{Unlautere geschäftliche Handlungen sind unzulässig, wenn sie
geeignet sind, die Interessen von Mitbewerbern, Verbrauchern oder
sonstigen Marktteilnehmern spürbar zu beeinträchtigen.}
\end{quote}

\noindent Eine dahingehende Öffnungsklausel im Urheberrecht wäre, zumindest in
ihrer ersten Phase, zurückhaltend mit mehreren Bremsen ausgestattet.
Dazu dienen insbesondere der internationale Drei-Stufen-Test zum Schutz
der RechtsinhaberInnen\footnote{Siehe oben I.9.}, sowie in Deutschland die
verfassungskonforme Anwendung der Öffnungsklausel im Lichte von Artikel 14
Grundgesetz (Eigentumsgarantie).

Friktionen der Co-Existenz zweier Regelungsmodelle deuten sich
allerdings an: Könnte etwa für neue Nutzungsformen \emph{subsidiär} die
Öffnungsklausel herangezogen werden, wenn analoge Vorgängernutzungen in
gesetzlichen Schrankenbestimmungen abgebildet sind? Es ist eine
Anforderung an die Urheberrechtswissenschaft, aus divergierenden
Regelungstechniken resultierende Spannungsfelder für Rechtsprinzipien
wie Subsidiarität und Rechtsfortbildung anhand der Analogie (siehe oben die
Problematik bei Privatkopie und elektronischem Kopienversand) zu
untersuchen und aufzulösen. Wertvoll ist daher der konkrete Vorschlag
für eine hybride, dynamische Abbildung der gesetzlichen
Nutzerbefugnisse, welcher die ExpertInnen der WITTEM-Group\footnote{\url{http://www.copyrightcode.eu/}.}
mit dem European Copyright Code\footnote{\url{http://www.copyrightcode.eu/index.php?websiteid=3}.}
im Jahr 2010 vorlegten. Eine als \enquote{Further limitations}
bezeichnete Öffnungsklausel ist in Artikel 5.5 des European Copyright
Code formuliert:

\begin{quote}
\enquote{Any other use that is comparable to the uses enumerated in art.
5.1 to 5.4(1) is permitted provided that the corresponding requirements
of the relevant limitation are met and the use does not conflict with
the normal exploitation of the work and does not unreasonably prejudice
the legitimate interests of the author or rightholder, taking account of
the legitimate interests of third parties.}\footnote{Siehe vor allem
  auch die Erläuterung unter Fußnote 48,
  \url{http://www.copyrightcode.eu/index.php?websiteid=3\#_ftn48}.}
\end{quote}

\hypertarget{klarheit-und-verstuxe4ndlichkeit-von-urheberrechtsnormen}{%
\subsection{Klarheit und Verständlichkeit von
Urheberrechtsnormen}\label{klarheit-und-verstuxe4ndlichkeit-von-urheberrechtsnormen}}

Neben Rechtssicherheit an der Spitze der grundlegend geforderten
Verbesserungen des Urheberrechts rangieren dessen Klarheit und
Verständlichkeit. Konkrete Vorschläge, wie das Urheberrecht
verständlicher in Gesetzesform und nachfolgende Lizenzterminologie
gegossen werden könnte, sucht man vergeblich. Erfolgreiche Modelle in
der Rechtsanwendung setzen auf eine Doppelstrategie, die Lizenzmodule
von Creative Commons gar auf eine Dreifachstrategie. Es erfolgen
separate Darstellungen für NutzerInnen sowie für maschinelle
Nachnutzungen. Neben dieser juristisch verknappten
\enquote{Lizenzversionen} steht jedoch ein traditionell fachjuristisch
elaborierter Lizenzvertragstext.

Zu Lasten der Normenklarheit und -verständlichkeit gehen
Differenzierungsanforderungen. So ist eine Unterscheidung etwa nach
einfachen und ausschließlichen Nutzungsrechten für das Lizenzmanagement
der AutorInnen, Verlage, NutzerInnen, Bibliotheken, Wissenschafts- und
anderen Informationseinrichtungen unverzichtbar. Urheberrechtliche
Fachterminologie mitsamt Vokabular schafft insofern einen kohärenten
Referenzrahmen, an den auch moderne Lizenzmodelle wie eben Creative
Commons anknüpfen können. Verlässlich sind die gesetzlich formulierten
Standards zum Beispiel auch, wenn Fachleute aus dem Bibliothekswesen und
Informationsmanagement Publikationsberatung leisten. Wenn in einem
Lizenzvertrag \enquote{einfache} Nutzungsrechte eingeräumt werden, kann
bei AutorInnenberatungen auf eindeutige rechtliche Bedeutung und
Rechtsfolgen vertraut werden.

\noindent Weithin fehlende Standardisierung behindert bislang
Zweitveröffentlichungen (Open Access Green Road). Zwar gestatten Verlage
-- oftmals individuell auf Zuruf -- als Rechtsinhaber eine
Zweitveröffentlichung, die fehlende Vereinheitlichung ihrer
Voraussetzungen und Rechtsfolgen behindert jedoch vielerorts den
lizenzrechtlich verlässlichen, strukturierten Aufbau von Repositorien
und anderen Dokumentenservern\footnote{Im Einzelnen verhandelt werden
  zum Beispiel die sog. Embargo-Dauer, die Einschränkung der
  Zweitveröffentlichungsbefugnis auf persönliche Internetseiten der
  AutorInnen oder die Pflicht, (nur) eine bestimmte Publikationsversion
  (Pre-Print, Post-Print, akzeptierte Autorenversion, Verlagslayout
  etc.) zu verwenden. Siehe dazu auch den Referentenentwurf des
  Bundesjustizministeriums vom 20.02.2013 für ein unabdingbares
  Zweitveröffentlichungsrecht im Bereich überwiegend öffentlich
  finanzierter Lehre und Forschung (siehe oben unter Fußnote 9). Die gegebenenfalls für
  diese gesetzliche Fallgruppe standardisierten
  Zweitveröffentlichungsbedingungen diskutiert \emph{Kuhlen}, Ein
  Referentenentwurf für das Zweitverwertungsrecht, aber wohl noch kein
  Ende der Debatte, IUWIS vom 21.02.2013,
  \url{http://www.iuwis.de/blog/ein-referentenentwurf-f\%C3\%BCr-das-zweitverwertungsrecht-aber-wohl-noch-kein-ende-der-debatte}.}
und verunsichert AutorInnen aufgrund der ungewissen
Rechtslage.\footnote{Laut PEER-Verhaltensstudie empfinden AutorInnen
  rechtliche Unklarheit bei den Embargo-Fristen und den
  Zweitveröffentlichungsbedingungen als wesentliche Hürden, um ihre
  Publikationen auch gemäß Open Access Green Road verfügbar zu machen,
  vergleiche PEER Behavioural Research - Baseline report (01.02.2010), S. 35,
  \url{http://www.peerproject.eu/fileadmin/media/reports/Final_revision_-_behavioural_baseline_report_-_20_01_10.pdf}.}

Ferner ist zweifelhaft, ob allgemein akzeptierte Elementaranforderungen
an Recht\footnote{Siehe oben I.} mit Forderungen nach möglichst radikaler
Vereinfachung zu vereinbaren sind. In Ergänzung zu den Hinweisen oben
unter I. kann auf das Steuerwesen oder auf das Straßenverkehrsrecht
verwiesen werden. Gemessen an Qualitätskriterien für Rechtsetzung wie
Bestimmtheit, Berechenbarkeit und Klarheit von Normen stehen steuer-
oder straßenverkehrsrechtliche Vorschriften in ihrer bloßen Komplexität
urheberrechtlichen Normen nichts nach. Im Urheberrechtsgesetz
veranschaulicht allein die seitenfüllende Privatkopie-Vorschrift des §\,53\footnote{Siehe oben II.2.} ein kaum zu durchdringendes Dickicht an
Voraussetzungen, Einschränkungen, Ausnahmen, Gegenausnahmen und
variantenreichen Sonderfällen. Die Frage \enquote{Muss Recht einfach
sein?}\footnote{Dazu sogleich unten beim Steuerrecht noch näher.} ist
legitim. Stellt man sie für das Urheberrecht, so öffnet sich der Blick
jedoch auch auf die meisten anderen Rechtsbereiche. Die Aufmerksamkeit
gilt zunächst dem Straßenverkehrsrecht. §\,1 der Straßen\-verkehrs-Ordnung
(StVO) lautet:

\pagebreak

\begin{quote}
\enquote{Die Teilnahme am Straßenverkehr erfordert ständige Vorsicht und
gegenseitige Rücksicht. Jeder Verkehrsteilnehmer hat sich so zu
verhalten, daß kein Anderer geschädigt, gefährdet oder mehr, als nach
den Umständen unvermeidbar, behindert oder belästigt wird.}
\end{quote}

\noindent Die gesetzliche Grundregel ist vollständig wiedergegeben und beinhaltet
eine Verhaltensanweisung, die ohne fachjuristische Terminologie gut
verständlich ist. Wenn jedeR diese allgemeine Verhaltensregel einhielte,
würden Dutzende weiterer gesetzlicher Einzelbestimmungen in der StVO und
anderen relevanten Gesetzesbestimmungen überflüssig, oder?

Ähnlich stellt sich die Frage für das Steuerrecht, das uns gemäß
populärer Forderung in die Lage versetzen soll, unsere gesetzlichen
Steuerpflichten auf der Größe eines Bierdeckels selbst zu erklären.
Zugunsten von Kürze, Verständlichkeit und Rechtssicherheit weithin
zurückgedrängt würde beispielsweise jedoch das Kernanliegen unseres
Steuersystems, nämlich das einer leistungsgerechten Besteuerung. Zur
Zielkonkurrenz bezieht Michael Sell aus dem Bundesfinanzministerium in
einem Meinungsbeitrag im Handelsblatt vom 26.02.2013 folgendermaßen
Stellung:

\begin{quote}
\enquote{Immer ertönt der Ruf nach einfachen Steuern. Komplexe Steuern
schließen aber Handhabbarkeit ebenso wenig aus wie komplizierte Autos
oder Handys.} Neben seiner Differenzierung zwischen einfachem
Steuerrecht und seiner einfachen Handhabung (ähnlich der oben
beschriebene Ansatz bei Creative Commons) sowie den (an sich
erwünschten) Lenkungswirkungen von (Steuer-)Recht mahnt der leitende
Fiskus-Steuerbeamte: \enquote{Wenn der Bürger zugleich auch etwas
weniger nach seiner individuell verstandenen Einzelfallgerechtigkeit
verlangt, kommen beide Seiten ein gutes Stück weiter.}\footnote{\emph{Sell},
  \enquote{Votum}-Beitrag \enquote{Müssen Steuern einfach sein?},
  Handelsblatt (S.~13) vom 26.02.2013.}
\end{quote}

\hypertarget{schlieuxdfung-urhebergesetzlicher-luxfccken-im-wissenschafts--bildungs--und-bibliothekswesen}{%
\subsection{Schließung urhebergesetzlicher Lücken im Wissenschafts-,
Bildungs- und
Bibliothekswesen}\label{schlieuxdfung-urhebergesetzlicher-luxfccken-im-wissenschafts--bildungs--und-bibliothekswesen}}

Nicht nur wegen der bereits aufgezeigten Schwachstellen besteht Skepsis,
ob Rechtssicherheit absolut so erstrebenswert ist, dass dafür die
zweifelsohne partiell bestehenden Gesetzeslücken umfassend geschlossen
werden sollten.

Die Kraft der Wissenschaft -- den Wissenschaftsdiskurs eingeschlossen --
zur Selbstregulierung ist juristisch anerkannt. Die grundgesetzlich
verbürgte Wissenschaftsfreiheit kennzeichnet wesentlich
Hochschulautonomie und Selbstverwaltung der Wissenschaftseinrichtungen.
Alles in allem hat es sich bewährt, dass sich die Wissenschaft ohne
Zutun des Gesetzgebers organisiert und strukturiert. Soft Law,
Richtlinien, Policies, eine Vielfalt an Institutsordnungen, vor allem
aber ungeschriebene \enquote{Gesetze} und tradierte
(Publikations-)Grundsätze haben wissenschaftliches Arbeiten, Innovation,
aber auch die Interessen eingebundener Partner, wie zum Beispiel Verlage
und Arbeitgeber, befördert und zugleich verlustfrei an sich
urheberrechtliche Aufgaben erfüllt.

Im allgemeinen Bibliothekswesen findet eine partnerschaftliche
Selbstregulierung seit langem die höchste Akzeptanz unter den
Beteiligten. Geräuschlos ohne Gesetzgeber treffen Bibliotheksvertretung
und Verwertungsgesellschaften beziehungsweise Börsenverein des deutschen
Buchhandels Gesamtvereinbarungen, um die Ausleihe an deutschen
Büchereien urheberrechtlich abzubilden. Mit der sogenannten
Bibliothekstantieme (Bibliotheksgroschen) verschaffen sich die
Beteiligten selbst verlässliche und zugleich gestaltbare Arbeits- und
Vergütungsgrundlagen für ihren Kern- und Massenbetrieb. Das
zugrundeliegende gesetzliche Konstrukt Verleihrecht (§§\,17 und 27 UrhG)
tritt in den Hintergrund, bleibt Spezialdomäne. In diesem Sinne --
dahingehend auch die kürzlich näher begründete Auffassung des
Verfassers\footnote{Weiterverkauf und \enquote{Verleih} online
  vertriebener Inhalte -- Zugleich Anmerkung zu EuGH, Urteil vom 3. Juli
  2012, Rs. C-128/11 -- UsedSoft ./. Oracle. In: GRUR Int. 2012, S.~980--989.} -- sollten die Beteiligten ernsthafter als bislang prüfen,
ob mit einer fairen und innovativen Selbstregulierung nicht auch
Herausforderungen wie \enquote{Verleih}-Modelle für digitale Ressourcen
am besten bewältigt werden könnten.

Stärker als bisher Optionen der Selbstregulierung einzubeziehen,
beleuchtet aus sozialwissenschaftlicher Perspektive \emph{Jeanette
Hoffmann}. Soziale Normen übernehmen die Funktionen des Urheberrechts in
ganzen kreativen Schaffensbranchen, wie sie für den Handel mit
TV-Formaten, Dojinshis oder Stand-up Comedy festzustellen
glaubt.\footnote{Vergleiche \emph{Hoffmann/Katzenbach/Münch},
  Kulturgütermärkte im Schatten des Urheberrechts -- zur Pluralität
  praktizierter Regelungsformen, in: APUZ 41-42/2012,
  \url{http://www.bpb.de/apuz/145384/\%0Dkulturguetermaerkte-im-schatten-des-urheberrechts}.}

Im Kontext des Exzellenzclusters Normative Ordnungen (Goethe-Universität
Frankfurt am Main) beschreibt der Jura-Professor \emph{Alexander
Peukert} eine \enquote{Eigengesetzlichkeit} der Wissenschaft. Seinem
Modell zufolge könnte ein verlagsfreies wissenschaftliches
Kommunikationssystem mithilfe (neuer) wissenschaftsinterner Regeln
realisiert werden, ohne dass dafür die Urheberrechtsordnung zu
reformieren wäre.\footnote{Vergleiche \emph{Peukert}, Das Wissenschaftsdilemma
  im Urheberrecht: Modelle für die Verwertung von und den Zugang zu
  wissenschaftlichen Werken, Vortragsdokumentation vom 19.01.2013 (insbesondere
  Vortragsfolien 10 und 11),
  \url{http://www.jura.uni-frankfurt.de/44705249/13-01-18-Peukert-Vortrag-Bayreuth-Wissenschaft-und-Urheberrecht.pdf\#Das\%20Wissenschaftsdilemma\%20im\%20Urheberrecht}.}

\hypertarget{fazit}{%
\section{Fazit}\label{fazit}}

Der Beitrag diskutiert mehrere Verfahrensgrundsätze und
Rechtsprinzipien, die elementare Bestandteile unseres demokratischen
Rechtsstaats sind. Sie stehen teils komplementär, teils aber auch
konkurrierend zu Rechtssicherheit. Vollkommene Rechtssicherheit sollte
deshalb in unserer Gesellschaft nicht immer die höchste Priorität
einnehmen.

Für persönliches Handeln ebenso wie für professionelles Arbeiten ist die
Forderung nach möglichst rechtssicheren Gesetzesbestimmungen
verständlich. Oftmals enthalten Rechtsvorschriften normative Leitlinien
sowie Wertentscheidungen für (drohende) Konflikte. Im Übrigen wird
100\,\%-ige Rechtssicherheit letztlich nur durch den gerichtlichen
Instanzenweg oder außergerichtliche Einigung zu erreichen sein. Die so
erlangte Rechtssicherheit erfordert hohen Einsatz und bleibt in ihrer
Strahlkraft zumeist auf den jeweils konkret vorgetragenen Einzelfall
begrenzt. Daher sollten -- gerade im Wissenschafts- und
Bibliotheksbereich -- mit gesetzlichen Unsicherheiten behaftete Fragen
als Gestaltungsaufgabe und Freiräume begriffen werden. Ehe der Ruf nach
(weiterer) Verrechtlichung erhoben wird, gilt es, bestehende Chancen,
Freiheiten sowie die damit einhergehende Verantwortung selbstbewusst
auszuloten, zu ergreifen und, soweit erwünscht auch rechtlich, selbst zu
gestalten.

\begin{center}\rule{0.5\linewidth}{0.5pt}\end{center}

%autor

\textbf{Thomas Hartmann} forscht zu Urheberrechtsfragen in der digitalen
Informationsgesellschaft am Max-Planck-Institut für Immaterialgüter- und
Wettbewerbsrecht (MPI IP), an der Max Planck Digital Library (MPDL),
beides in München, sowie an der Humboldt-Universität zu Berlin. Der
Verfasser dankt dem VG WORT Förderungsfonds für ein
Promotionsstipendium.

% {\sloppy\printbibliography[heading=subbibliography,notkeyword=this]}
\end{document}