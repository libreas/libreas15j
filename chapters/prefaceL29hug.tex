\documentclass[output=paper]{langsci/langscibook} 
\title{Das Warum} 
\author{Eva Bunge}  

\abstract{Marius Hug, Benjamin Fiechter, Christian Thomas, \enquote{Texte über fallende Steine. Alexander von Humboldts Praktiken wissenschaftlichen Arbeitens am Beispiel der Mondvulkane}. LIBREAS. Library Ideas, 29 (2016). \url{https://libreas.eu/ausgabe29/08hug/}}

\begin{document}
\maketitle

\noindent Für jemanden wie mich, die sich im Studium schwerpunktmäßig mit Astronomie beschäftigt hat und nun in einer Museums- und Spezialbibliothek für Wissenschafts- und Technikgeschichte arbeitet, übt ein Aufsatz über Alexander von Humboldt und rätselhafte, vom Himmel fallende Steine naturgemäß eine gewisse Faszination aus. Die Autoren kombinieren auf unterhaltsame Weise historische Forschung mit Detektivarbeit. Abgesehen vom inhaltlichen Thema hat mir dabei besonders gefallen, wie sie als Wissenschaftler ganz selbstverständlich auf die digitalen Dienstleistungen der Bibliotheken zurückgreifen – Recherchemöglichkeiten und Digitalisate aus verschiedensten Quellen kommen zum Einsatz und sind so selbstverständlich, dass sie keinen besonderen Kommentar wert sind. In den Digitalen Geisteswissenschaften zumindest scheint damit die digitale Bibliothek ebenso alltäglich zu sein wie die analoge – eine durchaus beruhigende Erkenntnis im bibliothekarischen Alltag, wenn man sich doch immer mal wieder über der Sinnhaftigkeit der eigenen Arbeit wundert.

\end{document}
