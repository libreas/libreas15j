\documentclass[output=paper]{langscibook}

\title{Das Knistern zwischen den Bücherregalen}
\author{Thomas Helbing}

\abstract{
Zitiervorschlag: Thomas Helbing, \enquote{Das Knistern zwischen den Bücherregalen}. LIBREAS. Library Ideas, 13 (2008). <https://libreas.eu/ausgabe13/005_a_hel.htm>

URN: \href{http://nbn-resolving.de/urn:nbn:de:kobv:11-10091583}{urn:nbn:de:kobv:11-10091583}

DOI: \url{http://doi.org/10.18452/8904}}

\begin{document}

%abstracts

%body
\emph{Der Beruf des Bibliothekars erlebt eine Renaissance. Deren
verschrobener Modestil inspiriert die Modedesigner; Streberlook und
Krankenkassenbrillengestelle gelten als hipp. Warum?}

Wie riecht es in einer Bibliothek? Nach einem englischen Roman,
russischem und marokkanischem Leder, getragener Kleidung und einer Spur
Holzpflegepolitur würde der Parfümeur Christopher Brosius antworten. Das
ganze trägt dann den logischen Namen \emph{In the Library} und ist ein
Parfüm, das tatsächlich aufgesprüht werden kann.

Virginia Woolf hätte es sicher gefallen. Nur wer würde das heute noch
auf seiner Haut tragen? Und wie kommt man darauf, einen Duft zu
kreieren, mit dem man eine verstaubte Bücherhalle assoziiert? Dazu
Brosius: ``Duft ist Leben''. Da hat anscheinend jemand gut seine Umwelt
recherchiert. Denn gerade in Brosius' Nachbarschaft, im hippen
Williamsburg in New York, wo der Parfümeur unter seinem Label I hate
perfume noch mehr obskure Duftelixiere in einem kleinen Shop anbietet,
versammelt sich derzeit eine sehr lebendige Szene aus jungen,
selbstbewussten Bibliothekaren, die sagen ``ja wir sind genauso cool wie
Werbetexter oder Grafikdesigner.''

Prompt erschien darüber kürzlich ein Artikel in der Style-Sektion der
renommierten New York Times.\footnote{``A Hipper Crowd of Shusher'', The
  New York Times, 8. Juli 2007.
  \url{http://www.nytimes.com/2007/07/08/fashion/08librarian.html?_r=1\&pagewanted=all}}
Und der räumte gründlich auf mit den alten Vorurteilen über
Bibliothekare, diese in unseren Augen verbitterten, etwas spröden,
grauen Mäuse, die lieber ihre Nasen zwischen verstaubte Buchdeckel statt
an die frische Luft stecken und mit strengem Blick durch
Kassenbrillengestelle sowie Finger-auf-dem-Mund-Geste zur Ruhe mahnen.
``Die heutigen Bibliothekaren-Generation ist absolut nicht mehr
vergleichbar mit dem klassischen Bild des Bibliotheksangestellten'',
sagt Rick Block, Professor für Bibliothekswissenschaft an der \emph{Long
Island University} und am \emph{Pratt Institut}. Und spielt darauf an,
dass der Beruf im Zeitalter von \emph{Google}, \emph{Myspace} und
\emph{iPhone} viel cooler, abwechslungsreicher und fortschrittlicher
geworden ist.{]}\{.text\}

Bibliothekare des 21. Jahrhunderts sind quasi die Melitta-Filtertüten
der globalen Informationsüberflutung aus Fernsehen, Zeitungen, Büchern
und dem World Wide Web. \emph{Google}, \emph{Facebook} und
\emph{Myspace} sind, neben wissenschaftlich-elektronischen Datenbanken,
längst gängiges Arbeitsinstrument zur Informationsrecherche,
Archivierung und Vernetzung. Und wenn Google-Gründer Larry Page und
Sergey Brin, Microsoft-Tycoon Bill Gates sowie Apple-Chef Steve Jobs wie
Popstars hofiert werden, nährt das auch das gesellschaftliche Comeback
der \emph{Librarian2.0}-Generation, wie die \emph{New York Times} die
Wiedergeborenen bezeichnete.

Ohne Nostalgie geht's dann aber doch nicht. Filme, wie die romantische
Komödie \emph{Desk Set} von 1957 mit Katharine Hepburn und Spencer
Tracy, \emph{It's a Wonderful Life} mit Donna Reed oder \emph{Foul Play}
mit Goldie Hawn, spielen mit den Klischees über Bibliothekare, zeigen
sie aber romantisch, süß oder als aufreizende Blondine. Die
\emph{Librarians2.0} verweisen genauso gerne darauf, wie auf die nerdy
Comic-Leinwandhelden Superman, Spiderman oder Batwoman, die tagsüber in
einer Leihbücherei arbeitet. Auch der Modefotograf Irving Penn bediente
sich gewisser nerdiger Elemente für seine Aufnahme mit dem Titel
\emph{The 1950}.

Von der Leinwand transportiert sich wohl auch bis heute in unseren
Köpfen der Kleidungsstil, den die Gesellschaft mit dem Bibliothekar
verbindet. Einen leicht antimodischen Stil, ein wenig altbacken,
skurril, wenig glamourös, immer ein wenig neben der Spur. Strickjacken
und Pullover kommen dabei über Blusen und Hemden -- natürlich mit langem
Feinripp darunter. Manchmal hängt der Strick auch locker über der
Schulter bzw. wird um den möglichst kniebedeckenden Faltenrock an der
Hüfte geknotet. Die Damen bedecken die Beine zusätzlich mit dicken
Strumpfhosen -- in Schwarz oder in bewusst gewählten Kontrastfarben. Von
Grau, Braun und dunklem Blau erstreckt sich die Farbpalette über Altrosa
bis hin zu einem Flaschengrün, wie dem der typischen Leselampenschirme
in altehrwürdigen Bibliotheken. Dazu kommen abstrakte 30er-Jahre Muster
und Blümchendrucke. Die Brille aus möglichst dickem, schwarzem oder
braunem Horn oder Metall ist ebenso wichtiges Identitätsmerkmal, wie die
zu einem lockeren Dutt geknoteten Haare.

Bibliothekare begegnen uns jedoch nicht nur in Film und Fernsehen. Heute
mutieren wir im privaten Alltag selbst zu welchen. Oder archivieren wir
nicht alle unsere Fotosammlung längst auf dem PC, haben eine
umfangreiche Musikbibliothek auf der Festplatte, dann auch Mediathek
genannt, und \emph{googlen} uns jede freie Minute durch unsere
Wissenslücken? Das geht sogar soweit, dass man sich plötzlich
aufdrängende Fragen per Handy beantworten lassen kann. Beispielsweise
von der Wissens-Community hiogi.de, die für ihren Service mit der
Kampagne ``Wissen ist sexy'' wirbt. Hiogi-CEO Björn Behrendt: ``Auf
Straßenplakaten, auf Edgar-Cards, im Radio und in unserer Web-Werbung
machen wir den Menschen klar, dass Wissen sexy ist. Und da doch jeder in
irgendeinem Bereich ein wertvolles Wissen hat, darf sich ab sofort jeder
sexy fühlen. Er muss nur aus seinem Elfenbeinturm heraustreten und sein
Wissen in die hiogi-Community einbringen.'' Dabei stellt die
Image-Kampagne gezielt Nerds in den Mittelpunkt, also jene Spezies
Mensch mit übersteigertem Wissen und im Look eines
Fünftklässler-Klassentrampels. Die Kampagne zeigt typische \emph{Nerds}
in Aldiletten, in Muttis Lieblingspullover und mit gegeltem
Mittelscheitel, die sich extrem sexy fühlen und sich entsprechend
sinnlich präsentieren. Passend dazu gibt es sogar einen
\emph{Nerdenizer} auf der Webseite wissen-ist-sexy.de zum Nachstylen des
typischen Looks. Das markante Hornbrillengestell gibt es gleich in vier
Styles.

Wie die Korrekturbrille ein wunderbares Styleelement sein kann, sah man
unlängst auf den Laufstegen der Modehauptstädte Mailand, Paris und New
York. Dolce \& Gabbana, Tim Hamilton oder Michael Kors beispielsweise
schickten ihre Models mit klassischen Modellen à la Clark Kent auf die
Runways. Ein Bild, dass sich auch in den Gesichtern der Popkultur
widerspiegelt, beispielsweise beim britischen Musiker Jarvis Cocker, den
Bands \emph{Hot Chip} und \emph{The Pipettes} oder dem derzeit
angesagtesten deutschen Produktdesigner Konstantin Grcic -- und längst
im täglichen Straßenbild als hipp gilt. Selbst ``Hollywood-Stars lieben
den Streber-Look'' titelte Bild-Online\footnote{``Hollywood-Stars lieben
  den Streber-Look'', Bild-Online, 8. April 2008.
  \url{http://www.bild.de/BILD/lifestyle/mode-beauty/mode/2008/04/stars-tragen-streber-brillen/trend-streber-brillen.htm}}
und entdeckte die Vorliebe für dicke, schwarze Hornbrillengestelle auf
den prominenten Näschen von Chloë Sevigny, Scarlett Johansson, Johnny
Depp und Nicole Kidman.

Eine Brise Retro war für Modedesigner schon immer die richtige Würze für
Inspiration. Die erste Designerin, die den \emph{librarian-style}
konsequent neu interpretierte, war die Italienerin Miuccia Prada (59).
In ihre Prêt-à-porter-Kollektionen für den Sommer 2001 zeigte sie
schmale Bleistift- und weit schwingende Faltenröcke, eng anliegende
Wolltops sowie kurze Jäckchen, Cardigans und Blazer für darüber, dazu
eine Farbpalette von Hellblau, Steingrau, Schwarz bis Grasgrün, Gelb und
Rosa. Der Modejournalist Armand Lymnander bescheinigte auf style.com dem
ganzen einen ``gewissen Schulmeisterinnen-Charme.''

Die Doyenne der Modekommentatoren, Suzy Menkes von der
\emph{International Herald Tribune}, nannte Prada einmal die
``Herrscherin über uns alle'', und meint damit Miuccia Pradas
einzigartige Fähigkeit, die Essenz des Modernen so zu destillieren,
kulturelles Erbe so gekonnt abzurufen und gesellschaftlichen Fortschritt
so präzise in Kleidung auszudrücken wie kein anderer Modedesigner. So
kann man Pradas Vision als Antagonist zu den mit Sex bestrichenen
Kleidchen eines Tom Ford für das Modehaus Gucci verstehen, dem damals
jeder hinterher hechelte. Schließlich steckte Prada in genau dieser
Saison auch die Männer in extrem taillenhohe, pludrige Bundfaltenhosen
im Stil von Hollywoodlegenden der 50er Jahre sowie kastenförmige
Doppelreiher-Sakkos mit darunter abstrakt-grafisch gemusterten Hemden,
alles in blassem Gelb, Beige und Grau.

Seitdem kehren nerdige Elemente immer wieder in Pradas Kollektionen auf.
2005 verwandeln Brillen Gesichter in gerahmte Kunstwerke, gekrönt von
Wollmützen. Der verpönte und gemeine
dicke-Wollsocken-in-Sandalen-zu-kurzen-Hosen-Look sollte dann im Sommer
2007 das gewisse Haben-Müssen-Gefühl auslösen.

Der \emph{nerdy}-Look Miucci Pradas, selbst immer im
Schullehrerinnen-Look auftretend, könnte für die Amerikanerin Rachel
Comey durchaus lehrreich gewesen sein. Die Mode der ehemaligen
Galeristin Comey kleidet nicht jede, spricht vor allem die
Intellektuellen New Yorks an, Künstler, Musiker, Galerienbesitzer.
Shorts mit hohem Bund und aufgesprenkelten Blüten kombiniert mit
niedlichen Blusen und Stricktops, geschnitten wie klassische Wifebeater,
dazu knielange, schwingende Röcke sowie kurze Jäckchen -- alles stets
komplettiert durch Brillen mit schweren Gestellen. Und auch für den
kommenden Winter bleibt Comey ihrem lässig, charmanten Vintage-Stil treu
mit Blusen unter Drei-Viertel-Arm-Pullovern. Und kleine weiße Söckchen,
handbedruckt mit Buchtiteln beispielsweise Séance on a Wet Afternoon von
Marc McShane, stecken in schwarzen Herrenhalbschuhen.

Das alles erinnert ein wenig an die Mode des Amerikaners Marc Jacobs,
dessen prätentiöser Modestil allerdings eher im Grunge angesiedelt sein
dürfte. Nur Jacobs selbst inszenierte sich jahrelang als der pummelige,
schüchterne \emph{Nerd} mit Pferdeschwanz und riesiger
Krankenkassenbrille, bis er das ganze einem strengen Fitness- und
Ernährungsplan unterwarf, und heute jeden neuen Muskel in einem anderen
Hochglanzmagazin feiert.

Wie Intellekt und eine Brise Irrsinn in der Mode zusammen gehen, zeigt
das niederländische Designer-Duo Viktor \& Rolf. Der stets gemeinsame
Auftritt mit dicker Büffelhornbrille ist zu ihrem persönlichen
Markenzeichen geworden, und absolutes Styleelement jeder
Herrenkollektion. Was Viktor \& Rolf die Hornbrille ist Karl Lagerfeld
seine Sonnenbrille. Lagerfeld ist nicht nur Modeschöpfer und Fotograf,
sondern generell ein Phänomen in Sachen Büchern und deshalb hier
erwähnt. Abgesehen davon, dass er selbst Bücher herausgibt, besitzt
Lagerfeld an die 300.000 Bücher und beschäftigt sogar eine eigene
Bibliothekarin.

Kleidung ist ja bekanntlich auch immer ein Spiegel unserer Gesellschaft,
Architektur und zeitgenössische Kunst ebenso. Früher war die Aufgabe
einer Bibliothek hauptsächlich das Bereitstellen ihrer Bücher zum Lesen,
und dem Publikum das Lesen zu ermöglichen. Heute ist der Lesesaal der am
wenigsten frequentierte Saal. Es fungieren als Lesesäle die Cafeteria im
Untergeschoss, wohin man die Bücher mitnehmen darf, wo man an einem
Tischchen sitzend mit einem Kaffee und einem Croissant, auch mit einer
Zigarette, weiterarbeiten kann.

{[}Die Künstlerin Candida Höfer feierte mit ihren monumentalen
Fotografien von Bibliotheken gerade großen Erfolg, als Bibliotheksneu-
und umbauten zu Prestigeobjekten vieler Kommunen wurden. Es sind vor
allem die modernen Bücherpaläste, wie die neue \emph{Bibliothèque
nationale de France} des Architekten Dominique Perrault, die
Universitätsbibliothek Cottbus von Herzog \& de Meuron, die von Renzo
Piano renovierte und erweiterte JP Morgan Library in New York Freie oder
die von Zaha Hadid geplante Bibliothek von Sevilla, die mit ihren
teilweise biomorphen Stahl-, Glas- und Kunststoffhüllen dem verstaubten
Charme traditioneller Bibliotheken ein modernes, fortschrittliches Image
verpassen. Ob es dort wohl auch noch nach alten englischen Romanen,
russischem Leder und Holzpflegepolitur riecht? Falls nicht, kann man es
ein bisschen \emph{In the Library} in die Luft sprühen.

%autor

\end{document}