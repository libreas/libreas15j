\documentclass[output=paper]{langsci/langscibook} 
\title{Das Warum} 
\author{Michaela Voigt  \affiliation{Technische Universität Berlin, Universitätsbibliothek}}  

\abstract{Thomas Hartmann, \enquote{Mantra Rechtssicherheit}. LIBREAS. Library Ideas, 22 (2013). \url{https://libreas.eu/ausgabe22/01hartmann.htm}}

\begin{document}
\maketitle

\noindent Rechtssicherheit, klar, die hätte jede\*r gern. Bibliothekar\*innen auf jeden Fall! Warum sie mitunter ein ersehnter Mythos bleibt, bleiben muss oder auch sollte, legt Thomas Hartmann in seinem Beitrag von 2013 ausführlich dar. Mich bewegt dieser Text stark, da er den Kern meines eigenen bibliothekarischen Handlungsfeldes (Open Access) trifft. Natürlich wünschte auch ich, dass es eindeutige Regelungen für -- nicht ausschließlich aber doch insbesondere -- Bibliotheken und die hier angesiedelten Services gibt. Doch hat vor allem Thomas Hartmanns Text mir dabei geholfen zu akzeptieren, dass es ist, wie es ist -- insbesondere in Bezug auf das gesetzlich verankerte Zweitveröffentlichungsrecht. In Gesprächen mit zweifelnden oder gar verzweifelten Kolleg\*innen zitiere ich diesen Text bis heute gern. Und ich möchte vor allem für das Fazit des Textes werben: Wir sollten mutig sein, rechtliche Unwägbarkeiten gestalten, es womöglich auch einmal auf eine Klage ankommen lassen, die mehr Klarheit für alle schaffen könnte -- aber womöglich nie kommt.

\end{document}
