\addchap{LIBREAS -- Wie es begann}  

\title{LIBREAS -- Wie es begann} 
\author{Ben Kaden and Maxi Kindling}

%body
Das Jahr 2005 war das Einsteinjahr. Und das Jahr der Rosskastanien, des
Uhus und des \enquote{Wir sind Papst!}. Es war das Antrittsjahr der
Bundeskanzlerin. Harold Pinter bekommt den Literaturnobelpreis und
Theodor Hänsch den für Physik. Wir studierten noch. Die Rettung des
Instituts für Bibliothekswissenschaft der Humboldt-Universität zu Berlin
wirkte noch nach. Im März wurde auf dem Campus der
Heinrich-Heine-Universität über die Frage diskutiert, wie das deutsche
Bibliothekswesen den Herausforderungen der Zukunft begegnen kann, ohne
seine Dienstleistungen angesichts sinkender Etats einschränken zu
müssen.\footnote{Bu: Bibliothekare zufrieden. In: Rheinische Post,
  19.03.2005.}

\enquote{Unser Konzept ist aufgegangen}, so die Direktorin der
Universitäts- und Landesbibliothek {[}Irmgart Siebert{]}, \enquote{es
herrscht eine deutlich spürbare Aufbruchstimmung nach dem Motto: Wir
wollen nicht klagen, sondern auf unsere Leistungen verweisen und nach
vorne blicken} las man in der Rheinischen Post.\footnote{ebd.} Wir
hatten T-Shirts an, auf denen stand \enquote{LIBREAS. Library Ideas}.
Wir hatten Visitenkarten. Wir hatten gerade unter www.libreas.de eine
Webseite online gestellt mit einer ersten Ausgabe.\footnote{\url{https://libreas.eu/ausgabe1/inhalt.htm}.}
\enquote{Bibliotheksbau} war das Thema. In Berlin bekamen die drei
Universitäten neue Bibliotheken. Kurz zuvor waren wir mit Professor
Walther Umstätter auf einer Tiefschnee-Exkursion in Kopenhagen und
hatten uns vom ebenfalls neuen \enquote{Schwarzen Diamanten}
(\enquote{Den Sorte Diamant}) der Königlichen Bibliothek einschüchtern
lassen.

Auf dem Düsseldorfer Bibliothekartag galten wir als Nachwuchs. Petra
Hauke hatte uns als Studierende ihres Projektseminars \enquote{Von der
Idee zum Buch} zu einem gemeinsamen Vortrag ermuntert, für den wir
jeweils ein oder zwei Folien zur Zukunft der Berliner
Bibliothekswissenschaft vorbereiteten: Das \enquote{Imageproblem der
deutschen Bibliothekswissenschaft} war und ist bis heute ein Thema. Wir,
die wir das Fach immerhin an der Humboldt-Universität studierten, hatten
von einem uns ansonsten unbekannten Fachreferenten namens Thomas
Hilberer die Aussage auf den Tisch bekommen: \enquote{Eine
Bibliothekswissenschaft -- dies nur am Rande gesagt -- gibt es nicht; es
kann sie so wenig geben wie eine Wissenschaft etwa von der Verwaltung
eines Einwohnermeldeamtes oder vom Betrieb eines Großkaufhauses. Die
bibliothekarische Tätigkeit ist eine Technik, ein \enquote*{Handwerk},
eine Praxis.}\footnote{Thomas Hilberer: Leserbrief zu: Jörn Klockow /
  Volker Roth-Plettenberg: Strukturmodelle für eine Ausbildung zum
  Höheren Bibliotheksdienst. In: Bibliotheksdienst, Heft. 3, 1991 S.
  334--345.}

Dieser, wie man heute sagen würde, \enquote{Hot Take} stammte zwar nur
aus einem Leserbrief im Bibliotheksdienst und aus einem anderen
Jahrhundert. Er trieb das Fach aber auch Anfang der 2000er Jahre aus
einem für uns nicht verständlichen Grund vor sich her. Wir ließen uns
ein bisschen mittreiben, denn es ging in diesen Jahren nicht um
irgendein Institut, sondern um unseres, das aufgrund des ewig klammen
Berliner Haushalts geschlossen werden sollte. Die berüchtigte
Hilberer-These spielte bei den betriebswirtschaftlichen Überlegungen gar
keine Rolle. Außer vielleicht in der Hinsicht, dass das Institut für
Bibliothekswissenschaft aus verschiedenen Gründen nicht allzu viel
zeigte, was eine selbstbewusste Stellung als Wissenschaft rechtfertigte.
Die hätte es durchaus geben können, wenn man ins Studienprogramm dieser
Jahre schaut. Wissenschaftsforschung und Bibliometrie waren für viele
Studierende Schwerpunkte, denen hausintern viel Bedeutung zukam. Viele
der Studierenden träumten jedoch tatsächlich von einer Anstellung als
Bibliothekarin und hatten sich davon zu einem Studium verleiten lassen,
das ihnen beispielsweise in Gestalt eines Pflichtseminars zu
\enquote{Mathematischen Grundlagen} ein Überdenken dieser Entscheidung
nahelegte. Dabei hatte es bereits Inge Lindtner, verwaltungstechnisch am
Institut zuständig für Studienangelegenheiten und obendrein Expertin für
das Thema Inhaltserschließung, in jeder Einführungswoche eindeutig zu
vermitteln versucht: \enquote{Wenn Sie Bibliothekar werden wollen, sind
Sie hier falsch.} Wir hörten nicht darauf. Wir wurden aber auch nicht
alle Bibliothekarinnen.

In Düsseldorf standen wir nun ziemlich euphorisch vor den
Bibliothekarinnen und ihren Ausbilderinnen, denn auf Folie 3 und in
unseren Herzen brachten wir die frohe Botschaft aus dem Februar mit: Das
Berliner Institut wird nicht geschlossen, sondern bekommt die Chance und
Auflage einer Neuerfindung seiner selbst. Da polterten wir natürlich
begeistert mit hinein und schossen in der Präsentation eine Breitseite
von, nun ja, Argumenten gegen die Hilberer-These und für eine
Bibliothekswissenschaft: \enquote{Warum kastrieren sich {[}die{]}
wissenschaftlichen Vertreter {[}des Bibliothekswesens{]} gleichsam
selbst, indem sie sich ausschließlich als Manager oder Verwalter
begreifen?} Das war vielleicht nicht die strategisch klügste Position,
wenn man demnächst eventuell diesen Vertreter*innen in
Bewerbungsgesprächen gegenüber sitzen würde. Aber dahinter stand eine
ehrliche Verwunderung. Warum so zaghaft? Warum so fantasiearm? Warum so
folgsam?

Glücklicherweise hatten wir nicht nur unsere Verwunderung nach
Düsseldorf mitgebracht, sondern eine der ersten Neuerscheinungen in
einer langjährigen Reihe von Buchprojekten von Petra Hauke. Die war bei
De Gruyter herausgekommen und verhandelte die Frage
\enquote{Bibliothekswissenschaft -- quo vadis?} Wohl aufgeschreckt durch
den nahen Einschlag der Beinaheschließung nicht nur des Instituts,
sondern im Prinzip auch der Disziplin in ihrer akademischen Variante,
fanden sich eine ganze Reihe der prominenten und weniger zurückhaltenden
Vertreterinnen und Vertreter aus dem Umfeld der Bibliothekswissenschaft
bereit, eine Lanze für selbige zu brechen. Dass der Band in einem
Seminar entstand und seine Produktion daher über weite Strecken von
Studierenden organisiert wurde, legte einen zusätzlichen Stein für das
Institut ins Brett, denn bei so viel Aktivität war es vielleicht doch
noch nicht derart obsolet, wie es lange schien. Aus Sicht der
Öffentlichkeitsarbeit war dieses \enquote{Von-der-Idee-zum-Buch}-Projekt
des Jahres 2005 ein äußerst geschickter Schachzug.

Der Eckstein war also gesetzt. Wir wussten nun irgendwie, wie es so geht
mit einer Herausgeberschaft. Das Institut war im Aufbruch. Und so wie es
seiner ungewissen, aber nun vorerst gesicherten Zukunft gegenüberstand,
ließ uns die Entwicklung Spielräume um unter anderem das Projektseminar
einfach weiterzutreiben. Rainer Kuhlen hatte als Informationsethiker und
Gastprofessor nicht nur seinen Hund Baltimore und für Außenstehende
erstaunliche Reibungslinien zum Informationsbegriff mit seinem Kollegen
und damaligen Institutsdirektor Walter Umstätter in die Dorotheenstraße
gebracht. Sondern auch die Frage: \enquote{Wem gehört Wissen im 21.
Jahrhundert?} und das sehr kurzlebige und zeitgeistige Schlagwort der
\enquote{Napsterisierung}.\footnote{Rainer Kuhlen: Napsterisierung von
  Wissen -- eine Herausforderung an Ethik, Ökonomie, Recht und Politik
  für den Umgang mit Wissen und Information in elektronischen Räumen.
  Vortrag auf dem 3. Berliner Forum Electronic Business. Berlin,
  04.07.2002} Peter Schirmbacher, sehr umtriebiger und
gestaltungswilliger Leiter des Rechenzentrums der Humboldt-Universität
wurde etwa zeitgleich mit der Aussicht und später Einlösung einer
Sonderprofessur für Informationsmanagement ans Institut gelockt und
brachte \enquote{Open Access} als sich global etablierendes Prinzip in
Theorie und vor allem auch in der Praxis mit. Die gemeinsame
Arbeitsgruppe Elektronisches Publizieren von Rechenzentrum und
Universitätsbibliothek betrieb einen der ersten Dokumentenserver an
einer deutschen Universität damals schon seit fast zehn Jahren. Dank des
forschen Vorantreibens im Haus verabschiedete die HU Berlin schließlich
ihre Open-Access-Erklärung. Für das, was LIBREAS, also die erste
Open-Access-Zeitschrift im deutschen Bibliothekswesen und der
deutschsprachigen Bibliothekswissenschaft werden sollte, hatten wir
damit den zweiten entscheidende Baustein. Die Idee des \enquote{Open
Access} war uns zuvor und ohne weitere Kenntnisse der Berliner
Open-Access-Erklärung bereits grundsympathisch. Und wir hatten
eigentlich auch keine andere Wahl als elektronisch und offen zu
publizieren.

Obwohl da auch ein Verlag war. Die Saur-Bibliothek war aufgrund eines
glücklichen Umstands Arbeits- und manchmal auch Lebensort eines
studentischen Mitarbeiters von Professor Umstätter. Aus Platzgründen
diente eine Ecke der Bibliothek als Büro und dadurch lag ein Schlüssel
zu Tür, Haus und Tor in den Händen eines kommenden LIBREAS-Herausgebers.
Mit diesem kam die Aufgabe des Schließdienstes für die Bibliothek, die
nun häufig jenseits offizieller Öffnungszeiten Diskussionsraum und
Kinozimmer wurde. Seminarraum war sie sowieso. Einmal die Woche gab es
eine Veranstaltung zum Internationalen Bibliothekswesen, die Elisabeth
Simon betreute, eine verdiente Bibliotheksstrategin aus dem frisch
versunkenen Deutschen Bibliotheksinstitut und seit neuestem
Mitbetreiberin eines Verlags namens Bibspider. Herausstechend war an
ihr, was wohl auch an uns herausstach: eine unmittelbare Entflammbarkeit
für gern auch wilde Ideen zum Bibliothekswesen. LIBREAS war entsprechend
ein willkommener Funke und auch ihre damalige und etwas zögerlichere
Verlagspartnerin Walburga Lösch ließ sich letztlich doch schnell
begeistern. Sie unterstützten uns mit Domain, T-Shirts und einem
digitalen Diktiergerät für eventuelle Interviews. Obendrein
organisierten sie die ISSN und warben einen Teil der ersten Autorinnen
und Autoren an. Auch thematisch waren wir für die ersten Ausgaben ein
Redaktionsteam. Dass das anfängliche gemeinsame Feuer der
Publikationsleidenschaft bei Bibspider bald nur noch ein leichtes
Schwelen war und schließlich verglomm, lag an äußeren Umständen, auf die
wir keinen Einfluss hatten. Es sorgte aber dafür, dass die geplante
Reihe der LIBREAS-Bücher mit dem Sammelband zur Sozialen
Bibliotheksarbeit\footnote{Ben Kaden, Maxi Kindling: Zugang für alle --
  soziale Bibliotheksarbeit in Deutschland. Berlin: bibspider, 2007.}
zwar vielversprechend anbrandete, aber mit diesem einen Band auch gleich
wieder verebbte. Dennoch: Bibspider war der dritte Impuls und ebenfalls
ein entscheidender Grundstein von LIBREAS.

Das ist nun alles unglaubliche fünfzehn Jahre her. Denken wir jetzt
zurück, denken wir sofort: Wie jung wir einmal waren. Und auch: Wie zäh
wir waren! LIBREAS war ähnlich wie die Bibliothekswissenschaft nicht
unbedingt gewollt, von manchen auch früh und von manchen auch später
noch abgeschrieben. Die Mailingliste INETBIB nahm Erfahrungen vorweg,
die heute -- leider -- so prägend für Twitter-Diskussionen sind. Die
Strategie, die wir hatten, nämlich nicht so viel darüber nachzudenken,
hat sich letztlich bewährt. Wir haben die Welt sicher nicht erschüttert.
Aber wir haben mit LIBREAS einiges in die Welt gebracht. Und wir sind an
LIBREAS gewachsen. Wir wiederholen es gern und immer wieder: Wir wollten
mit LIBREAS etwas schaffen, was wir gern gelesen hätten, was es aber
nicht gab und was wir daher selbst angehen mussten. Das ist uns in
diversen Verästelungen und Varianten gelungen. Und es gelingt uns auch
weiterhin. Im Stellenwert des individuellen Lebens ist LIBREAS
mittlerweile vielleicht gesunde zwei oder drei Positionen in den
Hintergrund gerückt. Wir sind gelassener und ruhiger. Aber das
Grundgefühl ist nach wie vor da. Bei jeder Ausgabe. Ein bisschen
bedauern wir ja, eine elektronische Zeitschrift zu sein. Denn dieses
Format manifestiert sich nicht im Regal. Man reiht nicht Heft an Heft.
Man sieht nicht so leicht, was sich in fünfzehn Jahren so ansammelt. In
der digitalen Liste mit den Beiträgen, immerhin, kann man es
nachvollziehen.\footnote{\url{https://libreas.eu/autorinnen/}.} Und da
sehen wir, dass es schon eine ganze Menge ist. Aus dieser Menge wiederum
haben wir uns als Redaktion erlaubt, jeweils einen Beitrag
herauszunehmen und dann doch mal drucken zu lassen. Die Zusammenstellung
findet sich in diesem Band. Es ist kein Best-Of, sondern ein
\enquote{Most loved} und \enquote{Moved most}. Das ist ein Unterschied.
Und genau um den geht es. LIBREAS nämlich war, ist und bleibt für alle,
die dabei als Herausgeberinnen und Redakteurinnen mitwirkten und
mitwirken, eine Herzensangelegenheit. Karsten Schuldt hat dies
wunderbarerweise in einer sprachlichen Kleinigkeit, zugleich
Erheblichkeit fixiert, in dem er LIBREAS einen bestimmten Artikel gab.
Die LIBREAS. Wir sind sehr froh, dass es sie gibt.

\begin{flushright}
Ben Kaden and Maxi Kindling\\
Berlin, Dezember 2020
\end{flushright}


%autor

