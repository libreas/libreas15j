\documentclass[output=paper]{langscibook}

\title{Cereal Boxes and Milk Crates Zine Libraries and Infoshops are… Now
}
\author{Lacey Prpic Hedtke}

\abstract{


URN: \href{http://nbn-resolving.de/urn:nbn:de:kobv:11-10091336
}{urn:nbn:de:kobv:11-10091336}

DOI: \url{http://doi.org/10.18452/8891}

Keywords: Bibliothekswissenschaft, Methodologie, Bibliothekstypologie, Dekonstruktion, Bibliotheksraum, Wissenschaftstheorie
}

\begin{document}
\maketitle

%body
\begin{quote}
\enquote{{[}\ldots{]} It's hard to find library materials that challenge
the for-profit, corporate culture. Our well-stocked county, community
college, and university libraries, though publicly funded, primarily
serve private middle-class constituencies -- businesses, professions,
students, job-seekers, and consumers. There's not much there for those
who don't share the American Dream.}\footnote{Atton, C. (1999) The
  infoshop: The alternative information centre of the 1990s. \emph{New
  Library World}, 100 (1146) 24.}
\end{quote}

In some circles a radical statement, in others, a motivation for action.
Zine libraries, infoshops, bookmobiles, street libraries, and zine
vending machines are all very different spaces and ideas, but all with
the same purpose to provide access to information outside the corporate
mainstream. I'm especially interested in how people who don't fit neatly
into categories create venues for their own access to information,
specifically through the establishment of zine libraries and infoshops.

Here's the breakdown: A zine library is a collection of zines (handmade
books like the one you're reading) organized by zine-lovers and makers
(often referred to as zinesters) in the hopes of preserving and making
accessible obscure materials. Since most zines typically have runs of
1-500, each and every one is rare.

An infoshop is what its name implies: a place to go for information.
Infoshops are usually, but not always, run by anarchists, but not
necessarily for anarchists. They are volunteer-run non-hierarchical
spaces where people can go to for lectures, meetings, events, concerts,
and activist resources. Some infoshops house libraries and reading
rooms. Many have cafés or at least a cup of tea available. Infoshops
sprung up in Europe and have caught on in the US in the past fifteen
years. Infoshops are ideal for activist travelers, functioning as a
place to stop in to find out where the coop is, where to crash, and to
find people with similar interests. They usually have free internet
access too.

The term \enquote{street library} hasn't caught on as much as
\enquote{infoshop} has, but it is basically the same idea with a twist.
A street librarian is a one who doesn't leave the library when they
leave the physical structure. Many librarians who are also activists
will offer their resource skills at protests and events, and are able to
bring their abilities wherever they go. People can go to a street
librarian to gain information on \enquote{underground} activities and
events. Although they're hard to pin down, when they're around you'll be
sooo happy.

There are millions of inventive and creative ways of getting resources
and materials to the interested. Zine bookmobiles and vending machines
are some venues I've stumbled across. Tool lending libraries and zine
recycling programs are another. It's surprising how many new ways there
are of exposing people to self-published media. All of these resource
centers have something in common: The aim of fulfilling the need for
access to materials and information otherwise difficult to obtain. You
won't find People magazine proudly showcased in these venues. I'm mainly
focusing on infoshops and zine libraries, as they are the most enduring
and organized forms of information centers, and have more evolved
methods of collection maintenance. It's interesting to see what happens
when people who aren't librarians by profession, or even by education,
get together to form a library or resource center.

\hypertarget{why-do-these-people-need-their-own-library-cant-they-just-go-to-the-public-or-academic-library}{%
\section*{Why do these people need their own library? Can't they just go
to the public or academic
library?}\label{why-do-these-people-need-their-own-library-cant-they-just-go-to-the-public-or-academic-library}}

No.~There's a reason why so many alternative libraries exist. There is a
clear gap in the information world. As stated above, most libraries keep
to the middle of the road. It is very hard to find any materials
published outside the mainstream, and especially hard to find materials
that have been self-published (zines and factsheets), or non-mainstream
periodicals, newspapers and tabloids. Also, infoshops and zine libraries
tend to have later hours, be connected with galleries, music show
spaces, and other resource venues such as darkrooms or screenprinting
shops, as is the case at ABC No Rio in New York City. It also seems
these types of libraries draw the paranoid, and rightfully so. Since the
PATRIOT Act was enacted, library records are no longer private
information. Activists and anarchists, and even sometimes artists are
watched by the FBI. Rather than give the government fodder to harass
them, through questionable library research, the use of a zine library
for information results in trackless searching. \enquote{In many groups
{[}\ldots{]} there is an emphasis on self-education {[}\ldots{]} Groups
often establish their own small}libraries" of relevant books,
periodicals and papers, sometimes in collaboration with a local
alternative bookshop or information centre. The rise of the
\enquote{infoshop} in recent years throughout Europe and the US is one
manifestation of such local activity. Usually based around a local
anarchist group, although it is of benefit to more than anarchists, it
acts as a communication and distribution point for any number of local,
national and international groups, movements and projects {[}\ldots{]}
The infoshop emphasizes empowerment, providing information freely (or
very cheaply) to enable people to work together, directly on issues that
affect their lives."\footnote{Ibid.}

\hypertarget{what-kind-of-spaces-are-they-housed-in-how-could-you-possibly-have-a-library-in-a-house-or-old-gas-station}{%
\section*{What kind of spaces are they housed in? How could you possibly
have a library in a house or old gas
station?}\label{what-kind-of-spaces-are-they-housed-in-how-could-you-possibly-have-a-library-in-a-house-or-old-gas-station}}

The people organizing these types of places aren't going for glamorous.
In most cases, they're going for whatever they can get. This is why
you'll find these libraries in people's living rooms, trucks, basements,
in tenements, galleries and student centers. Unless a non-profit
organization is backing the infoshop or library, be prepared for
creative solutions to space problems. Cheap rent in a bad neighborhood
usually equals a great place for an infoshop or zine library.
\enquote{Besides financial problems, neighborhood communication
difficulties are common {[}...{]} many infoshops are organized by white
youth in communities populated by minorities. The subculture that
patronizes the shop {[}\ldots{]} sticks out in contrast to the
surrounding neighborhood. Residents may perceive the infoshop as a
beach-head in the gentrification happening in that town.}\footnote{Dodge,
  C. (1998) Taking libraries to the street: Infoshops \& alternative
  reading rooms. \emph{American Libraries}, 29 (5) 62.}

The Mobilivre out of Canada travels across the continent in a stylish
Airstream trailer, bringing zines and workshops with them. The Anchor
Archive Regional Zine Project in Nova Scotia offers an
artist-in-residence program where artists can stay in a storage shed in
the backyard for a few weeks to make a zine. A few libraries accompany
\enquote{Food Not Bombs}, an organization in several cities handing out
free food at different parks or public areas weekly. In Japan, there are
bunkos. \enquote{Typically run by groups of women for their neighborhood
children} {[}\ldots{]} bunko is a network of \enquote{tiny outposts
which may be found in homes, converted train cars, community centers, or
even log cabins.}\footnote{Ibid.}

\hypertarget{what-are-in-these-libraries-how-are-they-run-how-do-they-acquire-materials-if-theyre-anarchists-i-heard-anarchists-dont-like-money.}{%
\section*{What are in these libraries? How are they run? How do they
acquire materials if they're anarchists? I heard anarchists don't like
money.}\label{what-are-in-these-libraries-how-are-they-run-how-do-they-acquire-materials-if-theyre-anarchists-i-heard-anarchists-dont-like-money.}}

It's true. They don't. But some infoshops sell things. Patches,
t-shirts, books, videos, art, etc. Most places operate collectively,
which often involves paying out of collective member's pockets, and most
frequently relying on donations of materials from people with a lot of
zines lying around, other zine libraries with duplicates, or donations
of cash. Zine-makers tend to understand a zine library's motives, and
since they aren't making zines to make money anyway, feel great donating
their creation to the library. It provides another venue for a reader to
stumble across their hand-bound lovingly screenprinted handmade book.

A way most libraries pay rent is through benefit concerts or sometimes
art auctions, or anything else that might be fun and also raise money.
In the case of the Papercut Zine Library, the group of librarians was
able to trade building labor for a free room.\footnote{Stockton, P.
  (2005, June 12) Ah, a new literary oasis, and she the zine queen.
  \emph{Boston Globe}.
  \url{http://archive.boston.com/news/local/articles/2005/06/12/ah_a_new_literary_oasis_and_she_the_zine_queen/}}
Rarely are materials bought outright. Sometimes library dumpsters are
raided for discarded books, and also for organizational materials
(bookshelves, magazine racks, etc.) Dumpstering is a fantastic way to
get a lot of what you need for free. But that's another topic. However
these libraries obtain their materials, almost all ask the subject
matter not be racist, sexist, or homophobic. Collective action entails
each member committing to the project, coming to meetings and voting on
each major decision, and each being equally responsible to maintain the
space, and everyone is also able to plan events or enact ideas within
the space. No one person is in charge of a collective. A collective is a
cooperative effort, which if done with a certain amount of enthusiasm
and respect for each other, can work out fantastically. If those basic
values aren't in place, there will be burn out, and the space could
fail. Luckily, people working on fun projects like zine libraries just
want to see the library succeed and grow, and they do!

\hypertarget{how-can-you-possibly-organize-information-that-hasnt-already-been-cataloged-by-another-person}{%
\section*{How can you possibly organize information that hasn't already
been cataloged by another
person?}\label{how-can-you-possibly-organize-information-that-hasnt-already-been-cataloged-by-another-person}}

Easy. You make it up. This is where Sanford Berman would argue the
access part comes in. How easy is it to find this information? In some
instances, there is no organization. Your findings are left up to fate,
chance, and synchronicity. Which is great if your psychic skills are
honed. Several zine libraries stick to the alphabetical system, but most
zine libraries catalog by topic. If you go in to the library searching
for a good book on bicycle maintenance, you'll also find a zine on good
routes to ride without getting hit, how to weld your own tall bike, and
riding safely, if you're searching the bicycling section. Since zine and
infoshop library collections tend to be radical in nature, their subject
headings are unique. DIY (do-it-yourself) is a HUGE category, with
several subsets to the category. A few others you won't find in the
public library are- radical menstruation, squatting, dumpstering,
protesting, XXX, sustainable living, fat, and grrrls. The people
cataloging this material respond to their material through topics and
organizational methods appropriate for their subject matter. It's
important to point out that although there are librarians by profession
involved in infoshops and zine libraries, most zine librarians are
either still in library school, or have never had any experience working
in a library at all. They just want to give people access to information
they might not even know was out there.

Take for instance the Papercut Zine Library, in Cambridge,
Massachusetts.\footnote{Ibid.} Even though this library is housed in the
Harvard Social Hall, in the same neighborhood as some of the world's
biggest and best libraries, this library is thriving. There is obviously
a need within the community for zines and all the information they
contain and offer. With over 2,000 zines organized by topic on small
shelves, and an online searchable catalog, the volunteer zine librarians
running the place have figured out how to catalog and organize their
material without burying it underneath unsearchable databases or vague
subject headings. None of these libraries use cataloging terminology or
systems. None of them have scanable barcodes, use the LC or Dewey
Decimal systems. I'm sure they have never once consulted the Library of
Congress subject headings to make sure they're using the appropriate
heading for the zine on home dentistry. Most record which zine or book
belongs in what section, maintain a list of what they have and what gets
checked out, if theirs in a circulating collection, and forget the rest.
The extents to which the digital cataloging systems go are Filemaker or
LibraryThing, making the catalog available online. It's important to
realize that even though these zine librarians aren't trained in
cataloging, they've merged systems that have already been invented with
their own original systems.

Although each system is different-some might throw zines into a box and
let you sort through, some cut the tops off cereal boxes for
organizational systems, and some have book racks, displaying items more
like a store, all have invented innovative ways of cataloging and
finding the material.

\hypertarget{who-uses-zine-libraries-how-do-they-find-out-about-them}{%
\section*{Who uses zine libraries? How do they find out about
them?}\label{who-uses-zine-libraries-how-do-they-find-out-about-them}}

Anyone who wants to access information not available at their public
library uses zine libraries or infoshops. Anyone interested in zines,
underground publishing, little magazines, one-offs, tabloids, art,
quirks, or free speech in general are excited by zine libraries.
Researchers, students, zinesters, artists, old hippies and beatniks,
those on the political left and outfield use them. Zine libraries and
infoshops don't advertise in newspapers or magazines because they're
poor. They're found through word of mouth, posters put up in co-ops,
bike shops and on telephone poles. There are a few websites about zines
that mention library locations. They are often moving and sadly closing.
But new ones are always opening, in different forms. The Zine Machine,
for instance, is a vending machine with zines inside. For a dollar or
two you can have your own zine to take home. Some are in university
libraries, and they are much more organized and professional-looking and
operating, which is why I didn't choose to focus on them here. Some
libraries are connected to other ventures, and if you look, or go to any
zine-related event (store, reading, zine fair), you'll be sure to find a
trail to the zine library or infoshop.

\hypertarget{how-do-the-libraries-attract-users-who-are-these-libraries-geared-toward-do-i-have-to-pay-to-get-in}{%
\section*{How do the libraries attract users? Who are these libraries
geared toward? Do I have to pay to get
in?}\label{how-do-the-libraries-attract-users-who-are-these-libraries-geared-toward-do-i-have-to-pay-to-get-in}}

Other than word of mouth, libraries will often bring cross-sections of
their collection for on-site checkout to zine fairs and events,
anarchist book fairs, or to zine and book readings in the punk
community. The libraries aren't necessarily geared toward anarchists or
punks, but due to the radical materials and DIY ethics of zines, these
groups are a large user base. These libraries are frequented by anyone
interested in the subjects they cover, and most importantly, it is
almost always free to check out a book or zine, if they don't think
their collection is too valuable or rare to let off-site. It is because
of this idea of libraries for all that such a wide variety of people are
drawn to the zine library or infoshop.

\hypertarget{are-zine-libraries-really-libraries-i-dont-know-about-this}{%
\section*{Are zine libraries really libraries? I don't know about
this\ldots{}}\label{are-zine-libraries-really-libraries-i-dont-know-about-this}}

Zine librarians take the stance that if anyone says it's a library, it
is. If Duchamp can say found objects are art, zine librarians spending
hours cataloging and organizing ephemera and oddities can call
themselves librarians and their creations libraries. In this sense,
anyplace that provides access to information in a somewhat organized or
searchable form, can be considered a library. The word library seems so
authoritative and smarty-pants. Zine librarians are taking the word and
applying its meaning to a wide range of information resources, including
a roomful or bagful of books or zines.

I do hope that zine libraries and infoshops grow in popularity and use.
I hope that the collectives running them find reliable methods of
funding so fewer are closing. And I do hope that public libraries will
become hip to the idea of zines and alternative/non-mainstream
periodicals and other materials. Some are starting to realize what a
valuable resource they are in terms of documenting cities, contemporary
culture and events otherwise not covered by the media. However, there
will always be a need and space for infoshops and zine libraries. No
matter how much information makes it into public and academic libraries,
unless these libraries are suddenly taken over by zine librarians, the
board of directors booted, and the institutions are run collectively,
zine libraries and infoshops will be filling the information gap in
storefronts, garages and shacks.

\begin{center}\rule{0.5\linewidth}{0.5pt}\end{center}

%autor
Lacey Prpic Hedtke studiert Library and Information Science am St. Catherine College in St. Paul, Minnesota, USA. Sie interessiert sich vor allem für ehrenamtlich geführte Bibliotheken, Veröffentlichungen von Zines sowie für den gemeinschaftlichen Einsatz von Infoshops und Resource Centers.

\end{document}
