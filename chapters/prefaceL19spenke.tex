\documentclass[output=paper]{langsci/langscibook} 
\title{Das Warum} 
\author{Jana Rumler}  

\abstract{Julia Spenke, \enquote{Bibliothekarische Berufsethik: Mit welchen Themen befassen sich bibliothekarische Ethikkodizes?}. LIBREAS. Library Ideas, 19 (2011). \url{https://libreas.eu/ausgabe19/texte/01spenke.htm}}

\begin{document}
\maketitle

\noindent Der beinahe zehn Jahre alte Text von Julia Spenke, der Untersuchungen und Ergebnisse ihrer Diplomarbeit übersichtlich und gut lesbar zusammenfasst, hatte an Aktualität der Thematik zugenommen, als sich die LIBREAS-Redaktion ab 2017 mit der Neutralitätsdebatte im Bibliothekswesen beschäftigte. Als Antwort auf den fragenden Titel und im Vergleich weltweit existierender Ethikkodizes, benennt Spenke damals die Sicherung der Informationsfreiheit, den Schutz des geistigen Eigentums, die Zensur und den Datenschutz als übergeordnete Themen. Um dann kritisch das deutsche Konzept von 2007, seine Entstehung, die Länge und inbesondere die fehlende Kommunikation in den Praxisbetrieb hinein zu beleuchten. Zwischenzeitlich hatte der BID 2017 eine Neufassung veröffentlicht, die wir uns aus Anlass genauer angesehen hatten, als es 2018 bei einer Podiumsdiskussion auf dem Bibliothekartag einen Meinungsaustausch gab, der uns wie aus der Zeit gefallen schien, nachdem insbesondere die öffentlichen Bibliotheken einen großen Anteil an der Willkommenskultur für Geflüchtete seit 2015 gehabt hatten. Wir haben unsere Eindrücke später im Editorial der Ausgabe 35 verarbeitet. Seither begegnen uns bei LIBREAS immer wieder Themen der verwandten Medien- und Informationsethik, beispielsweise das Tracking durch die Großverlage oder das Dekolonisieren von Beständen, die als Desiderat in der Berufpraxis zu sehen sind und sowohl Aspekte von Forschungsintegrität und Diversitätsorientierung aufnehmen, aber auch wieder vermehrt das Politische in bibliothekarischen, archivarischen und musealen Handlungsmaximen des Berufsstandes aufzeigen.   

\end{document}