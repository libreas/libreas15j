\documentclass[output=paper]{langsci/langscibook} 
\title{Das Warum} 
\author{ Maxi Kindling \lastand Karsten Schuldt}  

\abstract{Redaktion LIBREAS, \enquote{Scheitern in der Schreibwerkstatt: Aus der Redaktion der LIBREAS. Library Ideas}. LIBREAS. Library Ideas, 20 (2012). \url{https://libreas.eu/ausgabe20/texte/04redaktion01.htm}}

\begin{document}
\maketitle

\noindent Maxi Kindling

\noindent Scheitern war 2011, ist 2019 und wird auch weiter ein wichtiges Thema sein. Für Redaktion wie LIBREAS, aber auch für alle weiteren Projekte und Initiativen im Bibliotheksumfeld. Ich würde den Beitrag heute noch unterschreiben und wünsche mir nach wie vor, dass mehr und offener über Fehler, falsche Ansätze und Scheitern gesprochen wird.
\vspace{\baselineskip}

\noindent Karsten Schuldt

\noindent Im Laufe der Jahre haben wir als Redaktion gemeinsam viele Texte veröffentlicht: Fast alle Editorials, die Calls for Paper und noch einige weitere. Die Wahrheit ist allerdings, dass diese von Teilen der Redaktion geschrieben werden und der Rest ihnen dann zustimmt. Dieser Text ist anders: Alle (damaligen) Mitglieder der Redaktion arbeiteten an ihm gemeinsam, alle trugen etwas bei. Es geht ja auch um die Erfahrungen der Redaktionsarbeit selber. Insoweit finden sich hier auch Erfahrungen von uns allen wieder. (Eine Hoffnung wäre, dass sich solche Texte, die über das eigene Scheitern reden, öfter in bibliothekarischen Publikationen finden würden. Er ist auch inhaltlich weiterhin relevant, aber ich erinnere mich an ihn vor allem gerne, weil er ein wirklich gemeinsamer Text ist.)

\end{document}