\documentclass[output=paper]{langscibook}

\title{Notizen zur Bibliothekswissenschaft. Teil 1 und 2}
\author{Ben Kaden}

\abstract{
Zitiervorschlag: Ben Kaden, \enquote{Notizen zur Bibliothekswissenschaft. Teil 1 und 2}. LIBREAS. Library Ideas, 18 (2011). \url{https://libreas.eu/ausgabe18/texte/02kaden02.htm}

URN: \href{http://nbn-resolving.de/urn:nbn:de:kobv:11-100183850}{urn:nbn:de:kobv:11-100183850}

DOI: \url{https://doi.org/10.18452/8976}

Keywords: Bibliothekswissenschaft, Methodologie, Bibliothekstypologie, Dekonstruktion, Bibliotheksraum, Wissenschaftstheorie
}

\begin{document}
\maketitle

%abstracts

%body
\noindent Die nachfolgenden Notizen stellen den ersten und zweiten Teil einer
Serie mit grundsätzlichen Überlegungen zur Bibliothekswissenschaft dar.
Ziel dieser Reihe ist eine Präzisierung des Konzepts der
Bibliothekswissenschaft. Den Rahmen bildet die Frage, wie sich die
Institution Bibliothek in digitalen Kontexten, also als digitale
Bibliothek, definieren kann. Ich vertrete die Annahme -- und dies ist
zugleich die Hypothese für alle Überlegungen dieser Reihe -- dass es im
Zuge der Digitalisierung zu einer Art \enquote{semiotic turn} kommt,
also mit dem Verschwinden materieller Medienformen die semiotische
Dimension in den Mittelpunkt rückt. Die Bibliothekswissenschaft ist, so
meine Leitthese, in einem digitalen Umfeld nur als semiotische Disziplin
denkbar.

Im ersten Teil skizziere ich das Bewahren und Vermitteln von Narrativen
als Grundkonstante der Bibliothek und verankere die Bibliothek als
elementaren kulturellen Funktionsträger. Des Weiteren entwerfe ich in
Anlehnung an eine Idee der Soziologin Elena Esposito eine daraus
ableitbare mögliche Rolle der Bibliothekswissenschaft als einer Art
spezifische Narratologie. Im Mittelpunkt steht die Frage, aufgrund
welcher Kriterien Narrative in das Archiv Bibliothek ein- und
ausgeschlossen werden.

Im zweiten Teil überlege ich, wie sich die Rolle des Raumes der
Bibliothek, also der Bibliothek als Container, betrachten lässt und in
welcher besonderen Wechselbeziehung er zu den beiden anderen C-Faktoren
-- Content und Community -- steht. Meine These lautet hierbei, dass eine
bibliothekswissenschaftliche Betrachtung des realen und digitalen
Bibliotheksraumes immer auch mit der gegenseitigen Bedingtheit dieser
drei Gesichtspunkte auseinandersetzen muss.

Die Überlegungen sind als Problematisierung intendiert. Der Prozess der
Auseinandersetzung mit den genannten Thesen ist prinzipiell offen. Daher
ist Rückkopplung höchst willkommen.

~

\enquote{Und Schreiben heißt, auf eine bestimmte Weise die Welt (das
Buch) zerspalten und wieder zusammensetzen.} (Barthes, 1967, S. 88)

\hypertarget{teil-1-das-bewahren-und-seine-grenzen}{%
\section*{Teil 1: Das Bewahren und seine
Grenzen}\label{teil-1-das-bewahren-und-seine-grenzen}}

\subsection*{Das Manuskript, die Reise}

\begin{quote}
\enquote{{[}\ldots{]} sogar als ich das Riesenschiff nahm um in die USA
zu fahren war das gewiß nicht um in die USA zu fahren sondern um in die
Bibliothek zu fahren und also um nach Hause zu fahren in dies Heim ohne
festen Wohnsitz wo die Anhänger der Verbannung in ihrer
Unbestechlichkeit aufgenommen werden, die Idee der Verbannung wird dort
anerkannt ohne daß Kapital daraus geschlagen würde, weder berufsmäßige
Verbannung gibt es, noch Bereicherung, nur ein Dach aus Segeltuch über
der Lagerstätte.} (Cixous, 2010, S.77f.)
\end{quote}

\noindent Mitte der 1960er Jahre. Eine junge, wenn man so will und den Begriff
nicht scheut, Intellektuelle\footnote{Wobei Cixous mit dem Konzept des
  \emph{Intellektuellen} nicht ganz glücklich ist, wie sie u.a. in einem
  Interview aus den 1990er Jahren explizit äußert:
  \url{http://www.youtube.com/watch?v=ZKUQWv0irVw}, Zähler ca.
  01:00-02:00.} reist ganz klassisch und literaturnah mit einem
Passagierschiff über den Atlantik. Ein Walter Faber könnte über das Deck
schlendern, aber der ist in die andere Richtung und ein paar Jahre
früher unterwegs. Ein Karl Rossmann ist da näher, aber längst mit
verlorenem Koffer auf dem Weg nach Oklahoma verschollen. Das Ziel der
jungen Frau sind dann auch nicht etwa die Vereinigten Staaten als Land
der unbegrenzten Entfaltungschancen oder als Möglichkeit eines
persönlichen Exils. Obschon letzteres im Sinne einer Flucht als Motiv
lesbar ist, geht es ihr nicht um die USA. Sie fährt in die Bibliotheken
der USA. Erst dieser Raum, die Lesesäle in New York oder Yale, sind der
tatsächliche Fluchtpunkt ihrer Bewegung.\footnote{Und zugleich ihre
  Ankunft in der \enquote{Affäre Gregor}: \enquote{1. Ich habe ihn und
  übrigens ohne ihn zu sehen in der Bibliothek bemerkt, Manuskriptsaal
  für seltene Manuskripte 2. Er lachte beim Lesen von Milton
  {[}\ldots{]}} (Cixous, 2010, S. 83) Welche, für Cixous nicht
  untypische, Assoziationskette sich hier eröffnet: Manuskripte --
  Milton -- Das verlorene Paradies -- im Lesesaal!} Dort findet sie den
eigentlichen Gegenstand ihres Begehrens, ihres Heimatgefühls, das
wiederum um drei Odysseen zirkuliert: Homer, Shakespeare und natürlich
Joyces \enquote{Ulysses}\footnote{Und auch hier: Wieder Reisen,
  Vereinzelung, Irrfahrt - Die unendlichen literaturgeschichtlich
  durchwobenen Interpretationsansätze des Buches auch in Hinblick auf
  die Bibliothekswissenschaft wären ein wunderbares Unterfangen, das
  hier leider nicht gewagt werden kann. Ein weiteres Motiv ist die
  Relation Jonas-Wal. Vgl. dazu Derrida, Jaques (2003) Genesen,
  Genealogien, Genres und das Genie. Das Geheimnis des Archivs. Wien:
  Passagen, S.~20\,f.}, die mit den USA oberflächlich nichts anderes als
die Tatsache verbindet, dass man ihnen dort Obdach gewährt:

\begin{quote}
\enquote{Ich ging voll Leidenschaft und Ehrerbietung diese Manuskripte
konsultieren, leuchtende Überreste von Werken die in meinen Augen heilig
waren und die Europa und seine Schulen nicht gewollt hatten während die
gefräßigen amerikanischen Bibliotheken diese Reliquien ganz
offensichtlich doch gewollt hatten die also durch glücklichen Zufall auf
der anderen Seite der Verbannung aufgenommen und bewahrt waren
{[}\ldots{]}} (Cixous, 2010, S. 76)
\end{quote}

\noindent Die Bibliothek wird zur Zufluchtszone des in Europa
Geschmähten,\footnote{Sind es die Reisen, ist es das Unterwegssein
  selbst, das Verfolgen der Spuren bis zu einem Anfang, das in den
  Lesenischen der amerikanischen Bibliotheken unterschlüpft, während es
  das alte Europa verwirft, ausschließt, fortschickt?} das, wie es
scheint, nicht vollständig, sondern nur als Überrest, in jedem Fall mit
einer Bedeutungsverschiebung, mehr als wertvolles Sammelgut denn seiner
kulturgeschichtlichen Spuren wegen fernab des ursprünglichen
Bezugsraumes aufbewahrt wird.

\hypertarget{zwei-frauen}{%
\subsection*{Zwei Frauen}\label{zwei-frauen}}

In wenigstens einem Punkt treffen sich der Kern der Reise der
Protagonistin Hélène Cixous' mit dem der Autobiografie Azar
Nafisis\footnote{Nafisi, Azar (2003): Reading Lolita in Tehran. A Memoir
  in Books. New York: Random House} über ihre Zeit als Professorin für
englische Sprache in der iranischen Hauptstadt. Das Geschriebene wird
zur Zuflucht, zum Asyl. Jedoch unterscheidet sich die Form der
schützenden Nische: Cixous benötigt die Bibliothek als Schlupfwinkel,
für Nafisi ist sie fast ohne Belang. Das Wort Bibliothek findet sich in
Nafisis Buch kaum ein Dutzend Mal und nie als Ort des Lesens. Für Cixous
entfaltet sich in diesem Motiv ein zentraler Topos.\footnote{Genau
  genommen findet sich der Ausdruck \enquote{Bibliothek} etwas mehr als
  20-Mal im Text, ist aber als zentraler Gegenstand weitaus präsenter.
  Aber natürlich sind beide Bücher Literaturen über Literatur.} Das
Lesen findet bei Nafisi ausdrücklich nicht in der Bibliothek als
öffentlichem Ort, sondern in privaten Räumen statt. Dort eröffnet an
Donnerstagmorgenden amerikanische Literatur den Frauen im Wohnzimmer
Nafisis einen Erkenntnisraum und bildet zugleich einen Gegenpol zu einer
männlich dominierten Identitätspolitik. Bei Cixous ist es die
intellektuelle Französin, die sich in den Bibliotheken der USA in
nicht-amerikanische Literatur flüchtet und in eine unglaubliche,
tatsächlich und buchstäblich kafkaeske Affäre mit der Literatur
eintritt, die durch einen Gregor (!) und nicht etwa Karl verkörpert
wird\footnote{Wobei Karl Rossmann beispielsweise in der Grand Central
  Station erinnert wird und ausführlich bei einer Reflexion über Amerika
  und die USA seine Referenzen erhält.} und eigentlich Franz meint. Sie
verfällt einem Amerikaner, der sie als Kafka umfängt und so mit einem
Idealbild täuscht, das sich selbstverständlich nicht auf Dauer erhalten
lässt und in der Nachsicht grundlegend die Frage nach der Identität
eines Narrativs\footnote{Den Begriff, der zu einem zentralen Topos
  dieses Textes wird, erläutere ich unten ausführlicher.} aufwirft.
Dabei begegnen wir in beiden Fällen dezidiert feministischen
Perspektiven auf das Lesen: einmal in Gemeinschaft und zur Loslösung von
der dominanten Männlichkeit, das andere Mal in Vereinzelung direkt auf
eine Anbindung hinzulaufend. Das eine Mal wird das Narrativ einer
Gesellschaft hinterfragt, das andere Mal das des Einzelnen.

\hypertarget{ein-aus-musterungen}{%
\subsection*{Ein-/Aus-/Musterungen}\label{ein-aus-musterungen}}

Das Lesen bzw. die Schrift \emph{wirkt} je nach Anwendung als Binde-
oder Lösungsmittel. Sie wirkt, ist also prägend, lenkend und damit mit
dem Phänomen der Macht verbunden. Marco Roth bemerkte in seiner
Betrachtung zu \enquote{Reading Lolita in Tehran}\footnote{Roth, Marco:
  Why Reading Only Matters When It's Somewhere Else. In: n+1. Issue
  No.~1 (July 2004): Negation.},: \enquote{Nafisi erinnert uns daran,
dass Regierungen wie Autoren sind; sie stülpen der Gesellschaft eine
Narration über.} (Roth, 2008, S. 54)

Während die Regierungen versuchen, einer Gesellschaft mit mehr oder
weniger Erfolg und Nachdrücklichkeit Erzählungen
einzuschreiben\footnote{Hier liegt selbstverständlich die Assoziation zu
  Lyotards Konzept der \enquote{Großen Erzählungen} nahe.}, sind die
Bibliotheken traditionell der Ort, an der diese (und auch andere)
Erzählungen gesammelt, erschlossen und für den Zugriff vorgehalten
werden. Die Bibliothek ist aus ihrer Geschichte heraus also auf einer
Ebene ein durchweg politischer Ort, in jedem Fall ein Symbol, das für
einen bestimmten Gesellschaftsentwurf steht.\footnote{Ich betrachte an
  dieser Stelle nur Bibliotheken, die in irgendeiner Form der
  Öffentlichkeit zugänglich sein könnten. Privatbibliotheken und
  ähnliche Einrichtungen klammere ich in diesem Zusammenhang aus.} Dies
schließt zweifellos auch die Funktion der Bibliotheken als Ort der
Selbstaufklärung mit ein. Die Lenkung des Diskurses muss nicht
zwangsläufig auf der Ebene konkreter Botschaften oder Texte erfolgen,
sondern ist prinzipiell in der Entscheidung, wie und für wen eine
Bibliothek nutzbar ist, enthalten.

Das exklusive Bibliotheksbild Europas, wie es bei Cixous'
Manhattan-Projekt hindurch schimmert, findet seine Übersteigerung in den
Sammlungen deutlich zensurwilligerer Gesellschaftssysteme wie vielleicht
dem iranischen sowie ihr Gegenbild in freihändigen, inklusiven
Vorstellungen, die traditionell der amerikanischen Traditionslinie
nachgesagt werden. Es stehen hier zwei Dimensionen des Zugangs im
Zentrum der Betrachtung: der des Zugangs der Texte in die Bibliotheken
und der des Lesers zu den Quellen in den Bibliotheken. Das Erstere geht
naturgemäß dem Zweiten voraus. Beide Facetten und ihre Koordination
bilden das Herzstück der bibliothekarischen Arbeit, deren Ausdruck in
der grundsätzlichen Frage liegt: Was wird für wen zugänglich bewahrt?
Oder noch allgemeiner: Was wird bewahrt, was wird vermittelt, was wird
dem Verlorengehen anheimgegeben?\footnote{Die Frage lässt sich
  zweifellos auch als Abwandlung von Michel Foucaults Interpretation des
  Archivs als \enquote{Gesetz dessen, was gesagt werden kann} (1973, S.
  187) verstehen, wobei es mir im vorliegenden Gedankengang gerade auch
  darum geht, die Begriffe Archiv und Bibliothek zu separieren. Eine
  Betrachtung der Institution \enquote{Bibliothek} als Sonderform des
  Archivs (nach Foucault) muss an anderer Stelle ausführlicher erfolgen.}

Und als Zugabe: Was sträubt sich wie gegen dieses Ausgegrenzt-sein?
Inwiefern ist das Bewahren nur eine andere Form des Verlustes? Das
Verlorene, welches nicht in die Bibliothek gelangt, welches abgeschieden
ist, bleibt in gewisser Weise im Geheimnis.\footnote{Deutlicher wird es
  beim französischen \emph{Secret}: das Abgesonderte und damit nicht
  Geteilte. Vgl. dazu Derrida, (2003), S. 27.} Jedes Geheimnis, das
nicht in ein Archiv oder eine Bibliothek gelangt, bleibt verborgen und
vergeht mit seinen Trägern. Gelangt es dagegen in eine Bibliothek, geht
es in einen definierten Zustand des Bewahrt-seins über.

\hypertarget{das-lesbare-das-erzuxe4hlbare}{%
\subsection*{Das Lesbare, das
Erzählbare}\label{das-lesbare-das-erzuxe4hlbare}}

Wie lässt sich dieser Zustand beschreiben? Was geschieht in einer
Bibliothek mit dem Geheimnis?

\begin{quote}
\enquote{Man wird die Bibliothek im allgemeinen als jenen Ort
definieren, der dazu bestimmt ist, das Geheimnis zu wahren/aufzubewahren
(\emph{garder le secret}), aber als sich verlierendes. Ein Geheimnis
verlieren, das kann sowohl bedeuten, es zu enthüllen, es publik zu
machen, unter die Leute zu bringen, als auch, es derart tief in der
Krypta eines Gedächtnisses zu bewahren, daß man es vergißt oder sogar
aufhört, es zu verstehen und Zugang zu ihm zu haben. In diesem Sinne ist
ein gewahrtes/aufbewahrtes Geheimnis stets ein verlorenes
Geheimnis.}\footnote{vgl. ebd.}
\end{quote}

\noindent Enthüllung oder Verhüllung, Publikum oder Krypta -- diese
Gegensätzlichkeit am selben Ort steht als eine Art Leitklammer über
meinen Überlegungen, auch wenn wir, vielleicht unter dem Verlust der
Poesie, die immer eng verwandt mit dem Geheimnis (dem Abgeschiedenen,
dem Nicht-Bewussten, dem nur teilweise Erkennbaren) auftritt, das
Geheimnis streichen und das Erzählte daraus machen: Also das
\emph{Narrativ}. Ich gehe von der These aus, dass unsere Wahrnehmung
durch Narrative (die zweifelsohne ihren Ursprung im Mythischen/Mythos
haben) vor-, durch- und nachstrukturiert wird. Dies geschieht in einer
Anknüpfung an den Begriff der Fiktion bei Elena Esposito. (Esposito,
2007) Esposito analysiert in ihrem Essay zur \enquote{Fiktion der
wahrscheinlichen Realität} den fiktionalen Charakter unter anderem
ökonomischer Modelle und sieht diese wie auch die
Wahrscheinlichkeitstheorie als Konstrukte, die weder wahr noch falsch
sind, sondern Möglichkeitsformen von Realität formulieren, die im
Zusammenwirken die \emph{reale Realität} nach gewissen Kriterien
spiegeln, dieser aber nicht entsprechen. Besonders deutlich wird dies in
der literarischen \emph{Fiction}:

\begin{quote}
\enquote{\emph{Fiction} wird {[}\ldots{]} zum Spiegel, in dem die
Gesellschaft ihre eigene Kontingenz reflektiert, die Normalität einer
nicht mehr eindeutig festgelegten und bestimmbaren Form.} (Esposito,
2007, S. 18)
\end{quote}

\noindent Ich teile sowohl Espositos Sicht zur Rolle der Fiktion wie auch die
Diagnose der Herausforderung des Menschen durch die Kontingenz und seine
Schwierigkeiten im Umgang mit dem Nicht-Wissen. Für meine Betrachtungen,
die perspektivisch zu einem semiotischen Entwurf der
Bibliothekswissenschaft führen sollen, möchte ich den Begriff der
Fiktion jedoch ein wenig erweitern. Jede Form Sinnkonstruktion ist dabei
zugleich eine Fiktionalisierung, die freilich nicht beliebig sein kann,
sondern sich kohärent zu anderen Fiktionalisierungen inklusive der
Fiktion der Fiktion verhalten muss.\footnote{Zur literarischen Fiktion
  schreibt Esposito (2007): \enquote{Der Roman {[}\ldots{]} muß eine
  Welt entwerfen, die der direkt erfahrbaren Welt an Kohärenz entspricht
  oder diese gar übertrifft.} (S. 19) \emph{imaginär} betont bei ihr
  ausdrücklich den Gegensatz zu \emph{tatsächlich}.} Fiktionen
erscheinen dabei als -- aufgrund bestimmter Kriterien abgetrennter --
Beobachtungsbereiche. Obschon in ihnen alles in seiner Entwicklung einen
anderen Verlauf nehmen kann, muss der Verlauf im Vollzug stimmig
interpretierbar sein. Sinnkonstruktion und Ereignisse verhalten sich
dabei -- sofern es von menschlichem Verhalten geprägte Ereignisse
handelt -- immer in Wechselwirkung: Der Beobachter beeinflusst die
Wahrnehmung des Beobachteten (die Beobachtung) und zugleich verändert
sich seine Wahrnehmungsfähigkeit. Sind sich die beobachteten Akteure der
Beobachtung bewusst, verändern auch sie ihr Handeln und damit den
Gesamtverlauf. Die Beobachtung ist dabei ein interpretierendes
(Sinn-erzeugendes Handeln). Es geht ihr -- wie auch in einem Roman --
nicht um wahr oder falsch, sondern darum, dass das, was beobachtet wird,
stimmig erscheint. Sie muss wahrscheinlich sein. Zusätzlich beschäftigt
sich die Interpretation wenigstens bei postmodernen Ansätzen zur
Sinnerzeugung betont mit den Aporien und Idiosynkrasien. Die
Unstimmigkeit wird nicht ausgeblendet, sondern mit dem Ziel betont, eine
Stimmigkeit höherer Ordnung zu entwickeln.\footnote{Diese Unterscheidung
  zwischen Moderne und Postmoderne deckt sich in gewissem Umfang mit
  Esposito (2007), wenn diese schreibt: \enquote{Anstatt \emph{gegen}
  die Unbestimmtheit vorzugehen, die nichts anderes ist als Kontingenz
  und Komplexität, arbeitet man \emph{mit} ihr und versucht
  Anhaltspunkte {[}für ein angemessenes Handeln{]} aus ihr abzuleiten.}
  (S. 64).}

\hypertarget{das-narrativ}{%
\subsection*{Das Narrativ}\label{das-narrativ}}

Zur Beschreibung dieser multiplen fiktiven Realitäten, die sich als
Gesellschaftsentwürfe oder Wirtschaftssysteme in Institutionen
manifestieren und in gewisser Weise genauso real realit werden wie sie
als individuelle Biographien den tatsächlichen (auch körperlichen)
Lebensvollzug eines Menschen mit interpretatorisch und selbst-reflexiv
gewonnen Selbstbildern koppeln, greife ich auf den Ausdruck des
\emph{Narrativs} zurück. Die Auswahl ist in einem gewissem Rahmen
ebenfalls kontingent: Espositos \emph{Fiction} bzw. die Fiktion wie --
in Anlehnung an Gérard Genette -- \emph{récit} oder solche Phänomene wie
Richard Dawkins \emph{Mem} oder auch Michel Foucaults \emph{Episteme}
hätten als Ausgangspunkt herangezogen werden können.\footnote{Gegen
  Fiktion spricht allerdings, dass sie per se rein imaginär ist
  (\enquote{Sie {[}die Fiktion{]} konstruiert eine kohärente Welt auf
  der Grundlage ausdrücklich imaginärer Prämissen.} (Espositio, 2007, S.
  55f.), die vorliegende Idee aber gerade das Spiel zwischen
  Tatsächlichkeit, der sinnerzeugenden Wiedergabe der Wahrnehmung von
  Tatsächlichkeit und die graduelle Fiktionalisierung von
  Tatsächlichkeit während der Wiedergabe in den Mittelpunkt stellt.}

Der Ausdruck Narrativ erscheint mir deswegen zutreffender, weil er den
Werk- wie Textcharakter einer entsprechenden Sinndarstellung adressiert
und somit den Bezugsobjekten von Bibliotheken gerecht wird, weil er
erzählerische und interpretatorische (= Sinn konstruierende) Prozesse
enthält und zugleich eine Abgrenzung zur nicht-erzählten,
nicht-entworfenen Form des tatsächlichen Ereignisses\footnote{Wobei ein
  Text, also ein Narrativ wiederum Objekt eines Ereignisses sein.}
ermöglicht. Letzteres wird freilich durch Narrative verzeichnet und
vermittelt. Die Beziehung von Ereignis und Narrativ entspricht für mich
der Beziehung zwischen der Wahrnehmung und ihrer
Interpretation.\footnote{Interessant ist auch die Gegenüberstellung von
  \emph{Dokument} und \emph{Narrativ}. Ein Dokument ist nach meinem
  Verständnis in diesem Zusammenhang die Spur des Ereignisses ohne eine
  erzählerische und interpretatorische Aufbereitung. Ob etwas Dokument
  oder Narrativ ist, ist wiederum abhängig von der Perspektive:
  artefaktische Dokumente tragen immer auch eine narrative Inschrift
  (Vorgebung, Markierung, et cetera) wogegen Narrative selbst und mehr
  noch ihre Repräsentationen als Dokumente betrachtet und verarbeitet
  werden können. Die Übergänge sind, wie so oft, verwischt.} Unter
Narrativ verstehe ich also ein abgrenzbares und bestimmbares
Sinnkonstrukt, das auf den permanenten Prozess der Kultur einwirkt.
Narrative liegen sowohl explizit in Gestalt von Repräsentationen
(Text\footnote{Ich verstehe \emph{Text} weit gefasst als alles, was
  semiotisch strukturiert ist, also auf Zeichen, ihre Bedeutung und
  einer Verwendungsintention basiert und damit mit bestimmten
  Bedeutungen und Intentionen aufgeladen kommunizierbar ist.}) wie auch
implizit in die Kultur eingeschrieben vor.

Es geht um unsere Erklärungsmuster, die, auch wenn sie eine algebraische
Logik aufweisen, im Kern Erzählungen sind, aus denen wir Sinn gewinnen.
Nach der zitierten Derrida'ischen Feststellung zur Bibliothek stellt
sich nun weniger die Frage nach dem \enquote{Was es ist}, sondern
\enquote{Wie es ist}. Dabei gilt allgemein: Es ist, was wir sind. Die
Bibliothek fasst mit ihren Beständen unsere Kultur. Sezieren wir das,
wofür das Konzept der Bibliothek steht, legen wir einen zentralen Teil
dessen frei, woraus wir unsere Identität konstruieren: Was bewahren wir,
was verwerfen wir? Was halten wir für lesbar? Von uns? Und mit welchen
Auswirkungen?

\hypertarget{zukuxfcnfte}{%
\subsection*{Zukünfte}\label{zukuxfcnfte}}

Espositos Essay zur Prognostizierbarkeit von Zukunft wirft eine für
diese Frage maßgebliche Differenzierung auf: Sie unterscheidet den
Zeitpunkt der Perspektive auf die Zukunft. (Esposito, 2007, S. 30\,ff.)
Was wir unter Zukunft verstehen, wenn wir für diese Modelle entwickeln,
ist eine \enquote{zukünftige Gegenwart}. Unsere Position entspricht also
der \enquote{gegenwärtigen {[}Sicht auf die{]} Zukunft.} Analog dazu
kann man sich fragen, wie wir uns zukünftig prognostisch verhalten bzw.
zu welchen Vorannahmen über die Entwicklung von Kultur wir in der Lage
sein werden. Dabei müsste man von einer \emph{zukünftigen Zukunft}
sprechen. Ihr Gegenüber steht unsere Gegenwart, die aus der Rückschau
als \emph{zukünftige Vergangenheit} erscheint. Das ist besonders für die
Frage von moralischen Entscheidungen und ihrer Legitimation interessant:
Wie wollen wir, dass unser jetziges Handeln zu späterer Zeit beurteilt
wird? Für das Bewahren bedeutet dies, zu antizipieren, welche unserer
aktuellen Narrativen wir als relevant für zukünftige Generationen
ansehen. Eine historische Analyse der Bibliotheken, also die Geschichte
der Auswahl, der Ordnung und des Aufhebens von Narrativen bzw. der
Zugangsregulierung kann dabei hilfreich sein, eröffnet sie doch den
Blick auf \emph{vergangene Zukünfte}, d.h. auf \emph{vergangene
Gegenwarten} sowie unsere \emph{aktuelle Gegenwart}. Schließlich lässt
sich auch in einer meta-historischen Analyse dieser retrospektive
Betrachtungsansatz auffächern, wenn wir nach \emph{vergangenen
Vergangenheiten} fragen. Auf dieser Basis lässt sich anhand der
Diskursverläufe einzelner narrativer Einheiten die Entwicklung von
Deutungen nachzeichnen, aus denen sich möglicherweise Muster ergeben,
die wir nicht vermuten.

Eine diskursanalytische Durchleuchtung der Auswahl, Ordnungen und
Überlieferungen von Narrativen bietet dabei einer transdisziplinär
orientierten Bibliothekswissenschaft eine Grundlage für eine
Auseinandersetzung mit dem Phänomen der Kontingenz und der Chance auf
emergente Erkenntnisse. Auch diese werden uns die Zukunft nicht
unbedingt objektiv vorhersagbar machen. Aber sie können unser
Verständnis für das, was geschehen ist, vertiefen und damit unser
Handlungsbewusstsein schärfen. Das gilt gleichermaßen konkret für das
bibliothekarische Handeln wie abstrakter für das wissenschaftliche bzw.
jede andere Form des Handelns.

\hypertarget{schlieuxdfungen-zuguxe4nge}{%
\subsection*{Schließungen, Zugänge}\label{schlieuxdfungen-zuguxe4nge}}

Bibliotheken lassen sich prozessual als ein mehrschichtiges Ein- und
Ausschließen verstehen. Es lässt sich für sie (bzw. alle ähnlich
ordnenden Institutionen, also auch Archive und Museen) eine dreiteilige,
kategorial vorbestimmte (Un-)Zugänglichkeit des in ihnen Bewahrten (des
Geheimnisses, der Erzählungen) erkennen\footnote{Für Fiktionen findet
  sich eine Ähnliche Aufarbeitung bei Derrida: \enquote{Es sind
  praktische Fragen, zweifellos, praktisch im zunächst einmal
  technischen Sinne des Wortes (Klassifikation, Datierung,
  Kategorisierung, Erfassung in einer Kartei, interne Begrenzungen des
  Korpus), aber auch praktische Fragen im ethischen und deontologischen
  Sinne des Wortes (Was darf an zurecht als literarische Fiktion oder
  als nicht-literarisches Dokument klassifizieren? Wer autorisiert wen,
  in einem öffentlichen literarischen Werk etwas Geheimes oder etwas
  Nicht-Geheimes zu enthüllen? Wer autorisiert wen und erlaubt sich was,
  um die Bekanntmachung einer bestimmten identifizierbaren Filiation in
  der Genese des Werks zu gestatten {[}\ldots{]}} (Derrida, 2003, S. 63)
  Für die hier vorgenommene Betrachtung ist besonders der Aspekt der
  \emph{Autorisierung} hinsichtlich der Bestandsauswahl, -ordnung und
  -vermittlung relevant. Die genannten technischen Aspekte lassen sich
  ohnehin auch in Hinblick auf die Funktionsweise der Bibliothek
  spiegeln.}:

\begin{enumerate}
\def\labelenumi{\arabic{enumi}.}
\item
  was in den Bestand gelangen kann (welches Narrativ als gültig
  anerkannt wird),
\item
  nach welcher Ordnung erschlossen und in gewisser Weise wieder aus dem
  Bestand in eine Rezeption (oder auch: Aktualisierung) gerufen werden
  kann (Publikum oder Krypta, bzw. nach welchen Kriterien wird es als
  gültig erhalten bzw. aktualisiert wird), sowie
\item
  wer diese Aktualisierung vornehmen kann (welche Akteure dabei aktiv
  sind).
\end{enumerate}

\noindent Der letzte Punkt sowie die Aspekte der Aussonderung und Umordnung von
Beständen und der Umorganisation der Zugangs- und Nutzungsmöglichkeiten
ergänzt eine zusätzliche Dimension in der Zeit, wie sie vorgehend
beschrieben wurde.

Die Rolle der Bibliothek ist also die einer bewahrenden und damit
vermittelnden Institution, die sich über das Phänomen des Narrativs auch
als Modell für die Konstitutionsprozesse dessen heranziehen lässt, was
wir Kultur nennen.

\enquote{To Classify is human} überschreiben Geoffrey Bowker und Susan
Leigh Star die Einleitung ihres eher zufällig hoch
bibliothekswissenschaftlichen Buches \enquote{Sor\-ting Things Out}.
(Bowker und Star, 2000) Wir können der Klassifikation nicht entkommen.
Die Frage ist, wie wir sie ausüben. Mir geht es darum, mit der
Perspektive auf die \emph{Bibliothekswissenschaft} als einer besonderen
Narratologie greifbar zu machen, was eine für den absehbaren Rahmen der
Gegenwart und gegenwärtigen Zukunft Bibliothekswissenschaft auszeichnen
kann. Wollen wir dies ergründen und uns nicht darauf verlassen, dass es
sich um eine technisch dominierte Funktions- oder
Verwaltungswissenschaft handelt, dann haben wir in dem einfachen
Gedanken \emph{ich ordne},\footnote{klassifiziere, kategorisiere =
  unterscheide.} \emph{also bin ich} einen außerordentlich komplexen
Schlüssel. Die Bibliothekswissenschaft muss grundsätzlich nach den
Bedingungen und Folgen der jeweiligen Ordnungen und Ordnungstechnologien
für Narrative, die weitgehend selbst narrativ geprägt sind, fragen und
dabei versuchen, die Grenzen der Entwicklungsmöglichkeiten, also die
Kontingenz abzuschätzen.

\hypertarget{teil-2-bibliothekstopologische-uxfcberlegungen}{%
\section*{Teil 2: Bibliothekstopologische
Überlegungen}\label{teil-2-bibliothekstopologische-uxfcberlegungen}}

\enquote{Die mit der Fernkommunikation beseitigte Räumlichkeit der
'Topoi' taucht also ausgerechnet im Cyberspace wieder auf, der
vermeintlich den Sinn der Entfernung getilgt haben sollte.} -- Elena
Esposito, 2002, S. 46

\hypertarget{i.}{%
\subsection*{I.}\label{i.}}

Den Anlass zu dieser Betrachtung bot ein Kommentar Oliver Obsts in
dessen medinfo-weblog, der als Reaktion auf einen Beitrag\footnote{\url{http://libreas.wordpress.com/2011/02/10/731/}}
im LIBREAS-Weblog die Betrachtungsdichotomie von Content und Container
aufwarf. Ein Container als Ummantelungen verweist für mich naheliegender
Weise auf die Frage des Raumes und damit in diesem Zusammenhang auf die
Frage einer, wenn man so will, Bibliothekstopologie.

Sprechen wir von einer Bibliothekstopologie, dann beziehen wir uns auf
den Raum der Bibliothek und auf dessen Transformation. Die Richtung der
Veränderung ist durch einen Prozess der Dematerialisierung (oder auch
Immaterialisierung) bestimmt. Der Raum wird virtuell, also zu einem
\emph{scheinbaren}. Die Annäherung (ebenfalls ein Geschehen in einem
virtuellen Raum, nämlich in dem des Diskurses) erfolgt jedoch aus ganz
unterschiedlichen Winkeln. Oliver Obst argumentiert als Praktiker, der
(angenommene und erfragte) Nutzererwartungen in realweltliche Angebote
zur Informationsversorgung umwandeln muss. Ich betrachte das Ganze mit
einer dispositiv- und diskurstheoretischen Faszination, die ich
zufälligen Lektüren wie dieser verdanke:

\begin{quote}
\enquote{Ganz persönlich lässt mir die Existenz von Diskursen keine
Ruhe, die da sind, weil gesprochen worden ist; diese Ereignisse haben
einst im Rahmen ihrer ursprünglichen Situation funktioniert; sie haben
Spuren hinterlassen, bestehen weiter fort und üben durch dieses
Fortbestehen innerhalb der Geschichte eine Reihe manifester oder
verborgener Funktionen aus.} (Michel Foucault, 2009, S. 35)
\end{quote}

\noindent So formuliert Michel Foucault im Gespräch mit Raymond Bellour die
reizvolle Basis eines Bezugsrahmens, die die Bibliothek irgendwo in
einem Beziehungsgewebe aus Dispositiven, Diskursen und Spuren mit den
Handlungsweisen Diskursvollzug, Spurensicherung und Spurenlesen
verortbar macht. Jedenfalls gehe ich von einem solchen aus.

Lässt sich über die, wenn man so will, Diskursivschreibung des
Bibliotheksraums (im Gegensatz zur An- und Ausführungsbezeichnung der
\enquote{Bibliothek}) eine Apologie der Bibliothek als Raum erreichen?
Ist es möglich, auf dieser Grundlage eine Bibliothekstopologie zu
entwickeln, die vielleicht als Ergänzungsstück zur Bibliothekslogistik,
die natürlich selbst immer auch eine raumordnende Komponente besitzt,
verstanden werden kann? Der Unterschied der Positionen liegt weniger in
der Sache selbst, als in der Art, sich ihr anzunähern. Zunächst einmal
sollten wir also eine möglicherweise wahrgenommene Opposition
ausschließen, die sich als Nebenwirkung mit dem klobigen, aber doch
passenden Begriff des Containers in den Dialog schlich. Die Anerkennung
der Bedeutung des realen Bibliotheksraums meint keinesfalls die
Ablehnung des virtuellen. Eher im Gegenteil. Ob physisch oder digital:
Beide Raumformen sind Phänomene, zu denen eine neutrale Ausgangsposition
nicht der schlechteste Startpunkt ist.

Von dieser sehen wir, dass ein Irrtum in der ab und an anzutreffenden
Annahme liegt, dass wir den Raum verlieren, wenn wir digitalisieren.
Eines der Zitate, die Oliver Obst in den Dialog setzt, zeigt deutlich,
dass dem nicht so ist:

\begin{quote}
\enquote{The definition of the library will change as physical space is
repurposed and virtual space expands.} (ACRL Research Planning and Review Committee, 2010, S.~286--292)
\end{quote}

\noindent Die Digitale Bibliothek bedeutet kein Verschwinden des
Bibliotheksraumes, sondern führt ihn im Gegenteil zu einer virtuellen
Totalität. Zudem semiotisiert sie ihn durch die Entfernung der
physischen Merkmale. Der Raum der digitalen Bibliothek ist mehr denn je
logotopisch. Nimmt man die Komponente der Zeit in ihren vielschichtigen
Bezügen:

\begin{itemize}
\item
  Zeitpunkt der Erzeugung eines Inhalts (Kreation)
\item
  Zeitpunkt der Aufnahme eines Inhalts in den Bestand (Registrierung)
\item
  Zeitpunkte des Abrufs eines Inhalts (Rezeptionen)
\item
  Zeitpunkte der Referenzierung eines Inhalts (Relationierungen)
\item
  Mitunter: Zeitpunkt des Ausscheidens eines Inhalts (Exklusion)
\item
  Zeitliche Abstände zwischen den genannten Zeitpunkten
  (Chronorelationen)
\end{itemize}

\noindent wird deutlich, dass die Bibliothek idealerweise auch
\emph{chronotopisch} gedacht werden muss. Die Digitale Bibliothek, die
über automatisierte und präzise Mechanismen zur Markierung der
Zeitpunkte und generell zur Chronographie verfügt, nähert sich dieser
Vorstellung -- und damit auch der die Freizügigkeit des Zugriffs
flankierenden Überwachbarkeit der Nutzungen -- weitaus stärker an, als
es in analogen Bibliotheksräumen je realisierbar war. Insofern zeigt
sich die Spiegelung des genuin literaturwissenschaftlichen Bild des
Chronotopos wenigstens dann als interessante Denkfigur, wenn man
bedenkt, dass der virtuelle Raum ähnlich dem literarischen über Codes
und also Sprachen erzeugt wird. (Natürlich sind wir an dieser Stelle
schon etwas weiter von Bachtins ursprünglichem Begriffszuschnitt
entfernt.) Der digitale Bibliotheksraum wird also in der Zeit über
digital kommunizierte Zeichen abgesteckt, für jedes Nutzungsszenario
individuell erzeugt und behält vom Physischen notwendig nur das Sediment
der Kammer, in der der Server steht.

Was der Medienwissenschaftler Wolfgang Ernst in seinen Betrachtungen zum
\enquote{Rumoren der Archive} schreibt, zeigt sich auch für die
Transformation des Bibliotheksraumes stimmig: \enquote{Die modernen
Archive waren die längste Zeit architektonische Makrochips, auf Dauer
angelegt und ausdauernd, Monumente der Dokumentation {[}\ldots{]}} (S.
41). Der Raum der digitalen Bibliothek bricht die Einheit von Speichern
und Übertragen, wie sie noch in den Freihandaufstellungen (auch ein
Ernst'scher \enquote{Makrochip}) zu finden ist; und stellt die
Übertragung über den Aspekt des Speicherns dar. Dieser wird aus der
Wahrnehmung in ein digitales Magazin verbannt. Die digitale Bibliothek
ist in dieser Hinsicht eine vollautomatisierte Magazinbibliothek: Die
Bestände sind unsichtbar gelagert, werden aber auf Anfrage abgerufen und
nutzbar gemacht. Das raumgestalterische Engagement kann sich ganz auf
die Ausgestaltung der Bereiche der Übertragung und Nutzung
konzentrieren.

Wir haben es meiner Auffassung nach mit einer Kombination der drei
großen C: \emph{Community -- Container -- Content} zu tun, die sich in
einem vierten C, der \emph{Communication} bündeln. \enquote{Container}
entspricht dem Übertragungsmedium. Da eine Übertragung immer räumlich
gedacht wird und meines Erachtens auch vorgestellt werden muss, müssen
wir den Container \enquote{Bibliotheksraum} selbst als Medium
betrachten. Die Frage muss also besser lauten: Wie verändert sich das
Medium Bibliothek?

Das Verständnis der Bibliothek selbst als medialen Apparat, also als
Übertragungsinstanz, wird in digitalen Kontexten augenfällig (obwohl ich
davon ausgehe, dass sie immer schon gegeben war): die Bibliothek ist
genuin maßgeblicher Bestandteil der Geschichten gewesen, die in ihr
verwahrt und durch sie vermittelt werden (ausführlicher dazu Ben Kaden,
2010). Inhalt und Träger müssen also zusammen gedacht werden, was die
Frage danach aufwirft, wie eine Veränderung des Trägers für den Inhalt
wirkt -- bzw. auf den aus diesem Inhalt zu konstruierenden Sinn. Die
Verwandlung des strukturell entropischen realweltlichen Bibliotheksraums
in einen quasi neg-entropischen digitalen Cyberspace ist nicht
folgenlos. Was die Bibliothekswissenschaft leisten muss, ist, diese
Folgen in ihrer Qualität zu erschließen und zu beschreiben.

\hypertarget{ii.}{%
\subsection*{II.}\label{ii.}}

In der Januarausgabe 2011 der Zeitschrift \emph{Information Wissenschaft
\& Praxis} weist Rafael Capurro auf die Verschiebung von der
Blumenberg'schen Leitmetapher einer \enquote{Lesbarkeit der Welt} zu
einer, nicht ganz so poetisch klingenden, \enquote{Mitteilungskompetenz}
bzw. \enquote{digitalen Informierbarkeit der Welt} hin. Obschon ich
gegenüber der allzu paradiesischen Vorstellung, dass wir mit digitalen
Kommunikationsmöglichkeiten zu einer allumfassend literaten
Internetgesellschaft werden, skeptisch (oder wenigstens
skeptimistisch\footnote{\url{http://www.wordspy.com/words/skeptimistic.asp}}
bleibe, akzeptiere ich ohne zu zögern die strukturelle Tatsache einer
Verschiebung der Möglichkeiten zur erweiterten öffentlich sichtbaren,
genau dokumentierten und adressierbaren Kommunikation. Die prinzipielle
Teilhabeschwelle sinkt und zwar auch in dem per se auf Rückkopplung
ausgelegten Bereich der Wissenschaftskommunikation. Andererseits ist zu
erwarten, dass neue Distinktionsformen aus dieser digitalen
Kommunikationsgemeinschaft keine ideale, sondern eine sozial
durchrubrizierte machen, die die feinen Unterschiede der physischen Welt
reproduziert.

Der soziale Umgang mit Raum vollzieht sich auf der abstraktesten Ebene
in einer Wechselwirkung aus Markierung und einem Anerkennen oder Angehen
gegen diese Markierung. Man muss diesen Vorgang weder als fortwährenden
Kampf noch -- wie Michel Serres -- als permanente Kontamination
verstehen, sondern kann dazu ein mehr spielerisches Verhältnis
entwickeln. Mit der Digitalisierung des Raumes entsteht die Möglichkeit,
die Perspektive auf den Raum von einer am Konzept des Eigentums
orientierten Verfügungsressource in einer performativen
Entfaltungssphäre zu lenken. Eine physische Kontrolle des Cyberspace ist
so gut wie unmöglich (Es sei denn, man schaltet den Strom ab.) Die
Kontrollmöglichkeiten digitaler Räume beschränken sich auf die
Überwachung von Codierungen. Das Potential digitaler Subversion liegt im
permanenten Umcodieren (z.B. der Adressierungen, die die Grundlage der
Mobilität bestimmter Webseiten sind). Andererseits sinkt mit diesen
Verschiebungen unter Umständen die Chance auf Wahrnehmung. Das
Gegenmittel der Kontrollinstanzen zur digitalen Subversion ist daher
nicht die eigentliche Beherrschung des Geschehens, sondern die
permanente Marginalisierung. Insofern hat Rafael Capurro mit dem Konzept
der \enquote{digitalen Informierbarkeit der Welt} durchaus Recht,
allerdings hinsichtlich der machtpolitischen Dimension mit einer anderen
Konnotation: die richtige Mitteilung im richtigen Moment an der
richtigen Stelle zu platzieren ist eine -- vielleicht die einzige
innerhalb der webdiskursiven Spielregeln -- Möglichkeit der Kontrolle
dieses diskursiven Raumes. Denn gerade weil jede Äußerung präsent werden
kann, entwickelt sich eine neue Flüchtigkeit, die nicht dazu führt, dass
etwas nicht mehr gelesen werden kann, weil es verschwindet, sondern dass
etwas verschwindet, weil es nicht mehr gelesen wird. Eine Qualität, die
in jeder Überlegung zur digitalen Bibliothek mitschwingt, liegt in den
neuen Verknappungen einer Größe, die man grob als
\enquote{Aufmerksamkeit} bezeichnen kann. So wie die menschliche Physis
in ihrer Begrenztheit eine Einmaligkeit bedingt, nämlich die
Befindlichkeit an nur einem Ort in nur einem Moment, so wird dieses
Phänomen in der kognitiven Präsenz gespiegelt. Die Auffächerung der
Anwesenheit gelingt auch unter digitalen Bedingungen nur geringfügig:
Wir können dank kommunikativer Erweiterungen, zu denen übrigens auch die
Technologie der Erzählung bzw. des Textes zählt, die kognitive Präsenz
von der physischen ein Stück weit abspalten. Wir können im Kopf an einem
anderen Ort als der Kopf selbst sein.

Digitale Räume adressieren bislang nahezu ausschließlich die kognitive
Präsenz und da sie beliebig durchquert werden können, ermöglichen sie
Aufmerksamkeitssprünge ohne jede bremsende materielle Trägheit. Wenn es
um Konzentration geht, bleiben wir jedoch Monaden: schon allein rein
sinnlich erweist sich Multitasking als eine Herausforderung, die bei
sehr viel kognitivem Aufwand ziemlich kleine Verschiebungen verspricht.
Eine halbwegs zeitgleich auf mehrere Phänomene gerichtete Wahrnehmung
ist bestenfalls eine registrierende. Sobald wir anfangen, selbst zu
schreiben, müssen wir -- und sei es nur für die Dauer einer SMS -- das
akzeptierte Reizniveau absenken. Das, was gemeinhin als schätzenswerte
Eigenschaft physischer Bibliotheksräume genannt wird, ist die räumliche
Realisierung genau dieser Absenkung: Sie wirken durch Ausschluss von
Sinnesreizen und Handlungsoptionen konzentrierend.

Das zweite Zitat, das Oliver Obst zur Relativierung des Containers ins
Spiel bringt, läuft von dieser Warte aus etwas ins Leere (allerdings
fehlt mir auch die volle Kenntnis des Ursprungskontexts):

\begin{quote}
\enquote{Muss dieser \enquote{ideale} Raum überhaupt etwas mit der Bibliothek/den
Bibliothekaren zu tun haben (\enquote{Stören wir da schon?}). Wenn jeder
mit einer Smartcard rein kann, muss das nicht die \enquote{Bibliothek}
sein. Auf die folgende Frage wusste keiner wirklich eine Antwort:
\enquote{Wenn die Auskunft oder die Ausleihe wegfällt oder automatisiert
wird, was definiert dann noch diesen Raum als Bibliothek?} In diesen
Zusammenhang denkt man an die
Extinction
Timeline, die für 2020 die Auslöschung bzw. das Unwichtigwerden von
Bibliotheken vorsieht und für 2040 diejenige von kostenfreien
öffentlichen Orten (\url{http://www.nowandnext.com/PDF/extinction_timeline.pdf}).} (Obst. 2010.)
\end{quote}

Denn die Umweltreiz senkende Eigenschaft des Bibliotheksraums erweist
sich als eine passive Dienstleistung, die in der Diskussion zur Rolle
der Bibliothek häufig übersehen wird. Hinter dem Zitat scheint
jedenfalls eine eher eingegrenzte, funktionale und vielleicht auch
verwaltungstechnisch gestaltete Sicht auf die Bibliothek am Werk, die
vernachlässigt, dass es eine naheliegende Antwort gibt: Der Aspekt, der
einen Raum als Bibliothek definiert, könnte die Klausur sein -- die
realräumlich angebotene Möglichkeit einer gewissen Harmonisierung von
physischer und kognitiver Präsenz. Das mag kein Modell für jeden sein
und nicht selten ging man genau dieses Bild aktiv an, da es eine
disziplinierende Strenge enthält, die als unzeitgemäß erschien. Wenn man
aber die Auslastung des Grimm-Zentrums betrachtet, ist es offensichtlich
eine aktuelle Option für viele.

Der Lesesaal einer Bibliothek ist weder Ort des Speicherns noch der
Auskunft noch der Ausleihe. Er ist der Ort der Übertragung und in
gewisser Weise das physische Gegenstück zum Computer (auch wenn der
Computer selbst wieder im Lesesaal stehen kann). Die Rezeption am
Rechner erfolgt über den mittlerweile anteilmäßig immer größeren Teil
eines Bildschirms, der Freihandregal und Buchseite gleichermaßen
abbildet, aber eben selbst bei Touchscreen-Technologien von materiellen
Zu-Werten umgeben ist, die die Rezeption dispositiv begleiten. Der
Nutzungsbereich einer Bibliothek begleitet die Rezeption ebenso
dispositiv. Nur in ganz anderer Weise.

\hypertarget{iii.}{%
\subsection*{III.}\label{iii.}}

Rafael Capurro bedauert in seinem Interview die Regression der
Informationswissenschaft auf eine
\enquote{Information-Retrieval-Wissenschaft}, die eine Vielzahl von
begleitenden Aspekten ausblendet. Er empfiehlt eine Neubestimmung:

\begin{quote}
\enquote{Man könnte sich für diese Neubestimmung auf die 'médiologie'
von Régis Debray orientieren, die den Schwerpunkt auf die Materialität
der Träger {[}\ldots{]} sowie die Vermittlungsinstitutionen legt. Ich
meine aber auch, dass die Medienwissenschaft und das, was ich
'Angeletik' nenne, also eine empirische Wissenschaft, die sich mit den
Boten und Botschaften auseinandersetzt, zum Kern dieser neuen
Informationswissenschaft gehört.} (Rafael Capurro, S.41)
\end{quote}

\noindent Ich würde dies gleichlautend auf die Bibliothekswissenschaft übertragen,
die meines Erachtens aus ihrer Anlage heraus näher an konkreten sozialen
Phänomenen operiert als die Informationswissenschaft. Die
Vermittlungsinstitution in Gestalt der als Medium definierten Bibliothek
zeigt sich dabei als ein aktualisierbarer Container, der nicht zwingend
über das Submedium Buch bestimmt wird. Das Besondere dieses Containers
ist seine Offenheit -- sowohl für die Botschaften (den Content) wie auch
die Boten (die Community). Insofern ist die Rolle des Containers
innerhalb dieses Netzes von Wechselwirkungen nicht schwächer, sondern
eher noch höher einzuschätzen. Und schließlich gehören zu den Boten und
Botschaften auch die Spuren, die sie hinterlassen, die man mit Serres
als das eigent\emph{liche Übel} lesen kann oder neutraler einfach nur
als Grundeigenschaft des Menschen. Die Bibliothek und ihr Raum selbst
sind Spur, so wie jede Spur auch Container dessen ist, worauf sie
verweist und jeder Bote Container seiner Botschaft.

Die Notwendigkeit, die drei \emph{Cs} zu trennen, wenn es darum geht,
funktionale Entscheidungen zu treffen, leuchtet ein. Aber sie beruht
immer auf einer Perspektivität und ist jeweils nur in einem gewissen
Kontext gültig. In einem anderen Zusammenhang könnten die Elemente, die
hier als Container, Content und Community differenziert werden,
problemlos ihre Rollen tauschen. Das führt nicht etwa in eine
Beliebigkeit, sondern in eine Bestimmtheit, bei der man die Kategorien
nicht mehr als generelle Festlegung, sondern als sich aus den
Wechselwirkungen zwischen vier Instanzen ergebend betrachtet. Man kann
hier auch von einer sich selbst regulierenden \enquote{kybernetischen
Semantik} sprechen. (vgl. Elena Esposito, 2002) Es ist ein Blick
dazwischen: der des Beobachters auf diese Triade. Sind wir uns dessen
bewusst, haben wir weitaus mehr Gestaltungsspielräume, als würden wir
alles auf die halb ironischen, halb harten Annahmen einer im Kern eher
instabileren \emph{Extinction Timeline} eines einzelnen Futuristen
setzen. Die Dispositiv- und Diskurstheorie lehrt uns jedenfalls, dass
die Dinge auch dann fortbestehen, wenn wir sie nicht mehr sehen. Und
dass wir irgendwie wahrscheinlich eine Archäologie erfinden, mit der wir
sie wieder ausgraben.

\begin{center}\rule{0.5\linewidth}{0.5pt}\end{center}

\hypertarget{literatur}{%
\section*{Literatur}\label{literatur}}
\begin{itemize}

\item ACRL Research Planning and Review Committee. 2010 top ten trends in academic libraries: A review of the current literature. Coll Res Library News. 2010; 71(6): 286--292. \url{http://crln.acrl.org/content/71/6/286.full}.

\item Barthes, Roland: Kritik und Wahrheit. Frankfurt/Main: Suhrkamp, 1967.

\item Bowker, Geoffrey C.; Star, Susan Leigh: Sorting Things Out. Classification and its Consequences. Cambridge: The MIT Press, 2000.

\item Capurro, Rafael; Treude, Linda: Information Zeichen Kompetenz. Fragen an Rafael Capurro zu aktuellen und grundsätzlichen Fragen der Informationswissenschaft. In: Information Wissenschaft \& Praxis. (62) 1/2011. S. 37--42.

\item Cixous, Hélène: Manhattan. Schreiben aus der Vorgeschichte. Wien: Passagen, 2010.

\item Derrida, Jaques: Genesen, Genealogien, Genres und das Genie. Das Geheimnis des Archivs. Wien: Passagen, 2003.

\item Ernst, Wolfgang: Das Rumoren der Archive. Ordnung aus Unordnung. Berlin: Merve, 2002.

\item Esposito, Elena: Virtualisierung und Divination. Formen der Räumlichkeit der Kommunikation. In: Rudolf Maresch; Niels Weber: Raum-Wissen-Macht. Frankfurt am Main: Suhrkamp, 2002.

\item Esposito, Elena: Die Fiktion der wahrscheinlichen Realität. Aus dem Italienischen von Nicole Reinhardt. Frankfurt a. M.: Suhrkamp, 2007.

\item Foucault, Michel: Geometrie des Verfahrens. Schriften zur Methode. Frankfurt am Main: Suhrkamp, 2002.

\item Nafisi, Azar: Reading Lolita in Tehran. A Memoir in Books. New York: Random House, 2003.

\item Obst, Oliver: 2. Zukunftskolloquium der Zweigbibliothek Medizin der Universität Münster, 28./29. Juni 2010. GMS Medizin -- Bibliothek -- Information; 2010; 10(3):Doc31.  \url{https://dx.doi.org/10.3205/mbi000214}.

\item Obst, Oliver: \enquote{Bibliothek} wird zunehmend in Anführungszeichen
gesetzt. In: medinfo Weblog, 04.02.2011,
\url{http://medinfo.netbib.de/archives/2011/02/04/3869}.

\item Roth, Marco: Why Reading Only Matters When It's Somewhere Else. In: n+1. 1/2004: Negation.

\item Roth, Marco: Warum Literatur etwas bedeutet, wenn sie woanders stattfindet. In: Ein Schritt weiter. Die n+1-Anthologie. Frankfurt/Main:
Suhrkamp, 2008. S. 47--59.

\item Serres, Michel: Das eigent\emph{liche} Übel. Berlin: Merve, 2009.
\end{itemize}

\begin{center}\rule{0.5\linewidth}{0.5pt}\end{center}

%autor

\textbf{Ben Kaden} Wissenschaftlicher Mitarbeiter im Projekt IUWIS am
Institut für Bibliotheks- und Informationswissenschaft und
Mitherausgeber von LIBREAS.Library Ideas

% {\sloppy\printbibliography[heading=subbibliography,notkeyword=this]}
\end{document}
